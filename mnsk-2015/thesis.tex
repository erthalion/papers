\documentclass{article}
\usepackage{amsthm,amsfonts,amsmath,amssymb}
\usepackage[utf8]{inputenc}
\usepackage[english, russian]{babel}
\usepackage[final]{graphicx}
\textwidth 11.5cm \textheight 16cm

\begin{document}

\begin{flushright}
УДК 519.63    % вставить код УДК, наиболее точно характеризующий работу. См., например, http://pu.virmk.ru/doc/UDK/ 
\end{flushright}

\begin{center}
%НАЗВАНИЕ ДОКЛАДА
\textbf{МОДЕЛИРОВАНИЕ ДВИЖЕНИЯ ВЯЗКОЙ НЕОДНОРОДНОЙ ЖИДКОСТИ В КРУПНЫХ КРОВЕНОСНЫХ СОСУДАХ}
\\
\vspace{\baselineskip}
%АВТОР
Д. А. Долгов
\\
%ВУЗ
Кемеровский государственный университет
\end{center}
\vspace{\baselineskip}
%ТЕКСТ ТЕЗИСА

В последнее время интерес к математическому моделированию движения крови в искусственных сердечных клапанах человека существенно возрос, в связи с развитием новых методов лечения различных патологий сердечно сосудистой системы. В данной работе мы предлагаем новую математическую модель для описания динамики течения крови в крупных кровеносных сосудах и искусственном сердечном клапане, а также численный метод решения данной задачи. Исследование проводится совместно с НИИ КССЗ (Кемеровский кардиоцентр), в целях улучшения конструкции создаваемых искусственных клапанов.

Рассмотрим нестационарную задачу о течении крови внутри сосуда. Кровь состоит из плазмы и взвешенных в ней форменных элементов. Стенки сосуда являются гибкими и изменяют свою форму под воздействием течения крови. Будем моделировать кровь как вязкую несжимаемую двухкомпонентную жидкость, а стенки сосуда – как непроницаемую поверхность цилиндрической формы, обладающую заданной жесткостью. Задача о течении крови описывается нестационарной системой дифференциальных уравнений Навье-Стокса \cite{zaharov_miloshevich} с переменными вязкостью и плотностью. Т.к. физически кровь является неоднородной, то концентрацию примеси будем описывать уравнением конвекции \cite{zaharov_miloshevich}. Для моделирования динамики гибких стенок сосуда и створок искусственного сердечного клапана определяются силы, возвращающие их в равновесное положение \cite{boyce}.

Для решения полученной задачи воспользуемся методом погруженной границы \cite{boyce}. Влияние стенок сосуда и клапанов на течение будем учитывать с помощью добавления массовых сил в уравнение движения жидкости \cite{boyce}. Т.о. алгоритм решения будет следующим - на прямоугольной сетке с помощью схем расщепления по физическим факторам вычисляется значение скорости жидкости; затем решаем уравнение конвекции, т.е. определяем концентрацию примеси в области решения и пересчитываем значение плотности и вязкости. Далее вводим новую лагранжевую сетку, на которой определяем деформацию сосуда или створок клапана под воздействием движения жидкости, и вычисляем значение сил, противодействующих деформации. После этого находим новое распределение массовых сил в уравнении движения жидкости.

Полученная модель и численный метод решения были применены для задач развития аневризмы сосуда и течения крови в аортальном клапане. В рамках первой задачи были проведены расчеты, демонстрирующие возможность возникновения устойчивой аневризмы, а также ее влияние на распространение примеси. Для второй задачи получены результаты движения клапанов при различных перепадах давления.

\noindent\rule{50.0mm}{0.1mm}

\begin{enumerate}
\bibitem{zaharov_miloshevich}               % перед именем ссылки вставляется ФамилияИО докладчика, если несколько работ - то с номером
\textit{H. Milosevic, N.A. Gaydarov, Y.N. Zakharov} Model of incompressible viscous fluid flow driven by  pressure difference in a given channel // International Journal of Heat and Mass Transfer, vol. 62, July 2013.

\bibitem{boyce}
\textit{E.G. Boyce} Immersed boundary model of aortic heart valve dynamics with physiological driving and loading conditions // International Journal for Numerical Methods in Biomedical Engineering, (2011).
\end{enumerate}


\begin{flushright}
\textbf{\textit{Научный руководитель --- д-р физ.-мат. наук, проф.
Ю.Н.~Захаров}}

\end{flushright}
\end{document}
