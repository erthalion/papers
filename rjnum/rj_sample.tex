%%%%%%%%%%%%%%%%%%%%%%%%%%%%%%%%%%%%%%%%%%%%%%%%%%%%%%%%%%%%%%%%%%%%%%%%
%%%                                                                  %%%
%%% This is a sample LaTeX file of the article submitted to RJNAMM   %%%
%%%                                                                  %%%
%%% DO NOT EDIT AFTER THIS LINE %%%%%%%%%%%%%%%%%%%%%%%%%%%%%%%%%%%%%%%%
\documentclass[12pt,a4paper]{article}                                %%%
%%% Standard packages will be used by the Publisher:                 %%%
\usepackage{authblk,amsmath,amsfonts,amssymb,amsthm,graphicx}        %%%
% \usepackage[textwidth=205mm,textheight=240mm]{geometry}            %%%
%                                                                    %%%
%%% Special commands:                                                %%%
\newcommand\Jcomm[2]{\par\medskip\noindent{\bfseries #1: } #2\par}   %%%
\newcommand\keywords[1]{\Jcomm{Keywords}{#1}}                        %%%
\newcommand\received[1]{\Jcomm{Received}{#1}}                        %%%
\newcommand\revised[1]{\Jcomm{Revised}{#1}}                          %%%
\newcommand\classification[1]{\Jcomm{MSC 2010}{#1}}                  %%%
\newcommand\acknowledgement[1]{\Jcomm{Acknowledgement}{#1}}          %%%
\newcommand\funding[1]{\Jcomm{Funding}{#1}}                          %%%
%                                                                    %%%
%%% Theorem styles:                                                  %%%
\numberwithin{equation}{section}                                     %%%
\theoremstyle{plain}                                                 %%%
  \newtheorem{theorem}{Theorem}[section]                             %%%
  \newtheorem{lemma}{Lemma}[section]                                 %%%
  \newtheorem{proposition}{Proposition}[section]                     %%%
\theoremstyle{definition}                                            %%%
  \newtheorem{definition}{Definition}[section]                       %%%
  \newtheorem{remark}{Remark}[section]                               %%%
  \newtheorem{corollary}{Corollary}[section]                         %%%
  \newtheorem{example}{Example}%[section]                            %%%
%                                                                    %%%
%%% Greek letters:                                                   %%%
\let\epsilon=\varepsilon                                             %%%
\let\kappa=\varkappa                                                 %%%
\let\phi=\varphi                                                     %%%
\let\theta=\vartheta                                                 %%%
%                                                                    %%%
\date{} %%% don't want date printed                                  %%%
\renewcommand\Authands{, }                                           %%%
%%% DO NOT EDIT BEFORE THIS LINE %%%%%%%%%%%%%%%%%%%%%%%%%%%%%%%%%%%%%%%

%%% Specify russian language and russian encoding if required
% \usepackage[T1,T2A]{fontenc}
% \usepackage[utf8x]{inputenc} %%% specify cp866, cp1251, or koi8-r instead of utf8x if required
% \usepackage[english,russian]{babel}

\begin{document}

\title{Title of the article...}
\author[1]{Aname A. Afamily} %%% Specify Name and Family name of each author
\author[2]{Bname B. Bfamily}
\author[1,2]{Cname C. Cfamily}
\affil[1]{The first affiliation address...} %%% Specify the affiliation address of each author
\affil[2]{The second affiliation address...}
\affil[ ]{E-mail: Aaa@xyz.edu} %%% Specify email address if required

\maketitle

\abstract{Text of the abstract...} %%% Specify 3-5-7 phrases on your article
\keywords{Text of the keywords...} %%% Specify 3-5 keywords
\classification{65A01, 65B02,...}  %%% Specify 1-2-3 indices of MSC 2010 classification from http://www.ams.org/msc
\received{Date...} %%% Specify the first date of sending article to the Aditorial Board of the Journal
%\revised{Date...} %%% Specify the date of the article revision (if required)

\bigskip
Some non numerated introduction can be placed before the first section.
It should not contain any numerated formulae.

Please note, that the total volume of the article of no more that 20 pages is strictly recommented by the Editorial Board of the Journal.

\section{First section}\label{sec1}

Text of the first section...

New paragraph of the first section...

\subsection{First subsection}

Text of the first subsection...

The first equation:
\begin{equation}\label{eq1}
A x = b
\end{equation}
and the second one:
\begin{equation}\label{eq2}
\int_0^1 f dx = 0.
\end{equation}

The paragraph after equation with an indent.
The same equation in text mode: $\int_0^1 f dx = 0$.
The axample of bold mathematical symbols with \verb"\mbox\boldmath" command: $Ab\Sigma\Delta\alpha\beta\gamma\delta-\mbox{\boldmath${Bb\Delta\alpha\beta\gamma\delta+12345}$}+CcDd\cdot 12345$.

\begin{table}
\def\z{ABC}\def\T{ABCdef123\alpha\beta\gamma}
\caption{The example of different mathematical fonts.}\label{tab0}
\begin{tabular}{llc}
\hline
Mathematical symbols & How to type it \\ \hline
  $\T$                   & \verb"$...$"                   \\
  $\mathrm{\T}$          & \verb"$\mathrm{...}$"          \\
  $\mathit{\T}$          & \verb"$\mathit{...}$"          \\
  $\mathsf{\T}$          & \verb"$\mathsf{...}$"          \\
  $\mathtt{\T}$          & \verb"$\mathtt{...}$"          \\
  $\mathbf{\T}$          & \verb"$\mathbf{...}$"          \\
  $\mbox{\boldmath$\T$}$ & \verb"$\mbox{\boldmath$...$}$" \\
  $\mathcal{\z}$         & \verb"$\mathcal{...}$"         \\
  $\mathbb{\z}$          & \verb"$\mathbb{...}$"          \\
  $\mathfrak{\z}$        & \verb"$\mathfrak{...}$"        \\
\hline
\end{tabular}
\end{table}

The example of some mathematical symbols: $a\leq b\geq c$ (using `$\backslash$leq'
and `$\backslash$geq'), $d\neq e\ne f$ (using `$\backslash$neq').

\begin{definition}\label{def1}
Text of the definition typed in a normal font...
\end{definition}
Text after the definition...

\begin{theorem}\label{th1}
Text of the theorem typed in italic... With formulas \eqref{eq1}, \eqref{eq2} from \cite{b1,b2}
specified in Definition \ref{def1}.
\end{theorem}
Text after the theorem...

\begin{proof}
Text of the proof...
\end{proof}
Text after the proof...

\begin{figure}[t]
\framebox(200,100){\parbox{100\unitlength}{ The sample figure...}}
\caption{Text of the figure caption...}\label{fig1}
\end{figure}

\begin{table}
\caption{Text of the table caption...
The table should contain minimal number of horizontal and vertical lines...}\label{tab1}
\begin{tabular}{lcc}
\hline
Col1 & Col2 & Col3 \\ \hline
val1 & val2 & val3 \\
$1.1\cdot10^1$ & $2.2\cdot10^2$ & $3.3\cdot10^3$ \\
$ABCdef123\alpha\beta\gamma$ & $-\sum_{i=0}^n A^2_{\alpha+}$ & --- \\
\hline
\end{tabular}
\end{table}

For the problem formulation see equations (\ref{eq1})--(\ref{eq2}),
Section \ref{sec1}, Definition \ref{def1}, Theorem \ref{th1},
Figure \ref{fig1}, Table \ref{tab1}, and papers \cite{b1,b2,b3}.

\acknowledgement{Text of the acknowledgement...} %%% Specify text of acknowledgement to other persons if required...

\funding{The research was supported by...} %%% Specify the organizations and/or grants if required...

\bibliographystyle{plain}
\begin{thebibliography}{99} %%% The refences should be ordered by the authors
\bibitem{b1} A. Aaaa, {\em Heart disease and stroke statistics}. Publisher, City, 2001.
\bibitem{b2} B. Bbbb, Title of the article. {\em Title of the Journal} {\bf 22} (2002), 22-33.
\bibitem{b3} C. Cccc, Title of the article. In: {\em Proc. of the Conference.} 2003, pp. 301-333.
\end{thebibliography}

\end{document}
