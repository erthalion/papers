%\documentclass[14pt, notes]{beamer}
\documentclass[14pt]{beamer}

%\usepackage{pgfpages}
%\setbeameroption{show notes}
%\setbeameroption{show notes on second screen=right}

%encoding
\usepackage[utf8]{inputenc}

%language
\usepackage[russian]{babel}
\usepackage{amsmath}
\usepackage{bm}
\usepackage{graphicx}
\usepackage{hyperref}
\graphicspath{{images/}}%путь к рисункам

\setbeamerfont{author in head/foot}{size=\small}
\setbeamerfont{title in head/foot}{size=\footnotesize}

\title[Отчет: обзор литературы]{Обзор работ, посвященных численному и экспериментальному моделированию задач о течении в крупных кровеносных сосудах}
\date{\today}
\author[Долгов Д.А.]{Долгов Д.А.\\{\small Научный руководитель: Захаров Ю.Н.}}
\institute{Кемеровский Государственный Университет \\
    \vspace{0.7cm}
    \vspace{0.7cm}
} 
\usetheme[numbers, totalnumbers, minimal, nologo]{Statmod}
% Привычный шрифт для математических формул
\usefonttheme[onlymath]{serif}

\definecolor{statmodblue}{RGB}{100,10,30}
\definecolor{statmodsand}{RGB}{244,215,103}

\begin{document}
\maketitle

\begin{frame}
\frametitle{Содержание}
\begin{itemize}
    \item Монографии по биологической гидродинамике сосудов
    \item Работы, посвященные численному моделированию методом конечных элементов
    \item Работы, посвященные численному моделированию методом погруженной границы
\end{itemize}
\end{frame}
\note{Работы из первого раздела были написаны достаточно давно и являются "общеобразовательными"}

\begin{frame}
\frametitle{Биологическая гидродинамика}
\begin{itemize}
    \item Кардо К., Педли Т., Шротер Р., Сид У. Механика кровообращения. М.: Мир, 1981
    \item Педли Т. Гидродинамика крупных кровеносных сосудов. М.: Мир, 1983
\end{itemize}
\end{frame}
\note{Содержат общее биологическое описание + примеры, которые можно использовать в качестве тестовых}

\begin{frame}
\frametitle{Метод конечных элементов}
\begin{itemize}
    \item Zhang Y., Bajaj C., Finite element meshing for cardiac analysis // ICES Technical Reoprt 04-26, 2004
    \item Taylor C.A. Finite element modeling of blood flow: Relevance to atherosclerosis // Intra and Extracorporeal Cardiovascular Fluid Dynamics - Vol. 2, 249-289, 2000
\end{itemize}
\end{frame}
\note{Обе работы (и многие другие) посвящены построению и перестроению подходящих для расчета сеток}

\begin{frame}
\frametitle{Метод погруженной границы}
\begin{itemize}
    \item Классификация методов этого вида и их обзор
    \item Теоретическое обоснование и практическое применение оригинального метода
    \item Развитие оригинального метода
\end{itemize}
\end{frame}

\begin{frame}
\frametitle{Классификации и обзоры}
\begin{itemize}
    \item Mittal R., Iaccarino G., Immersed boundary methods // Annu. Rev. Fluid Mech. 2005. 37:239–61
    \item Bandriga H., Immersed boundary methods, Master Thesis in Applied Mathematics, 2010
    \item Мортиков Е.В., Применение метода погруженной границы для решения системы уравнений Навье-Стокса в областях сложной конфигурации // Вычислительные методы и программирование, 2010, Т. 11
\end{itemize}
\end{frame}

\begin{frame}
\frametitle{Теоретическое обоснование и практическое применение}
\begin{itemize}
    \item Peskin, C. S. The immersed boundary method // Acta Numerica 11, 479–517, 2002.
    \item Boyce E.G. Immersed boundary model of aortic heart valve dynamics with physiological driving and loading conditions // International Journal for Numerical Methods in Biomedical Engineering. 1–29, 2011
\end{itemize}
\end{frame}



\begin{frame}
\frametitle{Развитие оригинального метода}
\begin{itemize}
    \item Pilhwa L., Boyce E.G., Peskin C.S., The immersed boundary method for advection-electrodiffusion with implicit timestepping and local mesh refinement // Comput Phys. 2010 July 1; 229(13)
    \item Lai M., Tseng Y., Huang H., An immersed boundary method for interfacial flows with insoluble surfactant // Journal of Computational Physics. 01/2008; 
\end{itemize}
\end{frame}

\begin{frame}
\frametitle{Развитие оригинального метода}
\begin{itemize}
    \item Fai T.G., Boyce E.G., Mori Y., Peskin C.S. Immersed boundary method for variable viscosity and variable density problems using fast constant-coefficient linear solvers I: Numerical method and results. // SIAM Journal on Scientific Computing, 35(5):B1132–B1161, 2013
\end{itemize}
\end{frame}

\end{document}
