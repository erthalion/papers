\documentclass[12pt,a4paper,twoside]{article}
\usepackage{cmap}
\usepackage{amsmath}
\usepackage{amssymb}
\usepackage{graphicx}
\usepackage{fancyhdr}

% "Включение" русского языка
% cp866 - это "DOS-кодировка" текста, если Вы работаете
% с Win-кодировкой, то используйте
\usepackage[utf8]{inputenc}
%\usepackage[cp866]{inputenc}
\usepackage[russian]{babel}

\usepackage{nw2010}


\begin{document}

% Название доклада
\title{МОДЕЛИРОВАНИЕ ДВИЖЕНИЯ ВЯЗКОЙ НЕОДНОРОДНОЙ ЖИДКОСТИ В ИСКУССТВЕННОМ СЕРДЕЧНОМ КЛАПАНЕ}

% Автор(ы)
% Если несколько - отделять запятой
\author{Д.\,А. Долгов$^1$, Ю.\,Н. Захаров$^1$}

% Место работы: "Институт, город"
% Если несколько - отделять новой строкой: \\ (два обратных слеша)
\inst{$^1$Кемеровский государственный университет}

% Добавление в "Содержание"
% В первых фигурных скобках - автор(ы), во вторых - название
\add{Долгов Д.\,А., Захаров Ю.\,Н.}{Моделирование движения вязкой неоднородной жидкости в искусственном сердечном клапане}

% Добавление в "Авторский указатель"
% Каждый автор должен быть добавлен отдельно
\index{Долгов Д.\,А.} \index{Захаров Ю.\,Н.}


%  -----  Т Е З И С Ы  -----

В данной работе мы предлагаем математическую модель, которая описывает динамику течения крови в искусственном сердечном клапане, а также численный метод решения этой задачи. 
Мы рассматриваем нестационарную задачу о течении крови внутри сосуда к клапаном. Кровь состоит из плазмы и взвешенных в ней форменных элементов. Клапан и стенки сосуда являются гибкими и изменяют свою форму под воздействием течения крови. Будем моделировать кровь как вязкую несжимаемую двухкомпонентную жидкость, а клапан и стенки сосуда – как непроницаемую поверхность, обладающую заданной жесткостью. Для описания динамики створок искусственного сердечного клапана и гибких стенок сосуда мы определяем силы, возвращающие их в равновесное положение \cite{Griffith:Article}. Задача о течении крови описывается нестационарной системой дифференциальных уравнений Навье-Стокса \cite{Zakharov_Milosevic:Article} с переменными вязкостью и плотностью. Т.к. физически кровь является неоднородной, то концентрацию примеси будем описывать уравнением конвекции \cite{Zakharov_Milosevic:Article}.

Полученную задачу мы решаем с помощью метода погруженной границы \cite{Griffith:Article}. Влияние клапанов на течение будем учитывать с помощью добавления массовых сил в уравнение движения жидкости \cite{Griffith:Article}. Т.о. алгоритм решения будет следующим - на прямоугольной сетке с помощью схем расщепления по физическим факторам вычисляется значение скорости жидкости; затем решаем уравнение конвекции, т.е. определяем концентрацию примеси в области решения и пересчитываем значение плотности и вязкости. Далее вводим новую лагранжевую сетку, на которой определяем деформацию створок клапана под воздействием движения жидкости, и вычисляем значение сил, противодействующих деформации. После этого находим новое распределение массовых сил в уравнении движения жидкости.

Полученная модель и численный метод решения были применены для задач о течении крови в искусственном аортальном клапане, а также о развитии аневризмы сосуда. В рамках первой задачи получены результаты движения клапанов при различных перепадах давления. Для второй задачи были проведены расчеты, демонстрирующие возможность возникновения аневризмы стенок сосуда, а также ее влияние на распространение примеси.

Исследование проводится совместно с НИИ КССЗ (Кемеровский кардиоцентр), в целях улучшения конструкции создаваемых искусственных клапанов в рамках проектной части госзадания номер 1.630.1.2014/К.

\begin{thebibliography}{9}
    \bibitem{Griffith:Article} Griffith B.\,E. {\em Immersed boundary model of aortic heart valve dynamics with physiological driving and loading conditions.} International Journal for Numerical Methods in Biomedical Engineering, 28(3) 2012, 317-345.

	\bibitem{Zakharov_Milosevic:Article} Miloshevich H. Gaydarov N.\,A. Zakharov Y.\,N. {\em Model of incompressible viscous fluid flow driven by  pressure difference in a given channel.} International Journal of Heat and Mass Transfer, vol. 62, July 2013.

\end{thebibliography}
\end{document}
