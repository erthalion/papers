\documentclass[14pt, compress, aspectratio=169]{beamer}

% can be compiled by xelatex -shell-escape presentation.tex

\usetheme[usetitleprogressbar]{m}

\usepackage[utf8]{inputenc}
\usepackage[russian, english]{babel}
\usepackage{booktabs}
\usepackage[scale=2]{ccicons}
\usepackage{listings}
\usepackage{marvosym}
\usepackage{color}
\usepackage{xcolor}
\usepackage[document]{ragged2e}
\usepackage[export]{adjustbox}
\usepackage{fontawesome}
\usepackage{enumitem}
\usepackage{minted}
\usemintedstyle{tango}
%\usemintedstyle{monokai}

\usetikzlibrary{shapes,arrows,positioning}
\graphicspath{{images/}}
\newfontfamily{\FA}{FontAwesome}

\definecolor{check}{rgb}{0.1,2,0.3}

\def\twitter{{\FA \faTwitter}}
\def\github{{\FA \faGithubSign}}
\def\email{{\FA \faEnvelope}}
\def\check{\textcolor{check}{\FA \faCheck}}
\def\fail{\textcolor{fail}{\FA \faRemove}}
\def\question{\textcolor{question}{\FA \faSearch}}

\renewcommand{\ttdefault}{pcr}
\newfontfamily{\ttfamily}{Fira Code}

\definecolor{links}{HTML}{0099FF}
\hypersetup{colorlinks, linkcolor=, urlcolor=links}

\title{FP в Python}
\subtitle{это проще, чем вы думали}
\date{\today}
%\author{Дмитрий Долгов}
\institute{}
%\titlegraphic{\vspace{120pt}\hfill\includegraphics[height=1.5cm]{logo.png}}

\begin{document}

\maketitle

\section{}

\begin{frame}
    \frametitle{Причины?}
    \vspace{-35pt}
    \begin{figure}
        \includegraphics[width=0.8\textwidth,center]{first_option.png}
    \end{figure}
\end{frame}

\begin{frame}
    \frametitle{Причины?}
    \vspace{-35pt}
    \begin{figure}
        \includegraphics[width=0.8\textwidth,center]{second_option.png}
    \end{figure}
\end{frame}

\begin{frame}[fragile]
    \frametitle{Причины?}
    \begin{itemize}[label={\MVRightarrow}]
        \item FP - это парадигма, не привязанная к языку
        \item Python - мультипарадигменный язык, позволяющий писать функционально
        \item Это позволяет увидеть преимущества FP в вашем коде уже завтра
    \end{itemize}
\end{frame}


\section{Краткое введение в FP}

\begin{frame}[fragile]
    \frametitle{Термины и определения}
    \begin{itemize}[label={\MVRightarrow}]
        \item Неизменяемость данных (immutability)
        \item Чистые функции и side effects
        \item Функции высшего порядка
        \item Монады (?)
    \end{itemize}
\end{frame}

\begin{frame}
    \frametitle{Неизменяемость}
    Функциональные программы обычно оперируют \\
    неизменяемыми данными. Вместо изменения оригинального\\
    значения, изменяется его копия. Та часть данных,
    которая сохранилась и в оригинальном объекте и в его копии,
    переиспользуется, что предотвращает излишний расход памяти.
\end{frame}

\begin{frame}
    \frametitle{Чистые функции и side effects}
    A function or expression is said to have a side effect if it modifies some
    state or has an interaction with the outside world. For example, a
    particular function might modify a global variable, modify one of its
    arguments, raise an exception, perform IO operation, or call other
    side-effecting functions.
\end{frame}

\begin{frame}
    \frametitle{Разделение данных и логики}
    Abstract Data Type (ADT)
\end{frame}

\begin{frame}
    \frametitle{Function composition}
    Function composition is the act of pipelining the result of one function,
    to the input of another, creating an entirely new function.
\end{frame}

\begin{frame}
    \frametitle{Монады}
\end{frame}

\section{FP parts in Python}

\begin{frame}
    \frametitle{What we can use in Python}
    \vspace{-25pt}
    \begin{itemize}[label={\MVRightarrow}]
        \item Immutable data structures:
            \begin{itemize}
                \item string
                \item tuple/namedtuple
                \item fronzenset
            \end{itemize}
        \item Higher-order functions
        \item List comprehension
        \item Generators (another way to rewrite recursion)
        \item itertools
        \item functools
    \end{itemize}
\end{frame}

\begin{frame}
    \frametitle{What we can't}
    \begin{itemize}[label={\MVRightarrow}]
        \item Tail recursion
        \item Pure functions
        \item Pattern matching \check
        \item Automatic currying \check
        \item Monads \check
        \item ADT \check
    \end{itemize}
\end{frame}

\begin{frame}
    \frametitle{Стратегии}
    \begin{itemize}[label={\MVRightarrow}]
        \item Использование <<чистого>> Python
        \item Собственные надстройки
        \item Сторонние библиотеки
    \end{itemize}
\end{frame}

\section{Examples}

\setbeamercolor{background canvas}{bg=white}
\begin{frame}[fragile]
    \setbeamertemplate{section page}
    \frametitle{}
    \begin{center}
    %\vspace{-25pt}
    \inputminted[
        fontsize=\scriptsize\ttfamily,
        %bgcolor=black,
    ]{python}{examples/immutable.py}
    \end{center}
\end{frame}

\begin{frame}[fragile]
    \setbeamertemplate{section page}
    \frametitle{}
    \begin{center}
    %\vspace{-25pt}
    \inputminted[
        fontsize=\scriptsize\ttfamily,
        %bgcolor=black,
    ]{python}{examples/list_comprehension.py}
    \inputminted[
        fontsize=\scriptsize\ttfamily,
        %bgcolor=black,
    ]{haskell}{examples/list_comprehension.hs}
    \end{center}
\end{frame}

\begin{frame}[fragile]
    \setbeamertemplate{section page}
    \frametitle{}
    \begin{center}
    %\vspace{-25pt}
    \inputminted[
        fontsize=\scriptsize\ttfamily,
        %bgcolor=black,
    ]{python}{examples/list_comprehension_chain.py}
    \end{center}
\end{frame}

\begin{frame}[fragile]
    \setbeamertemplate{section page}
    \frametitle{}
    \begin{center}
    %\vspace{-25pt}
    \inputminted[
        fontsize=\scriptsize\ttfamily,
        %bgcolor=black,
    ]{python}{examples/generators.py}
    \end{center}
\end{frame}

\begin{frame}[fragile]
    \setbeamertemplate{section page}
    \frametitle{}
    \begin{center}
    %\vspace{-25pt}
    \inputminted[
        fontsize=\scriptsize\ttfamily,
        %bgcolor=black,
    ]{python}{examples/maybe_example.py}
    \end{center}
\end{frame}

\begin{frame}[fragile]
    \setbeamertemplate{section page}
    \frametitle{}
    \begin{center}
    %\vspace{-25pt}
    \inputminted[
        fontsize=\scriptsize\ttfamily,
        %bgcolor=black,
    ]{python}{examples/recursion.py}
    \end{center}
\end{frame}

\begin{frame}[fragile]
    \setbeamertemplate{section page}
    \frametitle{}
    \begin{center}
    %\vspace{-25pt}
    \inputminted[
        fontsize=\scriptsize\ttfamily,
        %bgcolor=black,
    ]{python}{examples/recursion_fp.py}
    \end{center}
\end{frame}

\begin{frame}[fragile]
    \setbeamertemplate{section page}
    \frametitle{}
    \begin{center}
    %\vspace{-25pt}
    \inputminted[
        fontsize=\scriptsize\ttfamily,
        %bgcolor=black,
    ]{python}{examples/curry_example.py}
    \end{center}
\end{frame}

\begin{frame}[fragile]
    \setbeamertemplate{section page}
    \frametitle{}
    \begin{center}
    %\vspace{-25pt}
    \inputminted[
        fontsize=\scriptsize\ttfamily,
        %bgcolor=black,
    ]{python}{examples/example1.py}
    \end{center}
\end{frame}

\begin{frame}[fragile]
    \setbeamertemplate{section page}
    \frametitle{}
    \begin{center}
    %\vspace{-25pt}
    \inputminted[
        fontsize=\scriptsize\ttfamily,
        %bgcolor=black,
    ]{python}{examples/example1_fp.py}
    \end{center}
\end{frame}
\note{http://programmers.stackexchange.com/questions/150837/maybe-monad-vs-exceptions}

\setbeamercolor{background canvas}{bg=palette primary.bg}

\section{Библиотеки}

\begin{frame}
    \frametitle{Список FP библиотек для Python}
    \vspace{-25pt}
    \begin{itemize}[label={\MVRightarrow}]
        \item fn.py \url{github.com/kachayev/fn.py}
        \item toolz \url{github.com/pytoolz/toolz}
        \item PyMonad \url{github.com/fnl/pymonad}
        \item PyFunctional \url{github.com/EntilZha/PyFunctional}
        \item Coconat \url{github.com/evhub/coconut}
        \item pyrsistent \url{github.com/tobgu/pyrsistent}
        \item effect \url{github.com/python-effect/effect}
        \item adt \url{github.com/llllllllll/adt}
    \end{itemize}
\end{frame}

\setbeamercolor{background canvas}{bg=white}
\begin{frame}[fragile]
    \setbeamertemplate{section page}
    \frametitle{}
    \begin{center}
    %\vspace{-25pt}
    \inputminted[
        fontsize=\scriptsize\ttfamily,
        %bgcolor=black,
    ]{python}{examples/fn.py}
    \end{center}
\end{frame}

\begin{frame}[fragile]
    \setbeamertemplate{section page}
    \frametitle{}
    \begin{center}
    %\vspace{-25pt}
    \inputminted[
        fontsize=\scriptsize\ttfamily,
        %bgcolor=black,
    ]{python}{examples/pymonad.py}
    \end{center}
\end{frame}

\begin{frame}[fragile]
    \setbeamertemplate{section page}
    \frametitle{}
    \begin{center}
    %\vspace{-25pt}
    \inputminted[
        fontsize=\scriptsize\ttfamily,
        %bgcolor=black,
    ]{python}{examples/pyfunctional.py}
    \end{center}
\end{frame}

\begin{frame}[fragile]
    \setbeamertemplate{section page}
    \frametitle{}
    \begin{center}
    %\vspace{-25pt}
    \inputminted[
        fontsize=\scriptsize\ttfamily,
        %bgcolor=black,
    ]{python}{examples/pyrsistent.py}
    \end{center}
\end{frame}
\setbeamercolor{background canvas}{bg=palette primary.bg}

\plain{Questions?}

\end{document}
