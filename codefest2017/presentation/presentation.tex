\documentclass[18pt, compress, aspectratio=169]{beamer}

% can be compiled by xelatex -shell-escape presentation.tex

\usetheme[]{m}

\usepackage[utf8]{inputenc}
\usepackage[russian, english]{babel}
\usepackage{booktabs}
\usepackage[scale=2]{ccicons}
\usepackage{listings}
\usepackage{marvosym}
\usepackage{color}
\usepackage{xcolor}
\usepackage[document]{ragged2e}
\usepackage[export]{adjustbox}
\usepackage{fontawesome}
\usepackage{enumitem}
\usepackage{minted}
\usemintedstyle{tango}
\usepackage[normalem]{ulem}
%\usemintedstyle{monokai}

\usetikzlibrary{shapes,arrows,positioning}
\graphicspath{{images/}}
\newfontfamily{\FA}{FontAwesome}

\definecolor{check}{rgb}{0.1,2,0.3}
\definecolor{fail}{rgb}{2,0.1,0.1}
\definecolor{question}{rgb}{0.9,0.9,0.0}

\def\twitter{{\FA \faTwitter}}
\def\github{{\FA \faGithubSign}}
\def\email{{\FA \faEnvelope}}
\def\check{\textcolor{check}{\FA \faCheck}}
\def\fail{\textcolor{fail}{\FA \faRemove}}
\def\question{\textcolor{question}{\FA \faSearch}}

\renewcommand{\ttdefault}{pcr}
\newfontfamily{\ttfamily}{Fira Code}
\makeatletter
\newcommand\HUGE{\@setfontsize\Huge{32}{41}}
\makeatother

\renewcommand{\ULthickness}{2.0pt}

\definecolor{links}{HTML}{0099FF}
\hypersetup{colorlinks, linkcolor=, urlcolor=links}

\setbeamerfont{section title}{family=\Book, size=\Huge, shape=\normalfont}
\setbeamerfont{frametitle}{family=\Book, size=\large, shape=\normalfont}
\setbeamerfont{title}{family=\Book, size=\HUGE, shape=\normalfont}
\setbeamerfont{subtitle}{size=\LARGE}
\setbeamerfont{author}{family=\ExtraLight, size=\normalsize}
\usebackgroundtemplate{\includegraphics[width=\paperwidth]{slide_background.png}}

\setbeamertemplate{title page}
{
  \vspace*{3.1cm}
  \begin{columns}[T,onlytextwidth]
  \column{0.3\textwidth}
  \column{0.7\textwidth}
  \begin{minipage}[b][\paperheight]{\textwidth}

    \ifx\inserttitle\@empty\else
    {{% \inserttitle is nonempty
      \raggedright%
      \linespread{1.0}%
      \usebeamerfont{title}%
      \usebeamercolor[fg]{title}%
      \if@noSmallCapitals%
        \inserttitle%
      \else%
        \scshape\MakeLowercase{\inserttitle}%
      \fi%
      \vspace*{0.3em}
    }}
    \fi

    \ifx\insertsubtitle\@empty\else
    {{% \insertsubtitle is nonempty
      \usebeamerfont{subtitle}%
      \usebeamercolor[fg]{subtitle}%
      \insertsubtitle%
      \vspace*{3.5em}%
    }}
    \fi

    \begin{columns}[T,onlytextwidth]
    \column{0.32\textwidth}
      \usebeamerfont{author}%
      \usebeamercolor[fg]{author}%
      \insertauthor%
      \vspace*{0.5em}%
    \column{0.32\textwidth}
      \usebeamerfont{author}%
      \usebeamercolor[fg]{author}%
      Senior Software Engineer
      \vspace*{0.5em}%
    \column{0.32\textwidth}
      \usebeamerfont{author}%
      \usebeamercolor[fg]{author}%
      Zalando
      \vspace*{0.5em}%
    \end{columns}

    \vfill
    \vspace*{2em}
  \end{minipage}
  \end{columns}
}

\setbeamertemplate{section page}
{
  \vspace{2em}
  \centering
  \begin{minipage}{22em}
    \usebeamercolor[fg]{section title}
    \usebeamerfont{section title}
    \insertsectionHEAD\\[-1ex]
  \end{minipage}
  \par
}

\setbeamertemplate{footline}
{
\begin{beamercolorbox}[wd=\textwidth,ht=3ex,dp=3ex,leftskip=0.3cm,rightskip=0.3cm]{structure}
  \usebeamerfont{page number in head/foot}
  \insertframenumber
\end{beamercolorbox}
}

\title{NoSQL внутри SQL}
\subtitle{тактика и стратегия}
\date{\today}
\author{Дмитрий\\ Долгов}
\institute{}

\begin{document}
{
  \usebackgroundtemplate{\includegraphics[width=\paperwidth]{title_background.png}}%
  \fontsize{17pt}{18}\selectfont
  \maketitle
}

\fontsize{21pt}{23}\selectfont
\section{}

\begin{frame}[fragile]
    \frametitle{}
    \begin{center}
        \textbf{Данные}
    \end{center}
    \begin{itemize}[leftmargin=*]
        \item <+->
    \end{itemize}

    \vspace{-40pt}

    \begin{columns}[T,onlytextwidth]
    \column{0.5\textwidth}
    \begin{itemize}[leftmargin=*]
        \item <+->\includegraphics[width=6cm,height=5cm]{relation.png}
    \end{itemize}

    \vspace{20pt}

    \column{0.5\textwidth}
    \begin{itemize}[leftmargin=*]
        \item <+->\includegraphics[width=6cm,height=5cm]{document.jpg}
    \end{itemize}
    \end{columns}
\end{frame}

\begin{frame}
    \frametitle{}
    \begin{center}
        \textbf{Данные нужно хранить в соответствующем формате:}
        \pause
        \begin{itemize}[label={\MVRightarrow}]
            \item <+-> Отдельные хранилища,\\ единый интерфейс
            \item <+-> Единое хранилище,\\ разные форматы
        \end{itemize}
    \end{center}
\end{frame}

\begin{frame}
    \frametitle{}
    \begin{center}
        \textbf{Отдельные хранилища}
        \pause
        \begin{itemize}[label={\MVRightarrow}]
            \item <+-> Конкретный формат обрабатывается наилучщим образом \check
            \item <+-> Производительность, дублирование \question
            \item <+-> Вопросы интеграции компонентов \fail
        \end{itemize}
    \end{center}
\end{frame}

\begin{frame}
    \frametitle{}
    \begin{center}
        \textbf{Единое хранилище}
        \pause
        \begin{itemize}[label={\MVRightarrow}]
            \item <+-> Не требует интеграции \check
            \item <+-> Производительность, дублирование \question
            \item <+-> Поддержка со стороны БД \question
        \end{itemize}
    \end{center}
\end{frame}
\note{
    если данные разного формата не сравнимы по объему, затраты на интеграцию
    и инфраструктуру могут не окупиться.
}

\begin{frame}
    \frametitle{}
    \begin{center}
    %\vspace{-10pt}
    \begin{figure}
        \includegraphics[width=0.9\textwidth,center]{cat_stories.jpg}
    \end{figure}
    \end{center}
\end{frame}

\begin{frame}
    \frametitle{}
    \begin{center}
        \textbf{Кто?}
        \begin{itemize}[label={\MVRightarrow}]
            \item Postgresql (hstore/json/jsonb)
            \item MySQL (json)
            \item Oracle
            \item MSSql
            \item db2
        \end{itemize}
    \end{center}
\end{frame}

\fontsize{13pt}{14}\selectfont
\begin{frame}
    \frametitle{}
    \vspace{2em}
    \centering
    \begin{minipage}{32em}
        \usebeamercolor[fg]{section title}
        \usebeamerfont{section title}
        Легкий способ начать \sout{бегать по утрам} использовать документы в реляционной базе
    \end{minipage}
\end{frame}
\fontsize{17pt}{18}\selectfont

\begin{frame}
    \frametitle{}
    \begin{center}
    \inputminted[
        fontsize=\Large,
    ]{sql}{sql/json_build.sql}
    \end{center}
\end{frame}

\begin{frame}
    \frametitle{}
    \begin{center}
    \inputminted[
        fontsize=\Large,
    ]{sql}{sql/json_agg.sql}
    \end{center}
\end{frame}

\begin{frame}
    \frametitle{}
    \begin{center}
    \inputminted[
        fontsize=\Large,
    ]{sql}{sql/load.sql}
    \end{center}
\end{frame}

\begin{frame}
    \frametitle{}
    \begin{center}
        \begin{itemize}[label={\MVRightarrow}]
            \item Загрузка дампа из внешних источников
            \item Некорректные данные с валидной структурой -- json5
            \item Битые данные -- ручное исправление, линтеры
        \end{itemize}
    \end{center}
\end{frame}

\fontsize{13pt}{14}\selectfont
\section{Производительность}
\fontsize{17pt}{18}\selectfont

\begin{frame}
    \frametitle{}
    \begin{center}
        \textbf{Факторы}
        \pause
        \begin{itemize}[label={\MVRightarrow}]
            \item <+-> Структура данных на диске
            \item <+-> Сериализация данных
            \item <+-> Поддержка индексов
        \end{itemize}
    \end{center}
\end{frame}
\note{
    оптимизация по размеру и пробеганию
}

\begin{frame}
    \frametitle{}
    \begin{center}
    \textbf{Bson}
    \begin{figure}
        \includegraphics[width=1.0\textwidth,center]{bson.png}
    \end{figure}
    \end{center}
\end{frame}

\begin{frame}
    \frametitle{}
    \begin{center}
    \inputminted[
        fontsize=\Large,
    ]{python}{sql/bson.py}

    \begin{figure}
        \includegraphics[width=1.0\textwidth,center]{bson_binary.png}
    \end{figure}

    \end{center}
\end{frame}

\begin{frame}
    \frametitle{}
    \begin{center}
    \textbf{Jsonb}
    \begin{figure}
        \includegraphics[width=1.0\textwidth,center]{jsonb.png}
    \end{figure}
    \end{center}
\end{frame}

\begin{frame}
    \frametitle{}
    \begin{center}
    \inputminted[
        fontsize=\Large,
    ]{python}{sql/jsonb_binary.sql}

    \begin{figure}
        \includegraphics[width=1.0\textwidth,center]{jsonb_binary.png}
    \end{figure}

    \end{center}
\end{frame}

\begin{frame}
    \frametitle{}
    \begin{center}
    \textbf{MySQL json}
    \begin{figure}
        \includegraphics[width=1.0\textwidth,center]{mysql_json.png}
    \end{figure}
    \end{center}
\end{frame}

\begin{frame}
    \frametitle{}
    \textbf{Сериализация данных}
    \begin{center}
        \begin{itemize}[label={\MVRightarrow}]
            \item MongoDB -- дерево Document -> Elements
            \item Postgresql -- JsonbValue со списком элементов
            \item MySQL -- древовидная структура
        \end{itemize}
    \end{center}
\end{frame}

\begin{frame}
    \frametitle{}
    \textbf{Индексы}
    \begin{center}
        \begin{itemize}[label={\MVRightarrow}]
            \item MongoDB -- индексы для полей
            \item Postgresql -- общий индекс, индексы для полей
            \item MySQL -- виртуальные колонки для индексирования
        \end{itemize}
    \end{center}
\end{frame}

\fontsize{13pt}{14}\selectfont
\section{Тестирование}
\fontsize{17pt}{18}\selectfont

\begin{frame}
    \frametitle{}
    \begin{center}
    %\vspace{-10pt}
    \begin{figure}
        \includegraphics[width=0.8\textwidth,center]{great_performance.jpg}
    \end{figure}
    \end{center}
\end{frame}

\begin{frame}
    \frametitle{}
    \begin{center}
        \begin{itemize}[label={}]
            \item YCSB 0.8, $10^{6}$
            \item Postgresql 9.5.4
            \item MongoDB 3.2.9
            \item MySQL 5.7.9
            \item AWS EC2 m4.xlarge
            \item 16GB memory, 4 core 2.3GHz
        \end{itemize}
    \end{center}
\end{frame}

\begin{frame}
    \frametitle{}
    \begin{center}
        \textbf{Воспроизводимость}
        \begin{itemize}[label={}]
            \item \href{https://github.com/erthalion/YCSB}{erthalion/YCSB}
            \item \href{https://github.com/erthalion/ansible-ycsb}{erthalion/ansible-ycsb}
        \end{itemize}
    \end{center}
\end{frame}

\begin{frame}
    \frametitle{}
    \begin{center}
        \textbf{Конфигурация}
        \begin{itemize}[label={}]
            \item shared\_buffers
            \item effective\_cache\_size
            \item innodb\_buffer\_pool\_size
            \item write concern
            \item transaction\_sync, method
        \end{itemize}
    \end{center}
\end{frame}

\begin{frame}
    \frametitle{}
    \begin{center}
        \textbf{Простая выборка по ключу с jsonb\_path\_ops индексом}
        \begin{itemize}[label={}]
            \item "Маленький документ"
            \item 10 полей
            \item без вложенности
        \end{itemize}
    \end{center}
\end{frame}

\begin{frame}
    \frametitle{}
    \begin{center}
    %\vspace{-10pt}
    \begin{figure}
        \includegraphics[width=0.75\textwidth,center]{benchmarks/workload_c_jsonb_path_ops/throughput.png}
    \end{figure}
    \end{center}
\end{frame}

\begin{frame}
    \frametitle{}
    \begin{center}
    %\vspace{-10pt}
    \begin{figure}
        \includegraphics[width=0.75\textwidth,center]{benchmarks/workload_c_jsonb_path_ops/latency_99.png}
    \end{figure}
    \end{center}
\end{frame}

\begin{frame}
    \frametitle{}
    \begin{center}
        \textbf{Простая выборка по ключу с Btree индексом}
        \begin{itemize}[label={}]
            \item "Маленький документ"
            \item 10 полей
            \item без вложенности
        \end{itemize}
    \end{center}
\end{frame}

\begin{frame}
    \frametitle{}
    \begin{center}
    %\vspace{-10pt}
    \begin{figure}
        \includegraphics[width=0.75\textwidth,center]{benchmarks/workload_c_btree/throughput.png}
    \end{figure}
    \end{center}
\end{frame}

\begin{frame}
    \frametitle{}
    \begin{center}
    %\vspace{-10pt}
    \begin{figure}
        \includegraphics[width=0.75\textwidth,center]{benchmarks/workload_c_btree/latency_99.png}
    \end{figure}
    \end{center}
\end{frame}

\begin{frame}
    \frametitle{}
    \begin{center}
        \textbf{Простая выборка по ключу с Btree индексом}
        \begin{itemize}[label={}]
            \item "Сложный документ"
            \item 3 уровня вложенности/4 потомка
        \end{itemize}
    \end{center}
\end{frame}

\begin{frame}
    \frametitle{}
    \begin{center}
    %\vspace{-10pt}
    \begin{figure}
        \includegraphics[width=0.75\textwidth,center]{benchmarks/workload_c_complex_object/throughput.png}
    \end{figure}
    \end{center}
\end{frame}

\begin{frame}
    \frametitle{}
    \begin{center}
    %\vspace{-10pt}
    \begin{figure}
        \includegraphics[width=0.75\textwidth,center]{benchmarks/workload_c_complex_object/latency_99.png}
    \end{figure}
    \end{center}
\end{frame}

\begin{frame}
    \frametitle{}
    \begin{center}
        \textbf{Срез по документу}
        \begin{itemize}[label={}]
            \item "Большой документ"
            \item 100 полей
            \item Из документа выбирается одно поле
        \end{itemize}
    \end{center}
\end{frame}

\begin{frame}
    \frametitle{}
    \begin{center}
    %\vspace{-10pt}
    \begin{figure}
        \includegraphics[width=0.75\textwidth,center]{benchmarks/workload_c_select_one/throughput.png}
    \end{figure}
    \end{center}
\end{frame}

\begin{frame}
    \frametitle{}
    \begin{center}
        \textbf{Срез по документу}
        \begin{itemize}[label={}]
            \item "Большой документ"
            \item 100 полей
            \item Из документа выбирается 10 полей
        \end{itemize}
    \end{center}
\end{frame}

\begin{frame}
    \frametitle{}
    \begin{center}
    %\vspace{-10pt}
    \begin{figure}
        \includegraphics[width=0.75\textwidth,center]{benchmarks/workload_c_select_slice/throughput.png}
    \end{figure}
    \end{center}
\end{frame}

\begin{frame}
    \frametitle{}
    \begin{center}
    %\vspace{-10pt}
    \begin{figure}
        \includegraphics[width=0.9\textwidth,center]{hack.jpg}
    \end{figure}
    \end{center}
\end{frame}

%\begin{frame}
    %\frametitle{}
    %\begin{center}
    %\vspace{-10pt}
    %\textbf{Простая выборка по ключу с индексом}
    %\begin{figure}
        %\includegraphics[width=0.7\textwidth,center]{benchmarks/simple_select_max_latency.png}
    %\end{figure}
    %\end{center}
%\end{frame}

\begin{frame}
    \frametitle{}
    \begin{center}
    %\vspace{-10pt}
    \begin{figure}
        \includegraphics[width=0.95\textwidth,center]{benchmarks/table_size.png}
    \end{figure}
    \end{center}
\end{frame}

%\begin{frame}
    %\frametitle{}
    %\begin{center}
    %\vspace{-10pt}
    %\textbf{Размер индексов}
    %\begin{figure}
        %\includegraphics[width=0.7\textwidth,center]{benchmarks/index_size.png}
    %\end{figure}
    %\end{center}
%\end{frame}

%\begin{frame}
    %\frametitle{}
    %\begin{center}
    %\vspace{-10pt}
    %\textbf{Выборка документов с большим количеством ключей}
    %\end{center}
%\end{frame}
%\note{
    %сериализация в ширину
%}

%\begin{frame}
    %\frametitle{}
    %\begin{center}
    %\vspace{-10pt}
    %\textbf{Выборка документов с большой вложенностью}
    %\end{center}
%\end{frame}
%\note{
    %сериализация в глубину
%}

%\begin{frame}
    %\frametitle{}
    %\begin{center}
    %\vspace{-10pt}
    %\textbf{Выборка среза по документам}
    %\end{center}
%\end{frame}
%\note{
    %множественный detoast
%}

%\begin{frame}
    %\frametitle{}
    %\begin{center}
    %\vspace{-10pt}
    %\textbf{Выборка документов без кэша}
    %\end{center}
%\end{frame}
%\note{
    %моделирование ситуации, когда данные не влезают в память
    %и идет активная работа с диском
%}

%\begin{frame}
    %\frametitle{}
    %\begin{center}
    %\vspace{-10pt}
    %\textbf{SET STORAGE EXTERNAL}
    %\end{center}
%\end{frame}
%\note{
    %хранение jsonb в распакованном формате
%}

\begin{frame}
    \frametitle{}
    \begin{center}
        \textbf{Вставка документов}
        \begin{itemize}[label={}]
            \item "Маленький документ"
            \item 10 полей
            \item без вложенности
            \item journaled
        \end{itemize}
    \end{center}
\end{frame}

\begin{frame}
    \frametitle{}
    \begin{center}
    %\vspace{-10pt}
    \begin{figure}
        \includegraphics[width=0.75\textwidth,center]{benchmarks/load_data/throughput.png}
    \end{figure}
    \end{center}
\end{frame}

\begin{frame}
    \frametitle{}
    \begin{center}
        \textbf{Выборка 50\%, обновление 50\%}
        \begin{itemize}[label={}]
            \item "Маленький документ"
            \item 10 полей
            \item без вложенности
            \item обновление одного поля
            \item transaction\_sync
        \end{itemize}
    \end{center}
\end{frame}

\begin{frame}
    \frametitle{}
    \begin{center}
    %\vspace{-10pt}
    \begin{figure}
        \includegraphics[width=0.75\textwidth,center]{benchmarks/workload_a_mongo_config/throughput.png}
    \end{figure}
    \end{center}
\end{frame}

\begin{frame}
    \frametitle{}
    \begin{center}
    %\vspace{-10pt}
    \begin{figure}
        \includegraphics[width=0.75\textwidth,center]{benchmarks/workload_a_mongo_config/latency_99.png}
    \end{figure}
    \end{center}
\end{frame}

\begin{frame}
    \frametitle{}
    \begin{center}
        \textbf{Выборка 50\%, обновление 50\%}
        \begin{itemize}[label={}]
            \item "Маленький документ"
            \item 10 полей
            \item без вложенности
            \item обновление одного поля
            \item journaled
        \end{itemize}
    \end{center}
\end{frame}

\begin{frame}
    \frametitle{}
    \begin{center}
    %\vspace{-10pt}
    \begin{figure}
        \includegraphics[width=0.75\textwidth,center]{benchmarks/workload_a_journaled/throughput.png}
    \end{figure}
    \end{center}
\end{frame}

\begin{frame}
    \frametitle{}
    \begin{center}
        \textbf{Выборка 50\%, обновление 50\%}
        \begin{itemize}[label={}]
            \item "Большой документ"
            \item 100 полей удвоенной длины
            \item обновление одного поля
            \item без вложенности
        \end{itemize}
    \end{center}
\end{frame}

\begin{frame}
    \frametitle{}
    \begin{center}
    %\vspace{-10pt}
    \begin{figure}
        \includegraphics[width=0.75\textwidth,center]{benchmarks/workload_a_large_document/throughput.png}
    \end{figure}
    \end{center}
\end{frame}

\begin{frame}
    \frametitle{}
    \begin{center}
    %\vspace{-10pt}
    \begin{figure}
        \includegraphics[width=0.75\textwidth,center]{benchmarks/workload_a_large_document/latency_99.png}
    \end{figure}
    \end{center}
\end{frame}

%\begin{frame}
    %\frametitle{}
    %\begin{center}
    %\vspace{-10pt}
    %\textbf{Вставка записей}
    %\end{center}
%\end{frame}

%\begin{frame}
    %\frametitle{}
    %\begin{center}
    %\vspace{-10pt}
    %\textbf{Вставка и обновление}
    %\end{center}
%\end{frame}

%\begin{frame}
    %\frametitle{}
    %\begin{center}
    %\vspace{-10pt}
    %\textbf{Чтение последних}
    %\end{center}
%\end{frame}

%\begin{frame}
    %\frametitle{}
    %\begin{center}
    %\vspace{-10pt}
    %\textbf{Чтение, изменение, запись}
    %\end{center}
%\end{frame}

\usebackgroundtemplate{\includegraphics[width=\paperwidth]{title_background.png}}%
\fontsize{17pt}{18}\selectfont
\begin{frame}
  \vspace*{2.5cm}
  \begin{columns}[T,onlytextwidth]
  \column{0.3\textwidth}
  \column{0.7\textwidth}
  \begin{minipage}[b][\paperheight]{\textwidth}

      \raggedright%
      \linespread{1.0}%
      \usebeamerfont{title}%
      \usebeamercolor[fg]{title}%
      \if@noSmallCapitals%
        Вопросы?
      \else%
        \scshape\MakeLowercase{Вопросы?}%
      \fi%
      \vspace*{0.3em}

      \usebeamerfont{subtitle}%
      \fontsize{13pt}{14}\selectfont
      \usebeamercolor[fg]{subtitle}%
        \begin{itemize}[label={}]
            \item {\github\ github.com/erthalion}
            \item {\twitter\ @erthalion}
            \item \email\ 9erthalion6 at gmail dot com
        \end{itemize}
      \vspace*{2.5em}%

    \begin{columns}[T,onlytextwidth]
    \column{0.32\textwidth}
      \usebeamerfont{author}%
      \usebeamercolor[fg]{author}%
      \insertauthor%
      \vspace*{0.5em}%
    \column{0.32\textwidth}
      \usebeamerfont{author}%
      \usebeamercolor[fg]{author}%
      Senior Software Engineer
      \vspace*{0.5em}%
    \column{0.32\textwidth}
      \usebeamerfont{author}%
      \usebeamercolor[fg]{author}%
      Zalando
      \vspace*{0.5em}%
    \end{columns}

    \vfill
    \vspace*{2em}
  \end{minipage}
  \end{columns}

\end{frame}

\end{document}
