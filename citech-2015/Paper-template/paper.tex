
%%%%%%%%%%%%%%%%%%%%%%% file typeinst.tex %%%%%%%%%%%%%%%%%%%%%%%%%
%
% This is the LaTeX source for the instructions to authors using
% the LaTeX document class 'llncs.cls' for contributions to
% the Lecture Notes in Computer Sciences series.
% http://www.springer.com/lncs       Springer Heidelberg 2006/05/04
%
% It may be used as a template for your own input - copy it
% to a new file with a new name and use it as the basis
% for your article.
%
% NB: the document class 'llncs' has its own and detailed documentation, see
% ftp://ftp.springer.de/data/pubftp/pub/tex/latex/llncs/latex2e/llncsdoc.pdf
%
%%%%%%%%%%%%%%%%%%%%%%%%%%%%%%%%%%%%%%%%%%%%%%%%%%%%%%%%%%%%%%%%%%%


\documentclass[runningheads,a4paper]{llncs}

\usepackage{amssymb, amsmath}
\setcounter{tocdepth}{3}
\usepackage{graphicx}


\usepackage{url}
\urldef{\mailsa}\path|9erthaion6@gmail.com, zaxarovyn@rambler.ru|
\newcommand{\keywords}[1]{\par\addvspace\baselineskip
\noindent\keywordname\enspace\ignorespaces#1}

\begin{document}

\mainmatter  % start of an individual contribution

% first the title is needed
\title{Numerical simulation of the artificial heart valve dynamics}

% a short form should be given in case it is too long for the running head

% the name(s) of the author(s) follow(s) next
%
% NB: Chinese authors should write their first names(s) in front of
% their surnames. This ensures that the names appear correctly in
% the running heads and the author index.
%
\author{D.A. Dolgov \and Y.N. Zakharov}
%
\authorrunning{CITech-2015, Almaty, Kazakhstan, 2015 September 24-27}
% (feature abused for this document to repeat the title also on left hand pages)

% the affiliations are given next; don't give your e-mail address
% unless you accept that it will be published
\institute{Kemerovo State University,\\
Kemerovo, Russia\\
\mailsa}

%
% NB: a more complex sample for affiliations and the mapping to the
% corresponding authors can be found in the file "llncs.dem"
% (search for the string "\mainmatter" where a contribution starts).
% "llncs.dem" accompanies the document class "llncs.cls".
%

\toctitle{Lecture Notes in Computer Science}
\tocauthor{Authors' Instructions}
\maketitle


\begin{abstract}
    In this paper we consider a mathematical model for describing the dynamics
    of the artificial heart valve, which is moved by the viscous inhomogeneous
    incompressible fluid flow with variable viscosity, and also a method
    of the numerical simulation of this problem. We present the results of modeling
    for the heart valve with three leaflets.
\keywords{viscous inhomogeneous fluid, artificial heart valve, immersed boundary method}
\end{abstract}


\section{Introduction}
It is difficult to overestimate the importance of medical researches of the human blood circulatory system,
because the knowledge from this area are extremely practical and significant. Each year, approximately 250 000
procedures are performed in the world to repair or replace damaged heart valves \cite{yoganathan}, and this
value tends to increace \cite{yacoub}. The solution of scientific and technical problems of the artificial valves 
creation depends on correct understanding of the fluid flow interaction with valve leaflets. Mathematical modeling
of the dynamics of artificial heart valves allows to get more complete understanding of processes inside them and helps
to find the ways to improve their design. There are many researches devoted to the mathematical and numerical modeling
of the dynamics of the heart valve. Most of them can be divided into two large groups.

First group is related to the finite element methods (\cite{taylor}, \cite{zhang}, \cite{black}). They may well handle
the complex geometry of the heart, bu the necessity to take into account the interaction between the fluid and flexible walls
leads to permanent rebuilding of the computational grid to meet the changing geometry of the object of research.
This results to significant expenses of time and computational resourses.

In this paper we consider a second approach, which is related to the immersed boundary method (\cite{pescin_1977},
\cite{boyce_2011}, \cite{ma_x_2013}, \cite{pilhwa_2010}). It can be used for the problems with complex geometry, and it doesn't
required the grid modification.

There are various improvements of this method, because of the need to model more and more complex problems. In the research \cite{fai_2013}
a formulation of this method was proposed for the three dimensional problem of the flow of two nonmixing (separated by flexible barries)
fluids of different viscosity and density. In the papers \cite{jian}, \cite{lee} an application of this method for the two dimensional problem of
the two component fluid flow.

In this work we propose to describe the blood flow in the flexible large blood vessels and the artificial heart valve as a three dimensional
nonstationary flow of the viscout incompressible fluid with variable viscosity and density (see \cite{gummel}, \cite{geidarov},
\cite{milosevic}, \cite{dolgov}). Thus the goal of this work is to build a mathematical model and a solution method of the problem
of artificial heart leaflet dynamics inside a blood vessel in view of the inhomogeneous structure of the blood, and also about the
mixture (formed elements) motion inside vessel.

\section{Mathematical formulation of the problem}

We consider a nonstationary problem of blood flow inside vessel with valve. Blood consists of the plasma and formed elements, which are approximately
45\% of the entire volume \cite{caro}. Vessel walls and valve leaflets consists of the large number of thin collagen fibers, they are flexibe
and can change their form depending on the fluid flow. The aortic valve, which are placed at the outlet from the left ventricle to the aorta, and provide
one-way movement of the blood, can be an example of this kind of systems (see Fig. \ref{fig:aortic_valve_example}):

\begin{figure}
\centering
\includegraphics[height=6.2cm]{images/aorta_scheme_gray.png}
\caption{Aortic valve and its location inside heart}
\label{fig:aortic_valve_example}
\end{figure}

We model the blood as a viscous incompressible inhomogeneous two component fluid with variable viscosity, and vessel wall and valve leaflets as a
liquid impermeable surface with specified stiffness. Vessel and valve leaflets are deformed by a fluid pressure.

\begin{figure}
\centering
\includegraphics[width=12.5cm]{images/area_3d.png}
\caption{Scheme of the boundaries of the computational domain}
\label{fig:area_3d}
\end{figure}

Because the source of the blood motion in vessels is the pressure during the cardiac cycle, we will describe the problem of the blood flow
by the following nonstationary system of differential equations of Navier-Stokes \cite{gummel}:
\begin{gather}
    \label{eq:navier_stokes:motion}
    \frac{\partial \vec{u}}{\partial t} + (\vec{u} \cdot \nabla) \vec{u} = - \frac{1}{\rho} \nabla p + \nabla \sigma + \vec{f}\\
    \label{eq:navier_stokes:continuity}
    \frac{\partial \rho}{\partial t} + \nabla \cdot (\rho \vec{u}) = 0 
\end{gather}

with the initial and boundary conditions:
\begin{gather}
    \label{eq:navier_stokes:velocity_conditions}
    \vec{u}(\bar{x}, 0) = \vec{u}_0 \qquad \vec{u}|_{\Gamma_1, \Gamma_4} = \vec{u}_b \qquad u_{\Gamma_2, \Gamma3} = 0\\
    \label{eq:navier_stokes:pressure_conditions}
    p_{\Gamma_2} = p_{in} \qquad p_{\Gamma_3} = p_{out}
\end{gather}

where $\bar{x}=(x,y,z) \in \Omega$, $\vec{u}=(u,v,w)$ - velocity vector, $\vec{u}_b$ - velocity of the vessel wall and valve leaflet motion during the deformation,
$\rho=\rho(\bar{x}, t)$ - density, $p=p(\bar{x}, t)$ - pressure, $\sigma = \mu (\nabla \vec{u} + (\nabla \vec{u})^T)$ - viscous stress tensor,
$\mu = \mu(\bar{x}, t)$ - viscosity of fluid, $\vec{f} = \vec{f}(\bar{x}, t)$ - vector of body forces, which is futher used to determine form of the vessel and valve leaflets.
Domain $\Omega$ is a vessel with boundary $\Gamma = \Gamma_1 \cup \Gamma_2 \cup \Gamma_3 \cup \Gamma_4$, where $\Gamma_1$ - vessel wall, $\Gamma_2$ and $\Gamma_3$ - intel and outlet, $\Gamma_4$ - valve leaflets (see Fig. \ref{fig:area_3d}).
As shown in \cite{ragulin}, the problem (\ref{eq:navier_stokes:motion}) - (\ref{eq:navier_stokes:continuity}) has a unique solution.

Density $\rho$ and viscosity $\mu$ are defined by following relations \cite{gummel}:
\begin{gather}
    \label{eq:viscosity}
    \mu = c (\mu_2 - \mu_1) + \mu_1\\
    \label{eq:density}
    \rho = c (\rho_2 - \rho_1) + \rho_1
\end{gather}

where $\rho_1$, $\mu_1$ - density and viscosity if fluid (plasma), $\rho_2$, $\mu_2$ - density and viscosity of mixture (formed elements), $c$ - concentration of mixture. Concentration $c=c(\bar{x}, t)$, $c \in [0, 1]$ of mixture are determined as a solution of equation:
\begin{gather}
    \label{eq:convection}
    \frac{\partial c}{\partial t} + \vec{u} \cdot \nabla c = 0
\end{gather}

with initial conditions:
\begin{gather}
    \label{eq:convection:conditions}
    c(\bar{x}, 0) = c_0(\bar{x}), \bar{x} \in \Omega
\end{gather}

and boundary conditions at the inlet boundary:
\begin{gather}
    \label{eq:convection:conditions}
    c(\bar{x}, t)|_{\Gamma_2} = c_s(\bar{x}, t)
\end{gather}

where $c_0, c_s$ are predefined functions.

Lack of definition for one component of velocity vector at the inlet-outlet is the one of issues for computational solution of this kind of problems.
It can be solved by the using the original equations (\ref{eq:navier_stokes:motion}) - (\ref{eq:navier_stokes:pressure_conditions}) at the boundaries
$\Gamma_2$, $\Gamma_3$ for determine of missing components of the velocity vector (see details \cite{gummel}).

Motion of the vessel walls and valve leaflets is defined by forces, which returned it to the original position. Valve leaflets can be deformated much more,
than vessel walls. To describe the forces, arising due the valve deformation, we use following formula:

\begin{gather}
    \label{eq:boundary_force}
    F = \frac{\partial}{\partial s} (T \tau) + \frac{\partial^2}{\partial s^2} (E \cdot I \frac{\partial^2}{\partial s^2} X)
\end{gather}

where $\bar{q} = (q, r, s) \in \Gamma_4$, $X(\bar{q})$ - function for describing the surface of leaflets at the moment $t$, the coordinates $q, r, s$ chosen so
that the surface $X$ was presented by the large amount of parametric lines $s \rightarrow X(q^0, r^0, s)$, $T$ - tension, that arises due the stretching along $s$,
$E$ - Young's modulus, $I$ - cross-sectional moment of inertia (see \cite{boyce_2011}, \cite{pescin_2002}). Physically the formula above means,
that the valve leaflets are resist extension, compression (it's related to the first term with $T$, which is dependent on stiffness coefficient $k$),
and bending (this related to the second term, where $E$ and $I$ are referred as a stiffness coefficient $k_b$). Formula (\ref{eq:boundary_force}) allows to
take into account any changes of the valve shape.

To compture forces, arising due the deformation of the vessel, we use another formula, which allows to take into account only small changes of the shape:

\begin{gather}
    \label{eq:boundary_force_simple}
    F = k \|X - X_0\|
\end{gather}

where $\bar{q} = (q, r, s, t) \in \Gamma_1$, $X(\bar{q}, t)$, $X_0(\bar{q}, 0)$ - functions for describing the surface of vessel walls at the moment $t$ and at the initial time, $k$ - stiffness coefficient.

As shown in the works \cite{pescin_1977}, \cite{boyce_2011}, for modeling of the interaction between vessel walls, valve leaflets and the fluid flow we need
to compute the vector of body forces $f$ in the equation of Navier-Stokes, based on the force $F$, and determine current surface $X(\bar{q}, t)$ of the vessel and valve, based on the velocity vector field $\vec{u}(\bar{x}, t)$. This is done by using the following equations:
\begin{gather}
    \label{eq:interaction:velocity}
    \frac{\partial X}{\partial t}(\bar{q}, t) = \int_{\Omega} \vec{u}(\bar{x}, t) \cdot \delta (x - X(\bar{q}, t))\; dx\; dy\; dz\\
    \label{eq:interaction:force}
    \vec{f}(\bar{x}, t) = \int_{\Gamma} \vec{F}(\bar{q}, t) \cdot \delta (x - X(\bar{q}, t))\; dq\; dr\; ds
\end{gather}

where $\bar{q} = (q, r, s) \in \Gamma$ - point at the vessel wall or valve leaflet, $X = X(\bar{q}, t)$ - function for describing the surface
of vessel wall and valve leaflet at the momen $t$, $F = F(\bar{q}, t)$ - the force of resistance to deformation,
$\vec{u}(\bar{x}, t)$ - fluid velocity vector, $\vec{f}(\bar{x}, t)$ - vector of body forces, $\delta$ - Dirac delta function.

Thus we build the model, which describes the motion of the viscous inhomogeneous incompressible fluid inside vessel with valve. In this model the fluid state
and the surface $\Gamma_1 \cup \Gamma_4$ are determined independently from each other, and the influence of the valve leaflets on the fluid is reflected by
relation (\ref{eq:interaction:force}) between the vector of body forces $\vec{f}(\bar{x}, t)$ from the equation (\ref{eq:navier_stokes:motion}) and the force of resistance to deformation $F = F(\bar{q}, t)$ from the equations (\ref{eq:boundary_force}), (\ref{eq:boundary_force_simple}).

\section{Method of solution}

As it was mentioned before, in this work we using the immersed boundary method \cite{pescin_1977}, which are based on that when fluid flows over a body,
it exerts a normal force on the surface, and if the surface is no-slip, the fluid also exerts a shear force. The surface exerts the same force of opposite sign.
This means that fluid flow around a body can be modeled by a corresponding field of the external body forces \cite{goldstain}.

Accordingly to the immersed boundary method, we will determine the fluid flow in the parallelepiped $\tilde{\Omega}$, which contains $\Omega$.
No-splip conditions are imposed at the boundaries if $\tilde{\Omega}$.
For the computation of the fluid flow we will use rectangular uniform staggered grid $\tilde{\Omega_h}$ with grid spacing $h_x$, $h_y$, $h_z$ and 
staggered arrangement of cells, where the pressure, velocity divergence and concentration are computed at the center of cell, the velocity vector components
and vector of external forces - at the boundaries of cell. To determine the deformation of the surface $\Gamma_1 \cup \Gamma_4$ we will introduce additional
area $\tilde{\Gamma}$ with Lagrangian coordinate system, which is related to the vessel walls and valve leaflets. In the $\tilde{\Gamma}$ we will construct
a new grid $\tilde{\Gamma_h}$, which cells are corresponding to the points at the $\Gamma_1 \cup \Gamma_2$. Algorithm of solution consists of the several steps:
at the grid $\tilde{\Gamma_h}$ we will solve the problem (\ref{eq:navier_stokes:motion})-(\ref{eq:navier_stokes:pressure_conditions}); then we will solve
the convection equation (\ref{eq:convection}), i.e. determine the concentration of mixture and recalculate the density and viscosity. After it we will use
the formulas (\ref{eq:boundary_force}), (\ref{eq:boundary_force_simple}) and (\ref{eq:interaction:velocity}), (\ref{eq:interaction:force}) to determine position of
leaflets and the vessel form.

Differential equation (\ref{eq:boundary_force}), (\ref{eq:convection:conditions}) is solved by the finite difference method.
To solve (\ref{eq:boundary_force}), (\ref{eq:boundary_force_simple}) we will use the splitting on physical factors scheme \cite{belotserkovsky}:
\begin{gather}
    \label{eq:splitting:intermediate_velocity}
    \frac{u^* - u^n}{\triangle t} = - (u^n \cdot \nabla) u^* - \frac{1}{\rho} \nabla \sigma + f^n\\
    \label{eq:splitting:poisson}
    \rho \triangle p^{n+1} - \nabla \rho \cdot p^{n+1} = \frac{\rho^2 \nabla u^*}{\triangle t}\\
    \label{eq:splitting:velocity}
    \frac{u^{n+1} - u^*}{\triangle t} = - \frac{1}{\rho} \triangle p^{n+1}
\end{gather}

Computational implementation of this scheme consists of three stages. At the beginning the intermediate field $u^*$  is computed from the known values of velocity
from the previous time step. For this equation (\ref{eq:splitting:intermediate_velocity}) is solved by the method of stabilizing corrections \cite{yanenko}.
After it a new pressure field is determined via the computational solution of (\ref{eq:splitting:poisson}) with the biconjugate gradient method usage.
And at the last stage a final velocity vector field is calculated by the formula (\ref{eq:splitting:velocity}).

After fluid flow parameters determining it is necessary to calculate new values of density and velocity. To do that a new time step for the convection
equation (\ref{eq:convection}) must be done using the obtained values of velocity components, and the density and viscosity are recalculated
by the formulas (\ref{eq:viscosity}), (\ref{eq:density}).

Next we need to determine the deformation of vessel walls and valve leaftets under influence of fluid flow, and also the distribution of body forces $f$
in the equation of fluid motion basen on this deformation. We can calculate the deformation of vessel walls and valve leaflets at this particular fluid pressure
and the appearing resistance forces using the equations (\ref{eq:interaction:velocity}) - (\ref{eq:interaction:force}), which are numerically integrated by
the any of quadrature formulas, and equations (\ref{eq:boundary_force}) - (\ref{eq:boundary_force_simple}). After it we recalculate the body forces $f$
and go to the next time step.

\section{Results}

We present several results of methodical calculations for the cases of constant and variable density and viscosity, whose purpose is to demonstrate the operability
of the described method and the possibility to get with its help the patterns of  leaflet deformation and mixture distribution inside the valve.
All calculations were performed in dimensionless variables. As a vessel, in which the valve was located, a circular cylinder with length $l = 1$,
radius $r = 0.11$ and wall stiffness $k = 1 \cdot 10^3$ was used, the domain $\tilde{\Omega}$ had the spatial parameters $1.0 \times 0.5 \times 0.5$,
spatial steps $h_x = h_y = h_k = 0.01$, time step $\triangle t = 0.01$.

The dynamics of valve with three leaflets under the influence of the fluid pressure with constans density and velosity is shown in Fig. \ref{fig:valve} and Fig. \ref{fig:valve_with_particles}.
The pressure differential $p_{in} - p_{out}$ changes periodically from 0 to 6. Coefficient of stretching resistance $k_b = 5 \cdot 10^3$ and coefficient of
bending resistance $k_b = 5 \cdot 10^3$ are specified for the valve leaflets.


\begin{figure}
\centering
\includegraphics[height=6.2cm]{images/valve_1_gray.png}

$a$

\includegraphics[height=6.2cm]{images/valve_2_gray.png}

$b$

\includegraphics[height=6.2cm]{images/valve_3_gray.png}

$c$

\caption{Dynamics of the Valve leaflets. Current leaflet shape is indicated by points, flow direction - by arrows.
It shows a side view (I), front view (II) and rear view (III). $k_s = 5 \cdot 10^3$, $k_b = 5 \cdot 10^3$, $\rho_1 = \rho_2 = 1$,
$\mu_1 = \mu_2 = 1 \cdot 10^{-2}$; a) $t=0$, b) $t=0.7$, c) $t=1.5$}
\label{fig:valve}
\end{figure}

\begin{figure}
\centering
\includegraphics[height=6.2cm]{images/valve_with_particles_1_gray.png}

$a$

\includegraphics[height=6.2cm]{images/valve_with_particles_2_gray.png}

$b$

\includegraphics[height=6.2cm]{images/valve_with_particles_3_gray.png}

$c$

\caption{Tracks of particles inside the valve. Flow direction is indicated by arrows. Calculation parameters are the same as for the Fig. \ref{fig:valve}
It shows a side view (I) and front view (II). $k_s = 5 \cdot 10^3$, $k_b = 5 \cdot 10^3$, $\rho_1 = \rho_2 = 1$,
$\mu_1 = \mu_2 = 1 \cdot 10^{-2}$; a) $t=0$, b) $t=0.7$, c) $t=1.5$}

\label{fig:valve_with_particles}
\end{figure}

As can be seen from the Fig. \ref{fig:valve} and Fig. \ref{fig:valve_with_particles}, the valve opens when pressure differential is increased,
and then returns to the original state when pressure is equalized.

The Fig. \ref{fig:valve_in_mixture} shows the dynamics of the valve with three leaflets under the fluid pressure with variable viscosity and
density. Pressure differential $p_{in} - p_{out}$ changes cyclically from 0 to 6. Coefficient of stretching resistance $k_s = 8 \cdot 10^3$ and coefficient of
bending resistance $k_s = 6 \cdot 10^3$ are specified for the valve leaflets. At the $\Gamma_2$ a constant mixture flow is imposed with concentration $c_s = 0.45$.

\begin{figure}
\centering
\includegraphics[height=6.2cm]{images/valves_in_mixture_gray_scale_400.png}

$a$

\includegraphics[height=6.2cm]{images/valves_in_mixture_gray_scale_500.png}

$b$

\includegraphics[height=6.2cm]{images/valves_in_mixture_gray_scale_600.png}

$c$

\caption{Valve leaflets motion with variable viscosity and density. A constant mixture flow $c_s|_{\Gamma_2} = 0.45$ is imposed at the inlet,
concentration of mixture at the initial moment $c_0 = 0.45$, $\rho_1=1$, $\rho_2=1.2$, $\mu_1 = 1 \cdot 10^2$, $\mu_2 = 1.2 \cdot 10^2$;
a) $t = 4$, b) $t=5$, c) $t=6$}
\label{fig:valve_in_mixture}
\end{figure}

As you can seen from the Fig. \ref{fig:valve_in_mixture}, initial uniform distribution of the mixture is disturbed by the leaflets motion. Over time
mixture motion begins to have an oscillatory character corresponding to the cycles of valve dynamics. Besides that it can bee seen,
that the mixture distribution along the section parallel to the axis $Oy$ is not symmetric, because the valve leaflets also are not symmetric about the axis $Oy$.

\section{Conclusion}

Constructed model of blood flow with variable viscosity and density allows to get the patterns of leaflet deformation and mixture distribution 
under the influence of flow inhomogeneous fluid.

\begin{thebibliography}{4}

\bibitem{yoganathan} Yoganathan A.P., He Z.M., Jones S.C.: Fluid mechanics of heart valves. Annu. Rev. Biomed Eng 6:331--362 (2004)

\bibitem{yacoub} Yacoub N, Takkenberg J.: Will heart valve tissue engineering change the world? Nat Clin Prac Cardiovas Med. 2:60--1 (2005)

\bibitem{taylor} Taylor C.A., Hughes T.J.R., Zarins C.K.: Finite Element Modeling of Blood Flow in Arteries.
Computer Methods in Applied Mechanics and Engineering, vol. 158, 155--196, (1998)

\bibitem{zhang} Zhang Y, Bajaj C.: Finite element meshing for cardiac analysis. ICES Technical Report, 4--26 (2004)

\bibitem{black} Black M.M., Howard I.C., Huang X., Patterson E.A.: A three-dimensional analysis of a bioprosthetic heart valve. J. Biomech 24(9), 793--801 (1991)

\bibitem{pescin_1977} Peskin C.S.: Numerical Analysis of Blood Flow in the Heart. JCP 25, 220--252 (1977)

\bibitem{boyce_2011} Boyce E.G.: Immersed boundary model of aortic heart valve dynamics with physiological driving and loading conditions. International Journal for Numerical Methods in Biomedical Engineering. 1-29 (2011)

\bibitem{ma_x_2013} Ma X., Gao H., Boyce E.G., Berry C., Luo X.: Image-based fluid–structure interaction model of the human mitral valve. Computers \& Fluids 71, 417–425 (2013)

\bibitem{pilhwa_2010} Pilhwa L., Boyce E.G., Peskin C.S.: The immersed boundary method for advection-electrodiffusion with implicit timestepping and local mesh refinement. Comput Phys. 229(13), (2010)

\bibitem{fai_2013} Fai T.G., Boyce E.G., Mori Y., Peskin C.S.: Immersed boundary method for variable viscosity and variable density problems using fast constant-coefficient linear solvers I: Numerical method and results. SIAM Journal on Scientific Computing, 35(5), B1132–B1161, (2013)

\bibitem{jian} Jian D., Robert D.G., Aaron L.F., An immersed boundary method for twofluid mixtures// Journal of Computational Physics, Volume 262, 231--243, (2014)

\bibitem{lee} Lee P., Boyce E.G., Peskin C.S.: The immersed boundary method for advection-electrodiffusion with implicit timestepping and local mesh refinement. Journal of computational physics, Volume 229, 5208-5227, (2010)

\bibitem{gummel} Gummel E.E., Milosevic H., Ragulin V.V., Zakharov Y.N., Zimin A.I.: Motion of viscous inhomogeneous incompressible fluid of variable viscosity. Zbornik radova konferencije MIT 2013, Beograd, 267-274 (2014)

\bibitem{geidarov} Geidarov N.A., Zakharov Y. N.,  Shokin Yi. I.: Solution of the problem of viscous fluid flow with a given pressure differential. Russian Journal of Numerical Analysis and Mathematical Modeling, V.26, No 1, pp. 39--48 (2011)

\bibitem{milosevic} Milosevic H., Gaydarov N. A., Zakharov Y. N.: Model of incompressible viscous fluid flow driven by  pressure difference in a given channel. International Journal of Heat and Mass Transfer, vol. 62, July 2013 ISSN: 0017-9310, pp. 242--246, (2013)

\bibitem{dolgov} Dolgov D. A., Zakharov Y. N.: Modeling of viscous inhomogeneous fluid flow in large blood vessels. Vestnik Kemerovo State University, 2 (62) T.1, pp. 30--35 (2015)

\bibitem{caro} Caro C. G., Pedley Т. J., Schroter R. C., Seed W. A. The Mechanics of the Circulation. Moscow: Mir, p. 624 (1981)

\bibitem{ragulin} Ragulin V.V. To the problem of flow viscous fluid through the limited area under given pressure differential. Dynamic of continuum: Novosibirsk, Vol 27. p. 78--92 (1976)

\bibitem{pescin_2002} Peskin, C. S.: The immersed boundary method. Acta Numerica 11, 479–517 (2002).

\bibitem{goldstain} Goldstein D., Handler R., Sirovich L.: Modeling a no-slip flow boundary with an external force field. Journal of computational physics, 105, 354-366 (1993)

\bibitem{belotserkovsky} Belotserkovskii O. M.: Numerical modelingin mechanics of continuum. Moscow: Science, p. 520 (1984)

\bibitem{yanenko} Yanenko N. N.: Method of fractional steps for solving multidimensional problems of mathematical physics. Novosibirsk: Science, p. 197 (1967)

\end{thebibliography}

\end{document}
