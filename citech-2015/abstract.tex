\documentclass[11pt]{amsart}
\usepackage{latexsym}
\usepackage{amsmath,amsthm,amssymb,amsfonts}
\usepackage[english,russian]{babel}
\usepackage[utf8]{inputenc}

\thispagestyle{empty}  

\allowdisplaybreaks


\setlength{\textwidth}{16cm} \setlength{\oddsidemargin}{0.2cm}
\setlength{\evensidemargin}{0.2cm} \setlength{\topmargin}{0.1cm}
\setlength{\footskip}{0.5cm}

\linespread{1}

 
\renewcommand{\refname}{Bibliography}



\newtheorem{theorem}{Theorem}
\newtheorem{remark}{Remark}
\newtheorem{lemma}[theorem]{Lemma}
\newtheorem{proposition}[theorem]{Proposition}
\newtheorem{corollary}[theorem]{Corollary}
\newtheorem{definition}[theorem]{Definition}



\begin{document}




%%%%%%%%%%%%%%%%%%%%%%%%%%%%%%%%%%%%%%%%%%%%%%%%%%%%%%%%%%%%%%%%%%%%%%%%%%%%%%%%%%%%%%%%%%%
%%%%%%%%%%%%%%%%%%%%%%%%%%%%%%%%%%%%%%%%%%%%%%%%%%%%%%%%%%%%%%%%%%%%%%%%%%%%%%%%%%%%%%%%%%%
%               Insert here your own definitions
%%%%%%%%%%%%%%%%%%%%%%%%%%%%%%%%%%%%%%%%%%%%%%%%%%%%%%%%%%%%%%%%%%%%%%%%%%%%%%%%%%%%%%%%%%%
%%%%%%%%%%%%%%%%%%%%%%%%%%%%%%%%%%%%%%%%%%%%%%%%%%%%%%%%%%%%%%%%%%%%%%%%%%%%%%%%%%%%%%%%%%%

\newcommand{\RR}{\mathbb{R}}
\newcommand{\NN}{\mathbb{N}}

%%%%%%%%%%%%%%%%%%%%%%%%%%%%%%%%%%%%%%%%%%%%%%%%%%%%%%%%%%%%%%%%%%%%%%%%%%%%%%%%%%%%%%%%%%%


 \vspace{0.5cm}


%%%%%%%%%%%%%%%%%%%%%%%%%%%%%%%%%%%%%%%%%%%%%%%%%%%%%%%%%%%%%%%%%%%%%%%%%%%%%%%%%%%%%%%%%%%
%%%%%%%%%%%%%%%%%%%%%%%%%%%%%%%%%%%%%%%%%%%%%%%%%%%%%%%%%%%%%%%%%%%%%%%%%%%%%%%%%%%%%%%%%%%
%    Please insert here: first name(s) and surname(s) of the author(s)
%    and the abbreviated addresses of the authors - each abstract should include: 
%    the name of: the University or the Research Institute, City, Country and an email address 
%    of at least one author.
%
%%%%%%%%%%%%%%%%%%%%%%%%%%%%%%%%%%%%%%%%%%%%%%%%%%%%%%%%%%%%%%%%%%%%%%%%%%%%%%%%%%%%%%%%%%%
%%%%%%%%%%%%%%%%%%%%%%%%%%%%%%%%%%%%%%%%%%%%%%%%%%%%%%%%%%%%%%%%%%%%%%%%%%%%%%%%%%%%%%%%%%%
%
%               Please underline the first name and the surname of the speaker.
%%%%%%%%%%%%%%%%%%%%%%%%%%%%%%%%%%%%%%%%%%%%%%%%%%%%%%%%%%%%%%%%%%%%%%%%%%%%%%%%%%%%%%%%%%%
%%%%%%%%%%%%%%%%%%%%%%%%%%%%%%%%%%%%%%%%%%%%%%%%%%%%%%%%%%%%%%%%%%%%%%%%%%%%%%%%%%%%%%%%%%%

$\blacksquare$ {\bf Долгов Д.А.} {\small email: {\tt 9erthalion6@gmail.com}}, 
{\bf Захаров Ю.Н.}\\


\begin{center}
{\textit {\large Моделирование движения вязкой неоднородной жидкости в крупных кровеносных сосудах}} 
\end{center}

\vspace{0.1 in} 

   
% Underlining the speaker's first name and surname is needed.


  \vspace{0.2 in} 


%%%%%%%%%%%%%%%%%%%%%%%%%%%%%%%%%%%%%%%%%%%%%%%%%%%%%%%%%%%%%%%%%%%%%%%%%%%%%%%%%%%%%%%%%%%



%\begin{center} \textbf{Abstract}
%\end{center}


%%%%%%%%%%%%%%%%%%%%%%%%%%%%%%%%%%%%%%%%%%%%%%%%%%%%%%%%%%%%%%%%%%%%%%%%%%%%%%%%%%%%%%%%%%%
%%%%%%%%%%%%%%%%%%%%%%%%%%%%%%%%%%%%%%%%%%%%%%%%%%%%%%%%%%%%%%%%%%%%%%%%%%%%%%%%%%%%%%%%%%%
%               Insert here your abstract.
%               You can use definitions theorems, etc., if you consider it necessary.
%%%%%%%%%%%%%%%%%%%%%%%%%%%%%%%%%%%%%%%%%%%%%%%%%%%%%%%%%%%%%%%%%%%%%%%%%%%%%%%%%%%%%%%%%%%
%%%%%%%%%%%%%%%%%%%%%%%%%%%%%%%%%%%%%%%%%%%%%%%%%%%%%%%%%%%%%%%%%%%%%%%%%%%%%%%%%%%%%%%%%%%

В последнее время интерес к математическому моделированию движения крови в искусственных сердечных клапанах человека существенно возрос, в связи с развитием новых методов лечения различных патологий сердечно сосудистой системы. В данной работе мы предлагаем новую математическую модель для описания динамики течения крови в крупных кровеносных сосудах и искусственном сердечном клапане, а также численный метод решения данной задачи. Исследование проводится совместно с НИИ КССЗ (Кемеровский кардиоцентр), в целях улучшения конструкции создаваемых искусственных клапанов.

Рассмотрим нестационарную задачу о течении крови внутри сосуда. Кровь состоит из плазмы и взвешенных в ней форменных элементов. Стенки сосуда являются гибкими и изменяют свою форму под воздействием течения крови. Будем моделировать кровь как вязкую несжимаемую двухкомпонентную жидкость, а стенки сосуда – как непроницаемую поверхность цилиндрической формы, обладающую заданной жесткостью. Задача о течении крови описывается нестационарной системой дифференциальных уравнений Навье-Стокса \cite{zaharov_miloshevich} с переменными вязкостью и плотностью. Т.к. физически кровь является неоднородной, то концентрацию примеси будем описывать уравнением конвекции \cite{zaharov_miloshevich}. Для моделирования динамики гибких стенок сосуда и створок искусственного сердечного клапана определяются силы, возвращающие их в равновесное положение \cite{boyce}.

Для решения полученной задачи воспользуемся методом погруженной границы \cite{boyce}. Влияние стенок сосуда и клапанов на течение будем учитывать с помощью добавления массовых сил в уравнение движения жидкости \cite{boyce}. Т.о. алгоритм решения будет следующим - на прямоугольной сетке с помощью схем расщепления по физическим факторам вычисляется значение скорости жидкости; затем решаем уравнение конвекции, т.е. определяем концентрацию примеси в области решения и пересчитываем значение плотности и вязкости. Далее вводим новую лагранжевую сетку, на которой определяем деформацию сосуда или створок клапана под воздействием движения жидкости, и вычисляем значение сил, противодействующих деформации. После этого находим новое распределение массовых сил в уравнении движения жидкости.

Полученная модель и численный метод решения были применены для задач развития аневризмы сосуда и течения крови в аортальном клапане. В рамках первой задачи были проведены расчеты, демонстрирующие возможность возникновения устойчивой аневризмы, а также ее влияние на распространение примеси. Для второй задачи получены результаты движения клапанов при различных перепадах давления.


%%%%%%%%%%%%%%%%%%%%%%%%%%%%%%%%%%%%%%%%%%%%%%%%%%%%%%%%%%%%%%%%%%%%%%%%%%%%%%%%%%%%%%%%%%%
%%%%%%%%%%%%%%%%%%%%%%%%%%%%%%%%%%%%%%%%%%%%%%%%%%%%%%%%%%%%%%%%%%%%%%%%%%%%%%%%%%%%%%%%%%%
%               Insert here the Bibliography:
% the references in the abstract must be complete i.e.: surname(s) and the initial(s) of first name(s) 
% of the author(s), title, standard abbreviation of journal title, volume, page numbers, year; 
% for a book, year and Publisher as it is below.
%               
%%%%%%%%%%%%%%%%%%%%%%%%%%%%%%%%%%%%%%%%%%%%%%%%%%%%%%%%%%%%%%%%%%%%%%%%%%%%%%%%%%%%%%%%%%%
%%%%%%%%%%%%%%%%%%%%%%%%%%%%%%%%%%%%%%%%%%%%%%%%%%%%%%%%%%%%%%%%%%%%%%%%%%%%%%%%%%%%%%%%%%%
\small  
\begin{thebibliography}{99}

\bibitem{zaharov_miloshevich}
Milosevic H., Gaydarov N. A., Zakharov Y. N., {\textit{Model of incompressible viscous fluid flow driven by  pressure difference in a given channel}}, International Journal of Heat and Mass Transfer, vol. 62, July 2013.

\bibitem{boyce}
Boyce E.G., {\textit{Immersed boundary model of aortic heart valve dynamics with physiological driving and loading conditions}}, International Journal for Numerical Methods in Biomedical Engineering, (2011).

\end{thebibliography}


%%%%%%%%%%%%%%%%%%%%%%%%%%%%%%%%%%%%%%%%%%%%%%%%%%%%%%%%%%%%%%%%%%%%%%%%%%%%%%%%%%%%%%%%%%%







\end{document}
