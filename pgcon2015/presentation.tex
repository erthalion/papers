\documentclass[14pt, compress]{beamer}

% can be compiled by xelatex -shell-escape presentation.tex

\usetheme[usetitleprogressbar]{m}

\usepackage{booktabs}
\usepackage[scale=2]{ccicons}
\usepackage{listings}
\usepackage{marvosym}
\usepackage[T1]{fontenc}
\usepackage{color}
\usepackage{xcolor}

\definecolor{light-gray}{gray}{0.95}

\usepgfplotslibrary{dateplot}
\renewcommand{\ttdefault}{pcr}

\lstset{
  language=SQL,
  escapeinside={(*@}{@*)},
  basicstyle=\tiny\ttfamily,
  keywordstyle=\bfseries,
  backgroundcolor=\color{light-gray},
  breaklines=true,
  framexleftmargin=10pt,
  framexrightmargin=10pt,
  framextopmargin=10pt,
  framexbottommargin=10pt,
}

\title{Update and Delete operations for jsonb}
\subtitle{providing some needed functions and operators for jsonb}
\date{\today}
\author{Andrew Dunstan, Dmitry Dolgov}
\institute{}

\begin{document}

\maketitle

\section{Introduction}

\begin{frame}[fragile]
  \frametitle{Hstore}

  Hstore is a key-value binary storage.
  Doesn't support tree-like nested structures,
  but there is a nested version of hstore.

\end{frame}

\begin{frame}[fragile]
  \frametitle{JSONB}

  Binary JSON storage. JSONB was introduced in PostgreSQL 9.4 and
  supports fast lookups and simple expression search queries using
  Generalized Inverted Indexes (GIN). It supposed to be document-oriented
  and was designed for the schema-less data.

\end{frame}

\begin{frame}[fragile]
  \frametitle{Lack of some JSONB functionality}

  \begin{itemize}
      \item[\MVRightarrow] Get element at arbitrary path ( \#> )
      \item[\MVRightarrow] Delete element at arbitrary path  ( ? )
      \item[\MVRightarrow] Update element at arbitrary path ( ? )
      \item[\MVRightarrow] Add a new element to arbitrary path ( ? )
  \end{itemize}

\end{frame}

\section{New features}

\begin{frame}[fragile]
  \frametitle{jsonbx}

  Pgxn extension for PostgreSQL 9.4, which contains implementation of
  some missing functionality. It based on nested version of hstore
  and provided this functions for the corresponding patch for 9.5

\end{frame}

\begin{frame}[fragile]
  \frametitle{jsonb\_pretty}

  \begin{lstlisting}[]
select jsonb_pretty('{"a":"test","b":[1,2,3],"c":"test3","d":{"dd":"test4","dd2":{"ddd":"test5"}}}'::jsonb);

        jsonb_pretty        
----------------------------
 {                         +
     "a": "test",          +
     "b": [                +
         1,                +
         2,                +
         3                 +
     ],                    +
     "c": "test3",         +
     "d": {                +
         "dd": "test4",    +
         "dd2": {          +
             "ddd": "test5"+
         }                 +
     }                     +
 }
(1 row)
  \end{lstlisting}

\end{frame}

\begin{frame}[fragile]
  \frametitle{jsonb\_set: replace}

  jsonb\_set(jsonb, text[ ], jsonb)

  \begin{lstlisting}[
      basicstyle=\footnotesize\ttfamily,
      showstringspaces=false,
  ]
select jsonb_set(
    '{"n":null, "a":{"b": 2}}'::jsonb,
    '{n}',
    '[1,2,3]'
);

            jsonb_set                                 
-------------------------------------
{"a": {"b": 2}, "n": [1, 2, 3]}
 (1 row)
 
  \end{lstlisting}

\end{frame}

\begin{frame}[fragile]
  \frametitle{jsonb\_set: create}
  jsonb\_set(jsonb, text[ ], jsonb)

  \begin{lstlisting}[
      basicstyle=\footnotesize\ttfamily,
      showstringspaces=false,
  ]
select jsonb_set(
    '{"a":{"b": 2}}'::jsonb,
    '{c}',
    '[1,2,3]',
    (*@\alert{true}@*)
);

            jsonb_set                                 
-------------------------------------
{"a": {"b": 2}, (*@\alert{"c": [1, 2, 3]}@*)}
 (1 row)
 
  \end{lstlisting}

\end{frame}

\begin{frame}[fragile]
  \frametitle{Path format for jsonb update functions}

  \begin{itemize}
      \item[\MVRightarrow] Path is an array of element for each nesting level
      \item[\MVRightarrow] Each element is a key (if target is an object) or index (if target is an array)
      \item[\MVRightarrow] Negative idx value is supported (countdown from the last key/element)
  \end{itemize}

\end{frame}
\note{probably jsquery can be more useful for this}

\begin{frame}[fragile]
  \frametitle{Path format: Examples}

  \begin{lstlisting}[
      basicstyle=\footnotesize\ttfamily,
      showstringspaces=false,
  ]
select jsonb_set(
    '{"a":{"b": (*@\alert{2}@*)}'::jsonb,
    '{a, b}'
    '[1,2,3]'
);

            jsonb_set                                 
-------------------------------------
{"a": {"b": (*@ \alert{[1, 2, 3]} @*)}}
 (1 row)
 
  \end{lstlisting}

\end{frame}

\begin{frame}[fragile]
  \frametitle{Path format: Examples}

  \begin{lstlisting}[
      basicstyle=\footnotesize\ttfamily,
      showstringspaces=false,
  ]
select jsonb_set(
    '{"a":{"b": [1, 2, (*@\alert{3}@*)]}}'::jsonb,
    '{a, b, -1}'
    '4'
);

            jsonb_set                                 
-------------------------------------
{"a": {"b": [1, 2,(*@ \alert{4}@*)]}}
 (1 row)
 
  \end{lstlisting}

\end{frame}

\begin{frame}[fragile]
  \frametitle{jsonb\_delete}

  jsonb\_delete(jsonb, text)

  \begin{lstlisting}[
      basicstyle=\footnotesize\ttfamily,
      showstringspaces=false,
  ]
select jsonb_delete(
    '{"a":1 , "b":2, "c":3}'::jsonb,
    'a'
);

   jsonb_delete   
------------------
 {"b": 2, "c": 3}
(1 row)

  \end{lstlisting}
\end{frame}
\note{jsonb\_delete(jsonb, text) allow to delete only top level key from object}

\begin{frame}[fragile]
  \frametitle{jsonb\_delete}

  jsonb\_delete(jsonb, int)

  \begin{lstlisting}[
      basicstyle=\footnotesize\ttfamily,
      showstringspaces=false,
  ]
select jsonb_delete(
    '["a", "b", "c"]'::jsonb,
    2
);

  jsonb_delete   
----------------
   ["a", "b"]
(1 row)

  \end{lstlisting}
\end{frame}

\begin{frame}[fragile]
  \frametitle{jsonb\_delete}

  jsonb\_delete(jsonb, text[])

  \begin{lstlisting}[
      basicstyle=\footnotesize\ttfamily,
      showstringspaces=false,
  ]
select jsonb_delete(
    '{"a":1 , "b":{"f": (*@\alert{[2, 3, 4]}@*) "c":5}'::jsonb,
    {b, f, -1}
);

            jsonb_delete   
--------------------------------------
 {"a":1, "b": {"f": (*@\alert{[2, 3]}@*)}, "c": 5}
(1 row)

  \end{lstlisting}
\end{frame}

\begin{frame}[fragile]
  \frametitle{jsonb\_delete}

  \begin{lstlisting}[
      basicstyle=\footnotesize\ttfamily,
      showstringspaces=false,
  ]
select '{"a":1 , "b":2, "c":3}'::jsonb - 'a'::text;

     ?column?     
------------------
 {"b": 2, "c": 3}
(1 row)

  \end{lstlisting}
\end{frame}

\begin{frame}[fragile]
\frametitle{jsonb\_concat}
  jsonb\_concat(jsonb, jsonb)

  (aka "shallow concatenation")

  \begin{lstlisting}[
      basicstyle=\footnotesize\ttfamily,
      showstringspaces=false,
  ]
select jsonb_concat(
    '{"a": 1, "b": [2, 3]}'::jsonb,
    '{"a": 4, "c": [5, 6]}'::jsonb
);
            jsonb_concat            
------------------------------------
 {"a": 4, "b": [2, 3], "c": [5, 6]}
(1 row)

  \end{lstlisting}
\end{frame}

\begin{frame}[fragile]
\frametitle{jsonb\_concat}
  \begin{lstlisting}[
      basicstyle=\footnotesize\ttfamily,
      showstringspaces=false,
  ]
select 
    '{"a": 1, "b": [2, 3]}'::jsonb ||
    '{"a": 4, "c": [5, 6]}'::jsonb;

              ?column?              
------------------------------------
 {"a": 4, "b": [2, 3], "c": [5, 6]}
(1 row)

  \end{lstlisting}
\end{frame}
\note{
    This operator is questionable, and I'm not sure how it will be in 9.5,
    but I think it will be present in this form in jsonbx extension
}

\section{Plans for the future}

\begin{frame}[fragile]
  \frametitle{General thoughts}

  There are still missing functionality and improvements, that can be useful for JSONB.
  Some of them will be implented as parts of jsonbx extension (for 9.5), and will be proposed for 9.6
\end{frame}

\begin{frame}[fragile]
  \frametitle{jsonb\_delete}
  jsonb\_delete(jsonb, jsonb)

  \begin{lstlisting}[
      basicstyle=\footnotesize\ttfamily,
      showstringspaces=false,
  ]
select jsonb_delete_jsonb(
    '{"a": 1, "b": {"c": 2}, (*@\alert{"f": [4, 5]}@*)}'::jsonb,
    '{"a": 4, (*@\alert{"f": [4, 5]}@*), "c": 2}'::jsonb
);

      jsonb_delete
----------------------------
  {"a": 1, "b": {"c": 2}}
(1 row)

  \end{lstlisting}

\end{frame}

\begin{frame}[fragile]
  \frametitle{jsonb\_intersection}
  jsonb\_intersection(jsonb, jsonb)

  \begin{lstlisting}[
      basicstyle=\footnotesize\ttfamily,
      showstringspaces=false,
  ]
select jsonb_intersection(
    '{(*@\alert{"a":2}@*), "d": {"f": 3}, (*@\alert{"g":[4, 5]}@*)}'::jsonb,
    '{(*@\alert{"a":2}@*), "f": 3, (*@\alert{"g":[4, 5]}@*)}'::jsonb
);

   jsonb_intersection
--------------------------
  {"a": 2, "g": [4, 5]}
(1 row)

  \end{lstlisting}

\end{frame}

\begin{frame}[fragile]
  \frametitle{jsonb\_deep\_merge}
  jsonb\_deep\_merge(jsonb, jsonb)

  \begin{lstlisting}[
      basicstyle=\footnotesize\ttfamily,
      showstringspaces=false,
  ]
select jsonb_deep_merge(
    '{"a": (*@\alert{\{"b":1\}}@*)}'::jsonb,
    '{"a": (*@\alert{\{"c":2\}}@*)}'::jsonb
);

     jsonb_deep_merge            
-------------------------------
   {"a": (*@\alert{\{"b":1, "c":2\}}@*)}
(1 row)

  \end{lstlisting}

\end{frame}
\note{
    Form of this function is questionable,
    but I think it will be present in this form in jsonbx extension
}

\begin{frame}[fragile]
  \frametitle{Shortlist}
  
  \begin{itemize}
      \item[\MVRightarrow] Operations with keys and values?
      \item[\MVRightarrow] Integration with Jsquery?
      \item[\MVRightarrow] More elegant syntax?
  \end{itemize}

\end{frame}

\section{Conclusion}

\begin{frame}{Summary}

  Jsonbx extension provide minimum required functionality for the purpose of updating JSONB in PostgreSQL 9.4\\
  Corresponding patch provide the same amount of new functions for PostgreSQL 9.5\\
  Work is in progress, and development of jsonb will be continued (for 9.4 and 9.5 separately)

\end{frame}

\plain{Questions?}

\end{document}
