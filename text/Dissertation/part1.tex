\chapter{Описание существущих математических моделей} \label{chapt1}

\section{Физика и биология процессов в сосудах и клапанах} \label{sect1_1}

\section{Математическая модель} \label{sect1_2}

\begin{figure}[htbp]
\centering
\includegraphics[width=0.9\textwidth]{aorta_valve_scheme.png}
\caption{\label{fig:aorta_valve_scheme}Изображение границ расчетной
области}
\end{figure}

Так как источником движения крови в сосудах является давление,
создаваемое сокращением сердца, то задачу о ее движении опишем следующей
нестационарной системой дифференциальных уравнений Навье-Стокса
\cite{gummel2013motion}:

\begin{gather}
    \label{eq:navier_stokes:motion}
    \frac{\partial \vec{u}}{\partial t} + (\vec{u} \cdot \nabla) \vec{u} = - \frac{1}{\rho} \nabla p + \nabla \sigma + \vec{f}\\
    \label{eq:navier_stokes:continuity}
    \frac{\partial \rho}{\partial t} + \nabla \cdot (\rho \vec{u}) = 0 
\end{gather}

с начальными и граничными условиями:
\begin{gather}
    \label{eq:navier_stokes:velocity_conditions}
    \vec{u}(\bar{x}, 0) = \vec{u}_0 \qquad \vec{u}|_{\Gamma_1, \Gamma_4} = \vec{u}_b \qquad u_{\Gamma_2, \Gamma3} = 0\\
    \label{eq:navier_stokes:pressure_conditions}
    p_{\Gamma_2} = p_{in} \qquad p_{\Gamma_3} = p_{out}
\end{gather}

где \(\bar{x}=(x,y,z) \in \Omega\), \(\vec{u}=(u,v,w)\) - вектор
скорость, \(u, v, w\) - \(x\)-, \(y\)-, \(z\)-компонента вектора
скорости, \(\vec{u}_b\) - скорость движения лепестков клапана под
воздействием деформации, \(\rho=\rho(\bar{x}, t)\) - плотность,
\(p=p(\bar{x}, t)\) - давление,
\(\sigma = \mu (\nabla \vec{u} + (\nabla \vec{u})^T)\) - вязкий тензор
напряжений, \(\mu = \mu(\bar{x}, t)\) - вязкость жидкости,
\(\vec{f} = \vec{f}(\bar{x}, t)\) - вектор массовых сил. Область
\(\Omega\) представляет собой сосуд с границами
\(\Gamma = \Gamma_1 \cup \Gamma_2 \cup \Gamma_3 \cup \Gamma_4\), где
\(\Gamma_1\) - стенки кровеносного сосуда, \(\Gamma_2\) and \(\Gamma_3\)
- области втекания/вытекания, \(\Gamma_4\) - лепестки клапана (см рис.
\ref{fig:aorta_valve_scheme}).

Отсутствие задания одной компоненты вектора скорости на участках
втекания-вытекания является одной из проблем при численном решении задач
подобного типа. Она решается с помощью использования исходных уравнений
(\ref{eq:navier_stokes:motion}) - (\ref{eq:navier_stokes:continuity}) на
границах \(\Gamma_2\) и \(\Gamma_3\) для вычисления недостающих
компонент вектора скорости (подробнее см. \cite{gummel2013motion}, \cite{geidarov2011solution}).

Как показано в \cite{gummel2013motion}, для того, чтобы моделировать движение
неоднородной жидкости (плазма и примеси), можно добавить к системе
уравнений (\ref{eq:navier_stokes:motion}),
(\ref{eq:navier_stokes:continuity}) уравнение переноса концентрации:

\begin{gather}
    \label{eq:convection}
    \frac{\partial c}{\partial t} + \vec{u} \cdot \nabla c = 0
\end{gather}

с начальными условиями:

\begin{gather}
    \label{eq:convection:conditions}
    c(\bar{x}, 0) = c_0(\bar{x}), \bar{x} \in \Omega
\end{gather}

с краевыми условиями для области втекания:

\begin{gather}
    \label{eq:convection:conditions}
    c(\bar{x}, t)|_{\Gamma_2} = c_s(\bar{x}, t)
\end{gather}

и связать переменную плотность и вязкость с концентрацией примеси
следующими линейными соотношениями:

\begin{gather}
    \label{eq:viscosity}
    \mu = c (\mu_2 - \mu_1) + \mu_1\\
    \label{eq:density}
    \rho = c (\rho_2 - \rho_1) + \rho_1
\end{gather}

Т.о. мы получим математическую модель течения крови, которая отражает ее
сложную структуру, а также позволяет легко расширить это описание для
описания большего количества компонент и более сложных условий
зависимости плотности и вязкости от концентрации.

Для того, чтобы иметь возможность моделировать движение тонких гибких клапанов,
необходимо расширить полученную модель, используя метод погруженной границы
(IBM) \cite{peskin2002immersed}, т.е. добавив в нее силы, возникающие при деформации лепестков клапана и
стремящиеся вернуть их в равновесное состояние.

Метод погруженной границы используется для описания систем
<<жидкость-препятствие>>, где эластичное препятствие погружено в вязкую
несжимаемую жидкость.  Впервые был предложен в работе \cite{peskin1972flow} для
моделирования механики сердечных клапанов и потока крови в них.  Суть метода
заключается в том, что при обтекании какого-либо тела жидкостью, она испытывает
влияние сил по направлению нормали к поверхности тела
\cite{goldstein1993modeling}.  Обтекаемое тело также испытывает влияние этих
сил с противоположным знаком.  Поэтому моделирование обтекания препятствия
потоком жидкости возможно с помощью формирования соответствующего поля внешних
массовых сил в уравнении Навье-Стокса.  Это позволяет производить вычисления на
простых прямоугольных сетках, которые могут не соответсвовать геометрии
расчетной области, что является одной из основных отличительных особенностей
метода.  При этом под термином <<метод погруженной границы>> обычно понимают
как математическую формулировку, так и схему для численного решения полученной
задачи. <<Погруженной границей>> в данном контексте обозначают любое гибкое
препятствие, погруженное в жидкость. В данной работе под этим будем
подразумевать стенки кровеносного сосуда, а также лепестки клапана,
расположенные внутри.  Существует также множество модификаций этого метода
(например, метод погруженного интерфейса, метод декартовых сеток, метод
фиктивных ячеек, метод усеченных ячеек, см. \cite{mittal2005immersed}), но
указаные методы не рассматриваются в данной работе.

Математическую формулировку этого метода и численную схему можно разделить на
три раздела:

\begin{itemize}
    \item Моделирование течения вязкой несжимаемой жидкости
    \item Моделирование деформации погруженной границы
    \item Моделирование взаимодействия между жидкостью и погруженной границей
\end{itemize}

Течение вязкой несжимаемой жидкости описывается системой нелинейных
дифференциальных уравнений Навье-Стокса в прямоугольной области
$\tilde{\Omega}$, которая включает в себя расчетную область $\Omega$.
Погруженная граница представлена в виде набора упругих безмассовых волокон,
имеющих <<нейтральную плавучесть>> \cite{griffith2012immersed}, расположение
которых описано в лагранжевых координатах, а эластичность описана в терминах
функционала энергии деформации. Взаимодействие осуществляется исходя из того,
что погруженная граница движется под давлением жидкости с той же скоростью, что
и сама жидкости, а внешние массовые силы в уравнении Навье-Стокса определяются
поверхностным напряжением, возникшим в результате деформации погруженной
границы. Рассмотрим подробнее математическую постановку метода погруженной границы
и вывод его основных положений (см. \cite{peskin2002immersed}).

Для того, чтобы построить модель деформации, опишем погруженную границу как
гибкий несжимаемый материал, который расположен в области $\tilde{\Omega}$, где
$(q, r, s) = \bar{q}$ - криволинейные координаты связанные с материалом так,
что зафиксированное значение $(q, r, s)$ обозначает одну точку этого материала.
Пусть $X(q, r, s, t) = X(\bar{q}, t)$ - позиция в декартовых координатах точки,
обозначенной $(q, r, s)$ в момент времени $t$. Тогда обобщая, $X(\bar{q}, t)$
описывает пространственную конфигурацию всей погруженной границы в момент
времени $t$, которая определяет соответствующую энергию деформации
$E[X(\bar{q}, t)]$ в момент $t$. Рассмотрим возмущение $\wp X(\bar{q}, t)$
конфигурации $X(\bar{q}, t)$, где $\wp$ - оператор возмущения.
%NOTE: перевести точнее, "up to terms of first order" - с точностью первого
%порядка?
Результирующая деформация упругой энергии, возникшая в результате этого
возмущения, есть линейная функция от возмущения конфигурации, поэтому она может
быть записана в форме: \begin{equation} \label{eq:elastic_energy_functional}
    \wp E[X(\bar{q}, t)] = \int_{\tilde{\Omega}} (-F(\bar{q}, t)) \cdot \wp
    X(\bar{q}, t)\; d\bar{q} \end{equation} где $-F(\bar{q}, t)$ - производная
Фреше от $E$, вычисленная для конфигурации $X(\bar{q}, t)$. Физически $F$ можно
интепретировать, как плотность силы, которая создается путем деформирования
погруженной границы.

Введем два вида погруженных границ:
\begin{itemize}
    \item Неподвижная или малоподвижная граница
    \item Гибкая граница 
\end{itemize}
С помощью неподвижной границы будем моделировать фиксированные участки
$\Gamma$, например, фиброзное кольцо, к которому крепятся лепестки клапана, или
стенки кровеносного сосуда, в том случае, если нас не интересуют их деформации.
Гибкие границы предназначены для моделирования подвижных участков $\Gamma$,
которые испытывают наибольшие деформации, например, лепестки клапана. Описанные
типы границ обладают разными характеристиками, поэтому для их описания мы будем
использовать разные модели.

Для случая неподвижной границы мы можем использовать простую формулу,
выражающую силу $F$ через смещение границы в момент $t$ относительно исходного
положения в момент $t_0$:
\begin{equation}
    \label{eq:simple_force}
    F = k \| X(\bar{q}, t_0) - X(\bar{q}, t) \|
\end{equation}
где $k$ - коэффициент жесткости.

Формула (\ref{eq:simple_force}) не подходит для гибких границ, т.к. позволяет
учитывать только небольшие деформации.  Соответствующее уравнение для них может
быть получено исходя из следующих соображений.  Как было сказано выше,
погруженная граница представлена набором волокон.  Для того, чтобы определить
$E$, удобно выбирать лагранжевые координаты $(q, r, s)$ так, что каждое
значение $(q, r)$ параметрически задает какое-либо одно конкретное волокно $s
\to X(q^0, r^0, s)$.  В этом случае функционал упругой энергии $E$ можно
представить в виде $E = E_s + E_b$, где $E_s$ - энергия растяжения волокна,
$E_b$ - энергия скручивания волокна, а силу $F$, созданную благодаря
деформации, в виде $F = F_s + F_b$.

Как показано в \cite{peskin2002immersed}, \cite{griffith2009simulating}
функционал энергии растяжения можно записать в виде:
\begin{equation}
    \label{eq:stretching_energy_functional}
    E_s = \int_{\tilde{\Omega}} \varepsilon_s \left(\left| \frac{\partial X}{\partial s} \right| \right) d\bar{q}
\end{equation}
а силу $F_s$:
\begin{equation}
    \label{eq:stretching_force_density}
    F_s = \frac{\partial}{\partial s} \varepsilon_s^{,} \left( \left| \frac{\partial X}{\partial s} \right| \right) \tau
\end{equation}
где $\varepsilon_s$ - локальная энергия растяжения, $\tau$ - единичный
тангенциальный вектор для выбранного волокна $s$.  Т.к. напряжение растяжения
$T = \varepsilon_s^{,} \left( \left| \frac{\partial X}{\partial s} \right|
\right)$, то уравнение (\ref{eq:stretching_force_density}) можно переписать в
виде, который соответствует обобщенному закону Гука:
\begin{equation}
    \label{eq:stretching_force_density_simplified}
    F_s = \frac{\partial}{\partial s} T \tau
\end{equation}

Т.к. площать попреречного сечения сосудов достаточно мала по отношению к их
длинне, для моделирования энергии напряжения и сил сопротивления скручиванию мы
можем воспользоваться уравнением Эйлера-Бернулли \cite{gere1997mechanics}:
% also see here https://en.wikipedia.org/wiki/Euler%E2%80%93Bernoulli_beam_theory
\begin{equation}
    E_b = \frac{1}{2} \int_{\tilde{\Omega}} k \cdot \left| \frac{\partial^2 X^0}{\partial s^2} - \frac{\partial^2 X}{\partial s^2} \right|^2 \; d\bar{q}
\end{equation}

\begin{equation}
    F_b = \frac{\partial^2}{\partial s} \left( k \cdot \left(\frac{\partial^2 X^0}{\partial s^2} - \frac{\partial^2 X}{\partial s^2} \right) \right)
\end{equation}
где $k = E \cdot I$, $E$ - модуль упругости, $I$ - момент инерции поперечного
сечения, $\frac{\partial^2 X^0}{\partial s^2}$, $\frac{\partial X}{\partial
    s^2}$ - отклонение погруженной границы от равновесного положения в
начальный и текущий момент времени.

Таким образом, плотность силы $F$, создаваемая при деформации погруженной
границы может быть выражена в виде:
\begin{equation}
    F = \frac{\partial}{\partial s} T \tau + \frac{\partial^2}{\partial s} \left( k \cdot \left(\frac{\partial^2 X^0}{\partial s^2} - \frac{\partial^2 X}{\partial s^2} \right) \right)
\end{equation}

Как показано в \cite{peskin2002immersed}, для того, чтобы описать
взаимодействие потока жидкости и погруженной границы, необходимо ввести в
рассмотрение прямоугольную область $(x, y, z) = \bar{x} \in \tilde{\Omega}$,
так что $\Omega \in \tilde{\Omega}$, а также область $(q, r, s) = \bar{q} \in
\Gamma$, которая соответствует точкам погруженной границы в лагранжевых
координатах.  В этих областях запишем следующую систему уравнений:
\begin{gather}
    \label{eq:navier_stokes:ibm:motion}
    \frac{\partial \vec{u}}{\partial t} + (\vec{u} \cdot \nabla) \vec{u} = - \frac{1}{\rho} \nabla p + \nabla \sigma + \vec{f}\\
    \label{eq:navier_stokes:ibm:continuity}
    \frac{\partial \rho}{\partial t} + \nabla \cdot (\rho \vec{u}) = 0\\
    \label{eq:interaction:velocity}
    \frac{\partial X}{\partial t}(\bar{q}, t) = \int_{\Omega} \vec{u}(\bar{x}, t) \cdot \delta (x - X(\bar{q}, t))\; d\bar{x}\\
    \label{eq:interaction:force}
    \vec{f}(\bar{x}, t) = \int_{\Gamma} \vec{F}(\bar{q}, t) \cdot \delta (x - X(\bar{q}, t))\; d\bar{q}
\end{gather}

Уравнения (\ref{eq:navier_stokes:ibm:motion}),
(\ref{eq:navier_stokes:ibm:continuity}) - это система уравнений Навье-Стокса,
которая используется для моделирования течения жидкости, и полностью записана в
эйлеровых координатах $(x, y, z) \in \tilde{\Omega}$.  Уравнения
\ref{eq:interaction:velocity},\ref{eq:interaction:force} являются уравнениями
взаимодействия жидкости с погруженной границей и позволяет переходить от
эйлеровых к лагранжевым координатам. В уравнениях
(\ref{eq:interaction:velocity}), (\ref{eq:interaction:force}) $\delta$ - дельта
функция Дирака, $F$ - плотность силы деформации, описанная в предыдущем разделе
\textbf{Моделирование деформации погруженной границы}.

Подробное доказательство системы уравнений (\ref{eq:navier_stokes:motion}) -
(\ref{eq:interaction:force}) опубликовано в \cite{peskin2002immersed}.  В
данной работы мы приведем его краткое описание. Рассмотрим более обший случай,
когда гибкий несжимаемый материал заполняет всю область $\tilde{\Omega}$, т.е.
$\tilde{\Omega} = \Gamma$.  Исходя из принципа наименьшего действия, на
временном промежутке $(0, T)$ система должна развиваться так, чтобы достигался
минимум:
\begin{equation}
    S = \int_0^T L(t) dt
\end{equation}
где $L(t)$ - разница между кинетической и потенциальной энергией, в нашем случае:
% далее везде опущена плотность, надо уточнить
% видимо, дальнейшая цель - взять это уравнение, как данность, и перевести
% все его параметры из лагранжевых в эйлеровы
\begin{equation}
    \label{eq:least_action}
    L(t) = \frac{1}{2} \int \left| \frac{\partial X}{\partial t}(\bar{q}, t) \right|^2 \; d\bar{q} - E(X(\bar{q}, t))
\end{equation}
Т.о. для любого возмущения $\wp X$ требуется:
\begin{equation}
    0 = -\wp S = \int_0^T \int \left( \frac{\partial^2 X}{\partial t^2} - F \right) \cdot \wp X \; d\bar{q} \; dt
\end{equation}
Вводя в рассмотрение скорость жидкости $u$ и скорость деформации $U$:
\begin{gather}
    u(X(\bar{q}, t), t) = \frac{\partial X}{\partial t}(\bar{q}, t) \\
    U(X(\bar{q}, t), t) = \wp X(\bar{q}, t)
\end{gather}
можно показать, что $\wp X$ удовлетворяет условию несжимаемости в лагранжевых
координатах, и можно записать следующие уравнения:
\begin{gather}
    \wp X(\bar{q}, t) = \int U(\bar{x}, t) \; \delta(\bar{x} - X(\bar{q}, t)) d\bar{x} \\
    \frac{\partial^2 X}{\partial t^2} \cdot \wp X(\bar{q}, t) = \int  \left(\frac{\partial u}{\partial t} + u \cdot \triangle u \right) U(\bar{x}, t) \; \delta(\bar{x} - X(\bar{q}, t)) d\bar{x}
\end{gather}
Подставляя эти формулы в (\ref{eq:least_action}) получим:
\begin{equation}
    0 = \int_0^T \int \int \left( \frac{\partial u}{\partial t} + u \cdot \triangle u - F(\bar{q}, t) \right) \cdot U(\bar{x}, t) \; \delta(\bar{x} - X(\bar{q}, t)) \; d\bar{x} \; d\bar{q} \; dt
\end{equation}
Произведя замену
\begin{equation}
    f(x, t) = \int F(\bar{q}, t) \; \delta(\bar{x} - X(\bar{q}, t)) \; d\bar{q}
\end{equation}
после ряда преобразований можно завершить переход от лагранжевых переменных к
эйлеровым.

Помимо этого, в \cite{peskin2002immersed} показано, что несмотря на отсутствие
предположений о неоднородности плотности жидкости, уравнение:
\begin{equation}
    \frac{\partial \rho}{\partial t} + u \cdot \triangle \rho = 0
\end{equation}
есть следствие приведенной системы уравнений. Метод погруженной границы
является следствием описанного случая, в котором эластичный несжимаемый
материал заполняет не всю область $\tilde{\Omega}$, а только часть, и
представляет собой поверхность, погруженную в жидкость.


Для описания сил, возникающих при деформации клапана, воспользуемся следующей
формулой \cite{peskin2002immersed}, \cite{griffith2012immersed}:
\begin{gather}
    \label{eq:resulting_force}
    F = \frac{\partial}{\partial s} (T \tau) + \frac{\partial^2}{\partial s^2} (E \cdot I \frac{\partial^2}{\partial s^2} X)
\end{gather}

где \(k = E \cdot I\), \(E\) - модуль упругости, \(I\) - момент инерции
поперечного сечения, \(\frac{\partial^2 X^0}{\partial s^2}\), \(\frac{\partial
        X}{\partial s^2}\) - отклонение погруженной границы от равновесного
положения в начальный и текущий момент времени.

Как показано в \cite{peskin2002immersed}, для того, чтобы описать
взаимодействие потока жидкости и клапана, необходимо ввести в рассмотрение
прямоугольную область \((x, y, z) = \bar{x} \in \tilde{\Omega}\), так что
\(\Omega \in \tilde{\Omega}\), а также область \((q, r, s) = \bar{q} \in
    \Gamma\), которая соответствует точкам клапана в лагранжевых координатах.
После этого, опишем взаимодействи с помощью следующих уравнений:
\begin{gather}
    \label{eq:interaction:velocity}
    \frac{\partial X}{\partial t}(\bar{q}, t) = \int_{\Omega} \vec{u}(\bar{x}, t) \cdot \delta (x - X(\bar{q}, t))\; dx\; dy\; dz\\
    \label{eq:interaction:force}
    \vec{f}(\bar{x}, t) = \int_{\Gamma} \vec{F}(\bar{q}, t) \cdot \delta (x - X(\bar{q}, t))\; dq\; dr\; ds
\end{gather}

где \(\delta\) - дельта функция Дирака, \(F\) - плотность силы деформации.
Уравнения \ref{eq:interaction:velocity}, \ref{eq:interaction:force} и позволяют
переходить от эйлеровых к лагранжевым координатам.

Таким образом, мы построили модель, описывающую движение вязкой неоднородной
несжимаемой жидкости внутри сосуда с клапаном. В этой модели состояние жидкости
и форма поверхностей \(\Gamma_1 \cup \Gamma_4\) определяются независимо друг от
друга, а влияние створок клапана на течение отражено с помощью соотношения
(\ref{eq:interaction:force}) между вектором массовых сил \(\vec{f}(\bar{x},
    t)\) уравнения (\ref{eq:navier_stokes:motion}) и силой сопротивления
деформации \(F=F(\bar{q}, t)\) из уравнения (\ref{eq:resulting_force}).

\section{Метод численного решения поставленной задачи} \label{sect1_3}

\subsection{Метод расщепления по физическим факторам} \label{subsect1_3_1}

\subsection{Метод решения уравнения конвекции} \label{subsect1_3_2}

\subsection{Метод погруженной границы} \label{subsect1_3_3}

Мы можем сделать \textbf{жирный текст} и \textit{курсив}.

%\newpage
%============================================================================================================================

\section{Ссылки} \label{sect1_2}
Сошлёмся на библиографию. Одна ссылка: \cite{Sychev}. Две ссылки: \cite{Sokolov,Gaidaenko}. Много ссылок: \cite{Lermontov,Management,Borozda,Marketing,Constitution,FamilyCode,Gost.7.0.53,Razumovski,Lagkueva,Pokrovski,Sirotko,Lukina,Methodology,Encyclopedia,Nasirova,Berestova,Kriger}. И ещё немного ссылок: \cite{Article,Book,Booklet,Conference,Inbook,Incollection,Manual,Mastersthesis,Misc,Phdthesis,Proceedings,Techreport,Unpublished}.

Сошлёмся на приложения: Приложение \ref{AppendixA}, Приложение \ref{AppendixB2}.

Сошлёмся на формулу: формула \eqref{eq:equation1}.

Сошлёмся на изображение: рисунок \ref{img:knuth}.

%\newpage
%============================================================================================================================

\section{Формулы} \label{sect1_3}

Благодаря пакету \textit{icomma}, \LaTeX~одинаково хорошо воспринимает в качестве десятичного разделителя и запятую ($3,1415$), и точку ($3.1415$).

\subsection{Ненумерованные одиночные формулы} \label{subsect1_3_1}

Вот так может выглядеть формула, которую необходимо вставить в строку по тексту: $x \approx \sin x$ при $x \to 0$.

А вот так выглядит ненумерованая отдельностоящая формула c подстрочными и надстрочными индексами:
$$
(x_1+x_2)^2 = x_1^2 + 2 x_1 x_2 + x_2^2
$$

При использовании дробей формулы могут получаться очень высокие:
$$
  \frac{1}{\sqrt(2)+
  \displaystyle\frac{1}{\sqrt{2}+
  \displaystyle\frac{1}{\sqrt{2}+\cdots}}}
$$

В формулах можно использовать греческие буквы:
$$
\alpha\beta\gamma\delta\epsilon\varepsilon\zeta\eta\theta\vartheta\iota\kappa\lambda\\mu\nu\xi\pi\varpi\rho\varrho\sigma\varsigma\tau\upsilon\phi\varphi\chi\psi\omega\Gamma\Delta\Theta\Lambda\Xi\Pi\Sigma\Upsilon\Phi\Psi\Omega
$$

%\newpage
%============================================================================================================================

\subsection{Ненумерованные многострочные формулы} \label{subsect1_3_2}

Вот так можно написать две формулы, не нумеруя их, чтобы знаки равно были строго друг под другом:
\begin{eqnarray}
  f_W & = & \min \left( 1, \max \left( 0, \frac{W_{soil} / W_{max}}{W_{crit}} \right)  \right), \nonumber \\
  f_T & = & \min \left( 1, \max \left( 0, \frac{T_s / T_{melt}}{T_{crit}} \right)  \right), \nonumber
\end{eqnarray}

Можно использовать разные математические алфавиты:
\begin{eqnarray}
\mathcal{ABCDEFGHIJKLMNOPQRSTUVWXYZ} \nonumber \\
\mathfrak{ABCDEFGHIJKLMNOPQRSTUVWXYZ} \nonumber \\
\mathbb{ABCDEFGHIJKLMNOPQRSTUVWXYZ} \nonumber
\end{eqnarray}

Посмотрим на систему уравнений на примере аттрактора Лоренца:

$$
\left\{
  \begin{array}{rl}
    \dot x = & \sigma (y-x) \\
    \dot y = & x (r - z) - y \\
    \dot z = & xy - bz
  \end{array}
\right.
$$

А для вёрстки матриц удобно использовать многоточия:
$$
\left(
  \begin{array}{ccc}
  	a_{11} & \ldots & a_{1n} \\
  	\vdots & \ddots & \vdots \\
  	a_{n1} & \ldots & a_{nn} \\
  \end{array}
\right)
$$


%\newpage
%============================================================================================================================
\subsection{Нумерованные формулы} \label{subsect1_3_3}

А вот так пишется нумерованая формула:
\begin{equation}
  \label{eq:equation1}
  e = \lim_{n \to \infty} \left( 1+\frac{1}{n} \right) ^n
\end{equation}

Нумерованых формул может быть несколько:
\begin{equation}
  \label{eq:equation2}
  \lim_{n \to \infty} \sum_{k=1}^n \frac{1}{k^2} = \frac{\pi^2}{6}
\end{equation}

В последствии на формулы (\ref{eq:equation1}) и (\ref{eq:equation2}) можно ссылаться.

%\newpage
%============================================================================================================================

\clearpage
