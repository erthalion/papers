\section*{\Large Метод погруженной границы}
\fontsize{14pt}{15pt}\selectfont
\subsection*{Введение, основные термины}
Метод погруженной границы используется для описания систем <<жидкость-препятствие>>,
где эластичное препятствие погружено в вязкую несжимаемую жидкость.
Впервые был предложен в работе \cite{peskin1972flow} для моделирования механики
сердечных клапанов и потока крови в них.
Суть метода заключается в том, что при обтекании какого-либо тела жидкостью,
она испытывает влияние сил по направлению нормали к поверхности тела \cite{goldstein1993modeling}.
Обтекаемое тело также испытывает влияние этих сил с противоположным знаком.
Поэтому моделирование обтекания препятствия потоком жидкости возможно
с помощью формирования соответствующего поля внешних массовых сил в уравнении Навье-Стокса.
Это позволяет производить вычисления на простых прямоугольных сетках, которые могут не соответсвовать
геометрии расчетной области, что является одной из основных отличительных особенностей метода.
При этом под термином <<метод погруженной границы>> обычно понимают как математическую формулировку,
так и схему для численного решения полученной задачи. <<Погруженной границей>> в данном контексте
обозначают любое гибкое препятствие, погруженное в жидкость. В данной работе под этим будем подразумевать
стенки кровеносного сосуда, а также лепестки клапана, расположенные внутри.
Существует также множество модификаций этого метода (например, метод погруженного интерфейса,
метод декартовых сеток, метод фиктивных ячеек, метод усеченных ячеек, см. \cite{mittal2005immersed}),
но указаные методы не рассматриваются в данной работе.

Математическую формулировку этого метода и численную схему можно разделить на три раздела:
\begin{itemize}
    \item Моделирование течения вязкой несжимаемой жидкости
    \item Моделирование деформации погруженной границы
    \item Моделирование взаимодействия между жидкостью и погруженной границей
\end{itemize}

Течение вязкой несжимаемой жидкости описывается системой нелинейных дифференциальных уравнений Навье-Стокса
в прямоугольной области $\tilde{\Omega}$, которая включает в себя расчетную область $\Omega$. Погруженная граница представлена в виде
набора упругих безмассовых волокон, имеющих <<нейтральную плавучесть>> \cite{griffith2012immersed},
расположение которых описано в лагранжевых координатах, а эластичность описана в терминах функционала
энергии деформации. Взаимодействие осуществляется исходя из того, что погруженная граница движется под давлением
жидкости с той же скоростью, что и сама жидкости, а внешние массовые силы в уравнении Навье-Стокса определяются
поверхностным напряжением, возникшим в результате деформации погруженной границы. В следующих разделах
эти задачи будут рассмотрены подробнее.

\subsection*{Моделирование течения вязкой несжимаемой жидкости}

Для моделирования течения используется система дифференциальных уравнений Навье-Стокса:
\begin{gather}
    \label{eq:navier_stokes:motion}
    \frac{\partial \vec{u}}{\partial t} + (\vec{u} \cdot \nabla) \vec{u} = - \frac{1}{\rho} \nabla p + \nabla \sigma + \vec{f}\\
    \label{eq:navier_stokes:continuity}
    \frac{\partial \rho}{\partial t} + \nabla \cdot (\rho \vec{u}) = 0 
\end{gather}
с начальными и граничными условиями:
\begin{gather}
    \label{eq:navier_stokes:velocity_conditions}
    \vec{u}(\bar{x}, 0) = \vec{u}_0 \qquad \vec{u}|_{\Gamma_1, \Gamma_4} = \vec{u}_b \qquad u_{\Gamma_2, \Gamma3} = 0\\
    \label{eq:navier_stokes:pressure_conditions}
    p_{\Gamma_2} = p_{in} \qquad p_{\Gamma_3} = p_{out}
\end{gather}
где $\bar{x}=(x,y,z) \in \Omega$, $\vec{u}=(u,v,w)$ - вектор скорости, $u, v, w$ - $x$-, $y$-, $z$-компоненты вектора скорости,
$\vec{u}_b$ - скорость движения погруженной границы (стенок сосуда и лепестков клапана) при деформации,
$\rho=\rho(\bar{x}, t)$ - плотность, $p=p(\bar{x}, t)$ - давление, $\sigma = \mu (\nabla \vec{u} + (\nabla \vec{u})^T)$ - вязкий тензор напряжений,
$\mu = \mu(\bar{x}, t)$ - вязкость жидкости, $\vec{f} = \vec{f}(\bar{x}, t)$ - вектор массовых сил.
Область $\Omega$ представляет собой сосуд с границами $\Gamma = \Gamma_1 \cup \Gamma_2 \cup \Gamma_3 \cup \Gamma_4$,
где $\Gamma_1$ - стенка кровеносного сосуда, $\Gamma_2$ and $\Gamma_3$ - область втекания/вытекания,
$\Gamma_4$ - лепестки клапана (см Рис. \ref{img:aorta_boundaries}).

\begin{figure}[h] 
  \center
  \includegraphics [width=14cm] {aorta_valve_scheme.png}
  \caption{Изображение границ расчетной области} 
  \label{img:aorta_boundaries}
\end{figure}

\subsection*{Моделирование деформации погруженной границы}
Опишем погруженную границу как гибкий несжимаемый материал, который расположен в области $\tilde{\Omega}$,
где $(q, r, s) = \bar{q}$ - криволинейные координаты связанные с материалом так, что зафиксированное значение
$(q, r, s)$ обозначает одну точку этого материала. Пусть $X(q, r, s, t) = X(\bar{q}, t)$ - позиция в декартовых координатах
точки, обозначенной $(q, r, s)$ в момент времени $t$. Тогда обобщая, $X(\bar{q}, t)$ описывает пространственную
конфигурацию всей погруженной границы в момент времени $t$, которая определяет соответствующую энергию деформации
$E[X(\bar{q}, t)]$ в момент $t$. Рассмотрим возмущение $\wp X(\bar{q}, t)$ конфигурации $X(\bar{q}, t)$,
где $\wp$ - оператор возмущения.
%NOTE: перевести точнее, "up to terms of first order" - с точностью первого порядка?
Результирующая деформация упругой энергии, возникшая в результате этого возмущения,
есть линейная функция от возмущения конфигурации, поэтому она может быть записана в форме:
\begin{equation}
\label{eq:elastic_energy_functional}
\wp E[X(\bar{q}, t)] = \int_{\tilde{\Omega}} (-F(\bar{q}, t)) \cdot \wp X(\bar{q}, t)\; d\bar{q}
\end{equation}
где $-F(\bar{q}, t)$ - производная Фреше от $E$, вычисленная для конфигурации $X(\bar{q}, t)$. Физически $F$
можно интепретировать, как плотность силы, которая создается путем деформирования погруженной границы.

Введем два вида погруженных границ:
\begin{itemize}
    \item Неподвижная или малоподвижная граница
    \item Гибкая граница 
\end{itemize}
С помощью неподвижной границы будем моделировать фиксированные участки $\Gamma$, например, фиброзное кольцо,
к которому крепятся лепестки клапана, или стенки кровеносного сосуда, в том случае, если нас не интересуют
их деформации. Гибкие границы предназначены для моделирования подвижных участков $\Gamma$, которые испытывают
наибольшие деформации, например, лепестки клапана. Описанные типы границ обладают разными характеристиками, поэтому
для их описания мы будем использовать разные модели.

Для случая неподвижной границы мы можем использовать простую формулу, выражающую силу $F$ через смещение границы
в момент $t$ относительно исходного положения в момент $t_0$:
\begin{equation}
    \label{eq:simple_force}
    F = k \| X(\bar{q}, t_0) - X(\bar{q}, t) \|
\end{equation}
где $k$ - коэффициент жесткости.

Формула (\ref{eq:simple_force}) не подходит для гибких границ, т.к. позволяет учитывать только небольшие деформации.
Соответствующее уравнение для них может быть получено исходя из следующих соображений.
Как было сказано выше, погруженная граница представлена набором волокон.
Для того, чтобы определить $E$, удобно выбирать лагранжевые координаты $(q, r, s)$ так,
что каждое значение $(q, r)$ параметрически задает какое-либо одно конкретное волокно $s \to X(q^0, r^0, s)$.
В этом случае функционал упругой энергии $E$ можно представить в виде $E = E_s + E_b$,
где $E_s$ - энергия растяжения волокна, $E_b$ - энергия скручивания волокна,
а силу $F$, созданную благодаря деформации, в виде $F = F_s + F_b$.

Как показано в \cite{peskin2002immersed}, \cite{griffith2009simulating} функционал энергии растяжения можно записать в виде:
\begin{equation}
\label{eq:stretching_energy_functional}
E_s = \int_{\tilde{\Omega}} \varepsilon_s \left(\left| \frac{\partial X}{\partial s} \right| \right) d\bar{q}
\end{equation}
а силу $F_s$:
\begin{equation}
\label{eq:stretching_force_density}
F_s = \frac{\partial}{\partial s} \varepsilon_s^{,} \left( \left| \frac{\partial X}{\partial s} \right| \right) \tau
\end{equation}
где $\varepsilon_s$ - локальная энергия растяжения, $\tau$ - единичный тангенциальный вектор для выбранного волокна $s$.
Т.к. напряжение растяжения $T = \varepsilon_s^{,} \left( \left| \frac{\partial X}{\partial s} \right| \right)$, то уравнение (\ref{eq:stretching_force_density})
можно переписать в виде, который соответствует обобщенному закону Гука:
\begin{equation}
\label{eq:stretching_force_density_simplified}
F_s = \frac{\partial}{\partial s} T \tau
\end{equation}

Т.к. площать попреречного сечения сосудов достаточно мала по отношению к их длинне, для моделирования энергии напряжения и 
сил сопротивления скручиванию мы можем воспользоваться уравнением Эйлера-Бернулли \cite{gere1997mechanics}:
% also see here https://en.wikipedia.org/wiki/Euler%E2%80%93Bernoulli_beam_theory
\begin{equation}
    E_b = \frac{1}{2} \int_{\tilde{\Omega}} k \cdot \left| \frac{\partial^2 X^0}{\partial s^2} - \frac{\partial^2 X}{\partial s^2} \right|^2 \; d\bar{q}
\end{equation}

\begin{equation}
    F_b = \frac{\partial^2}{\partial s} \left( k \cdot \left(\frac{\partial^2 X^0}{\partial s^2} - \frac{\partial^2 X}{\partial s^2} \right) \right)
\end{equation}
где $k = E \cdot I$, $E$ - модуль упругости, $I$ - момент инерции поперечного сечения, $\frac{\partial^2 X^0}{\partial s^2}$, $\frac{\partial X}{\partial s^2}$ -
отклонение погруженной границы от равновесного положения в начальный и текущий момент времени.

Таким образом, плотность силы $F$, создаваемая при деформации погруженной границы может быть выражена в виде:
\begin{equation}
    F = \frac{\partial}{\partial s} T \tau + \frac{\partial^2}{\partial s} \left( k \cdot \left(\frac{\partial^2 X^0}{\partial s^2} - \frac{\partial^2 X}{\partial s^2} \right) \right)
\end{equation}

\subsection*{Моделирование взаимодействия}
Как показано в \cite{peskin2002immersed}, для того, чтобы описать взаимодействие потока жидкости и погруженной границы,
необходимо ввести в рассмотрение прямоугольную область $(x, y, z) = \bar{x} \in \tilde{\Omega}$, так что $\Omega \in \tilde{\Omega}$, а также
область $(q, r, s) = \bar{q} \in \Gamma$, которая соответствует точкам погруженной границы в лагранжевых координатах.
В этих областях запишем следующую систему уравнений:
\begin{gather}
    \label{eq:navier_stokes:ibm:motion}
    \frac{\partial \vec{u}}{\partial t} + (\vec{u} \cdot \nabla) \vec{u} = - \frac{1}{\rho} \nabla p + \nabla \sigma + \vec{f}\\
    \label{eq:navier_stokes:ibm:continuity}
    \frac{\partial \rho}{\partial t} + \nabla \cdot (\rho \vec{u}) = 0\\
    \label{eq:interaction:velocity}
    \frac{\partial X}{\partial t}(\bar{q}, t) = \int_{\Omega} \vec{u}(\bar{x}, t) \cdot \delta (x - X(\bar{q}, t))\; d\bar{x}\\
    \label{eq:interaction:force}
    \vec{f}(\bar{x}, t) = \int_{\Gamma} \vec{F}(\bar{q}, t) \cdot \delta (x - X(\bar{q}, t))\; d\bar{q}
\end{gather}

Уравнения (\ref{eq:navier_stokes:ibm:motion}), (\ref{eq:navier_stokes:ibm:continuity}) - это система уравнений Навье-Стокса,
которая используется для моделирования течения жидкости, и полностью записана в эйлеровых координатах $(x, y, z) \in \tilde{\Omega}$.
Уравнения \ref{eq:interaction:velocity},\ref{eq:interaction:force} являются уравнениями взаимодействия жидкости с погруженной границей
и позволяет переходить от эйлеровых к лагранжевым координатам. В уравнениях (\ref{eq:interaction:velocity}), (\ref{eq:interaction:force})
$\delta$ - дельта функция Дирака, $F$ - плотность силы деформации, описанная в предыдущем разделе \textbf{Моделирование деформации погруженной границы}.

Подробное доказательство системы уравнений (\ref{eq:navier_stokes:motion}) - (\ref{eq:interaction:force}) опубликовано в \cite{peskin2002immersed}.
В данной работы мы приведем его краткое описание. Рассмотрим более обший случай,
когда гибкий несжимаемый материал заполняет всю область $\tilde{\Omega}$, т.е. $\tilde{\Omega} = \Gamma$.
Исходя из принципа наименьшего действия, на временном промежутке $(0, T)$ система должна развиваться так, чтобы достигался минимум:
\begin{equation}
    S = \int_0^T L(t) dt
\end{equation}
где $L(t)$ - разница между кинетической и потенциальной энергией, в нашем случае:
% далее везде опущена плотность, надо уточнить
% видимо, дальнейшая цель - взять это уравнение, как данность, и перевести
% все его параметры из лагранжевых в эйлеровы
\begin{equation}
    \label{eq:least_action}
    L(t) = \frac{1}{2} \int \left| \frac{\partial X}{\partial t}(\bar{q}, t) \right|^2 \; d\bar{q} - E(X(\bar{q}, t))
\end{equation}
Т.о. для любого возмущения $\wp X$ требуется:
\begin{equation}
    0 = -\wp S = \int_0^T \int \left( \frac{\partial^2 X}{\partial t^2} - F \right) \cdot \wp X \; d\bar{q} \; dt
\end{equation}
Вводя в рассмотрение скорость жидкости $u$ и скорость деформации $U$:
\begin{gather}
    u(X(\bar{q}, t), t) = \frac{\partial X}{\partial t}(\bar{q}, t) \\
    U(X(\bar{q}, t), t) = \wp X(\bar{q}, t)
\end{gather}
можно показать, что $\wp X$ удовлетворяет условию несжимаемости в лагранжевых координатах, и можно записать следующие уравнения:
\begin{gather}
    \wp X(\bar{q}, t) = \int U(\bar{x}, t) \; \delta(\bar{x} - X(\bar{q}, t)) d\bar{x} \\
    \frac{\partial^2 X}{\partial t^2} \cdot \wp X(\bar{q}, t) = \int  \left(\frac{\partial u}{\partial t} + u \cdot \triangle u \right) U(\bar{x}, t) \; \delta(\bar{x} - X(\bar{q}, t)) d\bar{x}
\end{gather}
Подставляя эти формулы в (\ref{eq:least_action}) получим:
\begin{equation}
    0 = \int_0^T \int \int \left( \frac{\partial u}{\partial t} + u \cdot \triangle u - F(\bar{q}, t) \right) \cdot U(\bar{x}, t) \; \delta(\bar{x} - X(\bar{q}, t)) \; d\bar{x} \; d\bar{q} \; dt
\end{equation}
Произведя замену
\begin{equation}
    f(x, t) = \int F(\bar{q}, t) \; \delta(\bar{x} - X(\bar{q}, t)) \; d\bar{q}
\end{equation}
после ряда преобразований можно завершить переход от лагранжевых переменных к эйлеровым.

Помимо этого, в \cite{peskin2002immersed} показано, что несмотря на отсутствие предположений
о неоднородности плотности жидкости, уравнение:
\begin{equation}
    \frac{\partial \rho}{\partial t} + u \cdot \triangle \rho = 0
\end{equation}
есть следствие приведенной системы уравнений. Метод погруженной границы является следствием описанного случая,
в котором эластичный несжимаемый материал заполняет не всю область $\tilde{\Omega}$, а только часть, и представляет
собой поверхность, погруженную в жидкость.

\subsection*{Численная схема}

Введем в области $\tilde{\Omega}$ сетку $\tilde{\Omega_h}$ с шагами $h_x, h_y, h_z$, в области $\Gamma$ введем сетку $\Gamma_h$
с шагами $h_q, h_r, h_s$. Как показано в \cite{peskin2002immersed}, численная схема метода погруженной границы,
которая отвечает взаимодействию жидкости и погруженной границы, представлена следующими уравнениями:
\begin{gather}
    f(\bar{x}, t_n) = \sum_{\bar{q} \in \Gamma_h}\; F(\bar{q}, t) \; \delta_h (\bar{x} - X(\bar{q}, t)) \; h_q\; h_r\;  h_s\\
    \frac{dX}{dt}(\bar{q}, t_n) = \sum_{\bar{x} \in \tilde{\Omega_h}}\; u(\bar{x}, t) \; \delta_h (\bar{x} - X(\bar{q}, t)) \; h_x\; h_y\;  h_z
\end{gather}
Эти уравнения получены из (\ref{eq:interaction:velocity}), (\ref{eq:interaction:force}),
путем численного интегрирование с помощью квадратурных формул прямоугольников.
Как показано в \cite{peskin2002immersed}, $\delta$ и $\delta_h$ для любого $X$ должны удовлетворять следующим условиям:
\begin{gather}
    \int_{\tilde{\Omega}} \; \delta(\bar{x} - X(\bar{q}, t)) d\bar{x} = 1\\
    \sum_{\bar{x} \in \tilde{\Omega_h}} \; \delta_h (\bar{x} - X(\bar{q}, t)) h_x\; h_y\; h_z = 1
\end{gather}
Это означает, что сила в лагранжевых и эйлеровых переменных выражается тождественно:
\begin{align}
\begin{split}
    \sum_{\bar{x} \in \tilde{\Omega_h}} f(\bar{x}, t)\; h_x\; h_y\; h_z =\\
    & \sum_{\bar{x} \in \tilde{\Omega_h}} \sum_{\bar{q} \in \Gamma_h} F(\bar{q}, t)\; \delta_h (\bar{x} - X(\bar{q}, t))\; h_q\; h_r\; h_s = \\
    & \sum_{\bar{q} \in \Gamma_h} F(\bar{q}, t)\; h_q\; h_r\; h_s
\end{split}
\end{align}
независимо от вида $\delta_h$ (см. аналогичное доказательство для массы, энергии и вращающего момента в \cite{peskin2002immersed}).

\begin{figure}[h] 
  \center
  \includegraphics [width=8cm] {delta_schema.png}
  \caption{Схема применения функции $\delta_h$} 
  \label{img:delta_schema}
\end{figure}
Выбор вида функции $\delta_h$ является критически важным для описываемого метода. Цель этой функции продемонстрирована на рис. \ref{img:delta_schema}.
Она определяет то, с каким коэффициентами точки (т.е. насколько сильно) $(x, y, z) \in \tilde{\Omega_h}$ влияют на каждую точку $(q, r, s) \in \Gamma_h$.
Например, на рис. \ref{img:delta_schema} на точку $A$ с ненулевым коэффициентом влияют только точки, обозначенные серым цветом.
Для простоты эту функцию можно представить в следующем виде:

\begin{equation}
    \delta_h (\bar{x}) = \frac{1}{h_x h_y h_z}
    \phi \left( \frac{x}{h_x} \right)
    \phi \left( \frac{y}{h_y} \right)
    \phi \left( \frac{z}{h_z} \right)
\end{equation}

Как показано в \cite{peskin2002immersed}, функция $\delta_h$ должна удовлетворять следующим условиям:
\begin{enumerate}
    \item \label{item:delta_conditions:first} $\phi(r)$ является непрерывной для всех $r$, что позволяет избежать прижков в скорости или массовой силе,
        когда точка погруженной границы пересекает границу ячейки в $\tilde{\Omega_h}$
    \item \label{item:delta_conditions:second} $\phi(r) = 0, |r| \geq 2$, т.к. это позволяет снизить количество вычилений, при этом $|r| \geq 2$
        является минимальным значением, при котором это условие может быть согласовано с другими
    \item \label{item:delta_conditions:third} $\sum_{j чет} \phi(r - j) = \sum_{j нечет} \phi(r - j) = \frac{1}{2}$ для всех $r$ - это позволяет
        удостовериться, что массовые силы/скорость распределяются равномерно
    \item \label{item:delta_conditions:fourth} $\sum_{j} (r - j)\; \phi(r - j) = 0$ для всех $r$, что является необходимым условием для вывода тождественности
        выражения вращающего момента в лагранжевых и эйлеровых координатах
    \item \label{item:delta_conditions:fifth} $\sum_{j} (\phi(r - j))^2 = C$ для всех $r$, что определяет независимость расположения точек на лагранжевой сетке
        относительно точек эйлеровой сетки
\end{enumerate}
В \cite{peskin2002immersed} приводятся два возможных вида $\phi(r)$:
\begin{align}
    \label{eq:delta_new}
    \phi(r) &= 
        \begin{cases}
            0 & r \leq -2\\
            \frac{1}{8} \left( 5 + 2r - \sqrt[2]{-7 - 12r -4r^2} \right), & -2 \leq r \leq -1\\
            \frac{1}{8} \left( 3 + 2r + \sqrt[2]{1 - 4r -4r^2} \right), & -1 \leq r \leq 0\\
            \frac{1}{8} \left( 3 - 2r + \sqrt[2]{1 + 4r -4r^2} \right), & 0 \leq r \leq 1\\
            \frac{1}{8} \left( 5 - 2r - \sqrt[2]{-7 + 12r -4r^2} \right), & 1 \leq r \leq 2\\
        \end{cases}
    \\
    \label{eq:delta_old}
    \phi(r) &= 
        \begin{cases}
            \frac{1}{4} \left( 1 + cos\left( \frac{\pi r}{2} \right) \right),\qquad ||r|| \leq 2\\
            0 ,\qquad ||r|| \geq 2\\
        \end{cases}
\end{align}
Формула (\ref{eq:delta_new}) является немного более сложной, но она удовлетворяет всем условиям 
\ref{item:delta_conditions:first}-\ref{item:delta_conditions:fifth}, в то время как в данной работе
используеся формула (\ref{eq:delta_old}), которая является более простой, но не удовлетворяет условию \ref{item:delta_conditions:fifth}.

Еще одним важным моментом данного метода является дискретизация по времени уравнения (\ref{eq:interaction:velocity}),
т.к. явные и полу-явные схемы приводят к необходимости
использовать очень маленький шаг по времени, чтобы добиться лучшей устойчивоти решения (см. например, \cite{griffith2012immersed}).
В данной работе мы используем метод Адамса-Мультона. Рассмотрим $n$-й шаг по времени, где $t_{n+1}$, $t_n$ - время начала и конца шага,
$\triangle t = t_{n+1} - t_n$, $X_{n+1}, X_n$ - конфигурация погруженной границы в моменты времени $t_{n+1}, t_n$,
$u_{n+1}^{ijk}, u_n^{ijk}$ - скорость жидкости в точке $i,j,k$ в моменты времени $t_{n+1}, t_n$.
Как было показано выше, вначале нашего аглогитма мы находим $u_{n+1}^{ijk}$, решая систему уравнений Навье-Стокса.
Затем нам необходимо определить скорость возмущения погруженной границы из уравнения:
\begin{gather}
    \frac{dX}{dt} = f(X, t)
\end{gather}
где $f(X, t_n) = \sum_{ijk}\; u_{n+1}^{ijk} \; \delta_h(X, i, j, k) \; h_x\; h_y\;  h_z$.
Используя метод Адамса-Мультона 5-го порядка, мы получим:
\begin{equation}
    X_{n+1} = X_n + \triangle t \sum_{\lambda=-1}^{k-1} a_{-\lambda} f(X_{n-\lambda}, t_{n-\lambda})
\end{equation}
где $k=5$, а коэффициенты $a_{1, _{\dots}, k}$ представляют собой набор $\frac{251}{720}, \frac{646}{720}, -\frac{264}{720}, \frac{106}{720}, -\frac{19}{720}$.
Т.к. для применения данной схемы необходимо знать первые $k$ значений функции $f$, для первых $k$ итераций алгоритма
используется аппроксимация первого порядка $X_{n+1} = X_n + \triangle t f(X, t)$. В силу того, что в начале рассчета
еще не возникает сильных деформаций, это не влияет на общую устойчивость описанной схемы.

\subsection*{Верификация и сравнения}
В последние годы публикуется все больше работ, в которых представлены результаты тестирования метода погруженной границы,
а также сравнение результатов, полученных для практических задач, с реальными данными. В \cite{luo2012effect} продемонстрировано
сравнение расчетов течения жидкости в двумерном канале, где часть верхней стенки является гибкой, с помощью метода погруженной границы и
произвольного Лагранжево-Эйлерова метода (Arbitrary Lagrangian-Eulerian Method). Результаты демонстрируют хороший уровень соответвия
(см. рис. \ref{img:luo_ale_ibm}).

\begin{figure}[h] 
  \center
  \includegraphics [width=12cm] {luo_ale_ibm.jpg}
  \caption{
      Сравнение расчетов течения жидкости в двумерном канале методами IBM и ALE(FBM).
      \textbf{a)} координата по $Oy$ центра гибкой границы в зависимости от времени.
      \textbf{b)} форма гибкой границы в момент времени $t=17s$
  } 
  \label{img:luo_ale_ibm}
\end{figure}

В \cite{flamini2015immersed} продемонстрировано сравнение некоторых парамеров, полученных в результате расчетов,
с реальными данными, а также с другими работами. Например, объем, перекачиваемый за один удар сердца и пиковый объем
по значениям отличаются от клинических данных \cite{murgo1980aortic} не более, чем на 15\%, а пиковое давление на стенки
величиной $220kPa$ хорошо совпадают с литературными данными, в частости \cite{conti2010dynamic}. Параметры лепестков клапа
в этой работе определялись эмпирически, жесткость волокон была равна $7.5 \cdot 10^6 dyne/cm$

\bibliography{biblio}
