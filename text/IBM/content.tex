\section*{\Large Метод погруженной границы}
\fontsize{14pt}{15pt}\selectfont
\subsection*{Введение, основные термины}
Метод погруженной границы используется для описания систем <<жидкость-препятствие>>,
где эластичное препятствие погружено в вязкую несжимаемую жидкость.
Впервые был предложен в работе \cite{peskin1972flow} для моделирования механики
сердечных клапанов и потока крови в них.
Суть метода заключается в том, что при обтекании какого-либо тела жидкостью,
она испытывает влияние сил по направлению нормали к поверхности тела \cite{goldstein1993modeling}.
Обтекаемое тело также испытывает влияние этих сил с противоположным знаком.
Поэтому моделирование обтекания препятствия потоком жидкости возможно
с помощью формирования соответствующего поля внешних массовых сил в уравнении Навье-Стокса.
Это позволяет производить вычисления на простых прямоугольных сетках, которые могут не соответсвовать
геометрии расчетной области, что является одной из основных отличительных особенностей метода.
При этом под термином <<метод погруженной границы>> обычно понимают как математическую формулировку,
так и схему для численного решения полученной задачи. <<Погруженной границей>> в данном контексте
обозначают любое гибкое препятствие, погруженное в жидкость. В данной работе под этим будем подразумевать
стенки кровеносного сосуда, а также лепестки клапана, расположенные внутри.
Существует также множество модификаций этого метода (например, метод погруженного интерфейса,
метод декартовых сеток, метод фиктивных ячеек, метод усеченных ячеек, см. \cite{mittal2005immersed}),
но указаные методы не рассматриваются в данной работе.

Математическую формулировку этого метода и численную схему можно разделить на три раздела:
\begin{itemize}
    \item Моделирование течения вязкой несжимаемой жидкости
    \item Моделирование деформации погруженной границы
    \item Моделирование взаимодействия между жидкостью и погруженной границей
\end{itemize}

Течение вязкой несжимаемой жидкости описывается системой нелинейных дифференциальных уравнений Навье-Стокса
в прямоугольной области $\tilde{\Omega}$, которая включает в себя расчетную область $\Omega$. Погруженная граница представлена в виде
набора упругих безмассовых волокон, имеющих <<нейтральную плавучесть>> \cite{griffith2012immersed},
расположение которых описано в лагранжевых координатах, а эластичность описана в терминах функционала
энергии деформации. Взаимодействие осуществляется исходя из того, что погруженная граница движется под давлением
жидкости с той же скоростью, что и сама жидкости, а внешние массовые силы в уравнении Навье-Стокса определяются
поверхностным напряжением, возникшим в результате деформации погруженной границы. В следующих разделах
эти задачи будут рассмотрены подробнее.

\subsection*{Моделирование течения вязкой несжимаемой жидкости}

Для моделирования течения используется система дифференциальных уравнений Навье-Стокса:
\begin{gather}
    \label{eq:navier_stokes:motion}
    \frac{\partial \vec{u}}{\partial t} + (\vec{u} \cdot \nabla) \vec{u} = - \frac{1}{\rho} \nabla p + \nabla \sigma + \vec{f}\\
    \label{eq:navier_stokes:continuity}
    \frac{\partial \rho}{\partial t} + \nabla \cdot (\rho \vec{u}) = 0 
\end{gather}
с начальными и граничными условиями:
\begin{gather}
    \label{eq:navier_stokes:velocity_conditions}
    \vec{u}(\bar{x}, 0) = \vec{u}_0 \qquad \vec{u}|_{\Gamma_1, \Gamma_4} = \vec{u}_b \qquad u_{\Gamma_2, \Gamma3} = 0\\
    \label{eq:navier_stokes:pressure_conditions}
    p_{\Gamma_2} = p_{in} \qquad p_{\Gamma_3} = p_{out}
\end{gather}
где $\bar{x}=(x,y,z) \in \Omega$, $\vec{u}=(u,v,w)$ - вектор скорости, $u, v, w$ - $x$-, $y$-, $z$-компоненты вектора скорости,
$\vec{u}_b$ - скорость движения погруженной границы (стенок сосуда и лепестков клапана) при деформации,
$\rho=\rho(\bar{x}, t)$ - плотность, $p=p(\bar{x}, t)$ - давление, $\sigma = \mu (\nabla \vec{u} + (\nabla \vec{u})^T)$ - вязкий тензор напряжений,
$\mu = \mu(\bar{x}, t)$ - вязкость жидкости, $\vec{f} = \vec{f}(\bar{x}, t)$ - вектор массовых сил.
Область $\Omega$ представляет собой сосуд с границами $\Gamma = \Gamma_1 \cup \Gamma_2 \cup \Gamma_3 \cup \Gamma_4$,
где $\Gamma_1$ - стенка кровеносного сосуда, $\Gamma_2$ and $\Gamma_3$ - область втекания/вытекания,
$\Gamma_4$ - лепестки клапана (см Рис. \ref{img:aorta_boundaries}).

\begin{figure}[h] 
  \center
  \includegraphics [width=14cm] {aorta_valve_scheme.png}
  \caption{Изображение границ расчетной области} 
  \label{img:aorta_boundaries}
\end{figure}

\subsection*{Моделирование деформации погруженной границы}
Опишем погруженную границу как гибкий несжимаемый материал, который расположен в области $\tilde{\Omega}$,
где $(q, r, s)$ - криволинейные координаты связанные с материалом так, что зафиксированное значение
$(q, r, s)$ обозначает одну точку этого материала. Пусть $X(q, r, s, t)$ - позиция в декартовых координатах
точки, обозначенной $(q, r, s)$ в момент времени $t$. Тогда обобщая, $X(q, r, s, t)$ описывает пространственную
конфигурацию всей погруженной границы в момент времени $t$, которая определяет соответствующую энергию деформации
$E[X(q, r, s, t)]$ в момент $t$. Рассмотрим возмущение $\delta X(q, r, s, t)$ конфигурации $X(q, r, s, t)$,
где $\delta$ - оператор возмущения.
%NOTE: перевести точнее, "up to terms of first order" - с точностью первого порядка?
Результирующая деформация упругой энергии, возникшая в результате этого возмущения,
есть линейная функция от возмущения конфигурации, поэтому она может быть записана в форме:
\begin{equation}
\label{eq:elastic_energy_functional}
\delta E[X(q, r, s, t)] = \int_{\tilde{\Omega}} (-F(q, r, s, t)) \cdot \delta X(q, r, s, t)\; dq\; dr\; ds
\end{equation}
где $-F(q, r, s, t)$ - производная Фреше от $E$, вычисленная для конфигурации $X(q, r, s, t)$. Физически $F$
можно интепретировать, как плотность силы, которая создается путем деформирования погруженной границы.

Как было сказано выше, погруженная граница представлена набором волокон. Для того, чтобы определить $E$,
удобно выбирать лагранжевые координаты $(q, r, s)$ так, что каждое значение $(q, r)$ параметрически задает
какое-либо одно конкретное волокно $s \to X(q^0, r^0, s)$. В этом случае функционал упругой энергии $E$
можно представить в виде $E = E_s + E_b$, где $E_s$ - энергия растяжения волокна, $E_b$ - энергия скручивания волокна,
а силу $F$, созданную благодаря деформации, в виде $F = F_s + F_b$.

Как показано в \cite{peskin2002immersed}, \cite{griffith2009simulating} функционал энергии растяжения можно записать в виде:
\begin{equation}
\label{eq:stretching_energy_functional}
E_s = \int_{\tilde{\Omega}} \varepsilon_s \left(\left| \frac{\partial X}{\partial s} \right| \right) dq \; dr \; ds
\end{equation}
а силу $F_s$:
\begin{equation}
\label{eq:stretching_force_density}
F_s = \frac{\partial}{\partial s} \varepsilon_s^{,} \left( \left| \frac{\partial X}{\partial s} \right| \right) \tau
\end{equation}
где $\varepsilon_s$ - локальная энергия растяжения, $\tau$ - единичный тангенциальный вектор для выбранного волокна $s$.
Т.к. напряжение растяжения $T = \varepsilon_s^{,} \left( \left| \frac{\partial X}{\partial s} \right| \right)$, то уравнение (\ref{eq:stretching_force_density})
можно переписать в виде, который соответствует обобщенному закону Гука:
\begin{equation}
\label{eq:stretching_force_density_simplified}
F_s = \frac{\partial}{\partial s} T \tau
\end{equation}

Т.к. площать попреречного сечения сосудов достаточно мала по отношению к их длинне, для моделирования энергии напряжения и 
сил сопротивления скручиванию мы можем воспользоваться уравнением Эйлера-Бернулли \cite{gere1997mechanics}:
% also see here https://en.wikipedia.org/wiki/Euler%E2%80%93Bernoulli_beam_theory
\begin{equation}
    E_b = \frac{1}{2} \int_{\tilde{\Omega}} k \cdot \left| \frac{\partial^2 X^0}{\partial s^2} - \frac{\partial^2 X}{\partial s^2} \right|^2 \; dq\; dr\; ds
\end{equation}

\begin{equation}
    F_b = \frac{\partial^2}{\partial s} \left( k \cdot \left(\frac{\partial^2 X^0}{\partial s^2} - \frac{\partial^2 X}{\partial s^2} \right) \right)
\end{equation}
где $k = E \cdot I$, $E$ - модуль упругости, $I$ - момент инерции поперечного сечения, $\frac{\partial^2 X^0}{\partial s^2}$, $\frac{\partial X}{\partial s^2}$ -
отклонение погруженной границы от равновесного положения в начальный и текущий момент времени.

Таким образом, плотность силы $F$, создаваемая при деформации погруженной границы может быть выражена в виде:
\begin{equation}
    F = \frac{\partial}{\partial s} T \tau + \frac{\partial^2}{\partial s} \left( k \cdot \left(\frac{\partial^2 X^0}{\partial s^2} - \frac{\partial^2 X}{\partial s^2} \right) \right)
\end{equation}

\subsection*{Моделирование взаимодействия}

Сердечно-сосудистые заболевания являются одной из наиболее острых проблем современного общества. Во всех странах их количество существенно опережает остальные, поэтому трудно переоценить значимость исследований в этой области. В последние годы наблюдается резкий рост интереса к проблеме сердечно-сосудистых заболеваний, развиваются новые методики исследования, появляются все более точные измерительные приборы. Каждый год в мире проводится примерно 250 000 операций восстановлению или замене поврежденных сердечных клапанов\footnote{
    Yoganathan A.P., He Z.M., Jones S.C.: Fluid mechanics of heart valves. Annu. Rev. Biomed Eng 6:331--362 (2004)
} % replace with new statistics
и ожидается, что в ближайшие годы это значение будет только увеличиваться\footnote{
    Yacoub N, Takkenberg J.: Will heart valve tissue engineering change the world? Nat Clin Prac Cardiovas Med. 2:60--1 (2005)
}. При этом многие сложности, связанные с созданием искусственных клапанов или протезированием сосудов, относятся к динамике течения крови внутри. Поэтому математическое моделирование данных явлений позволяет получить более глубокое понимание происходящих процессов и найти пути усовершенствования их конструкции. 

При изучении подобных явлений методами математического моделирования зачастую удовлетворительные результаты можно получить с помощью модели вязкой несжимаемой жидкости, которая описывается системой дифференциальных уравнений Навье-Стокса, выписанных в форме естественных переменных <<скорость-давление>>. Помимо этого требуется учесть, что кровь является неоднородной по своей природе и состоит из плазмы и форменных элементов (лейкоциты, эритроциты и т.д.), а сосуды и клапаны являются гибкими и изменяют форму под воздействием различных параметров. Необходимость описывать взаимодействия гибких непроницаемых тканей с неоднородной жидкостью приводит к существенным трудностям при постановке задачи и ее численном решении, связанными с построением расчетной сетки и ее изменением в соотвтетсвии с движением лепестков клапана и деформацией сосуда. 

Существует несколько устоявшихся подходов, для того, чтобы избежать эти трудности. В данном исследовании используется метод погруженной границы, который предназначен для моделирования тонких препятствий произвольной жесткости. Это позволяет численно решать прикладные задачи оптимизации структуры искусственного клапана.


\textbf{Целью} данной работы является разработка технологии решения нестационарной трехмерной задачи о движении створок искусственного клапана внутри крупных кровеносных сосудов с учетом неоднородной структуры крови, а также о движении примеси (форменных элементов) внутри сосуда. Для достижения этой цели был создан программный комплекс, с помощью которого можно моделировать работу клапана, деформацию стенок кровеносных сосудов и получать картины течения внутри них.

Диссертационная работа была выполнена в рамках проектной части госзадания 1.630.1.2014/K.

\textbf{Основные положения, выносимые на~защиту:}
\begin{enumerate}
 \item Технология численного решения задачи о течении вязкой несжимаемой неоднородной жидкости
     в канале с гибкими границами при заданном перепаде давления, включающая алгоритм взаимодействия
     жидкости с границей, а также способ расчета движения примесей.
 \item Программный комплекс, предназначенный для моделирования движения вязкой несжимаемой неоднородной
     жидкости, вызванного перепадом давления, в каналах сложной формы с гибкими границами. Комплекс содержит
     библиотеку классов для рашения задач в указанной постановке, модуль для подготовки геометрии расчетной области,
     модуль для анализа полученных данных и их визуализации.
 \item Результаты расчетов трехмерных задач о работе искусственного сердечного клапана, а также течения вязкой несжимаемой
     неоднородной жидкости в крупных кровеносных сосудах.
% и так далее, если нужно
\end{enumerate}

\textbf{Научная новизна:}
\begin{enumerate}
 \item Предложена оригинальная технология решения трехмерной задачи о движении вязкой несжимаемой неоднородной
     жидкости в каналах сложной формы с гибкими границами.
 \item С помощью созданного программного комплекса впервые были решены трехмерные задачи о работе створок клапана, а также о деформации
     стенок крупных кровеносных сосудов с учетом неоднородной структуры крови.
\end{enumerate}

\textbf{Практическая значимость} диссертационной работы заключается в возможности использования полученных методов
и разработанного программного комплекса для оптимизации строения искусственных сердечных клапанов.

\textbf{Достоверность} изложенных в работе результатов обеспечивается использованием полностью консервативных схем для аппроксимации
решаемой дифференциальной задачи, применением сходящихся итерационных методов для решения систем нелинейных алгебраических уравнений, а также
совпадением полученных результатов методических расчетов с решениями, полученными другими авторами.

\textbf{Представление работы.}
Основные результаты работы докладывались и обсуждались на следующих конференциях:
Международной конференции <<Информационно-вычислительные технологии и математическое моделирование (ИВТ\&ММ)>>, Россия, Кемерово, 2013 г.;
XIII Международной научно-практической конференции <<Информационные технологии и математическое моделирование>> им. А. Ф. Терпугова, Россия, Анжеро-Судженск, 2014 г.;
Международной научной студенческой конференции, Россия, Новосибирск, 2015 г.;
VIII Международной конференции, посвященной 115-летию со дня рождения академика Михаила Алексеевича Лаврентьева, Россия, Новосибирск, 2015 г.;
International Conference <<Computational and Informational Technologies in Science, Engineering and Education>>, Казахстан, Алматы, 2015 г.;
III International Conference in memory of V.I.Zubov <<Stability and Control Processes>>, Россия, Санкт-Петербург, 2015 г.;

Также, результаты работы докладывались на семинаре <<Математические модели. Методы решения>>, Кемерово (под рук. проф. Ю.Н. Захарова). 

\textbf{Публикации.} Основные результаты по теме диссертации изложены в 7 печатных изданиях, 1 из которых изданы в журналах, рекомендованных ВАК, 2 -- в прочих изданиях, 4 --- в тезисах докладов.


\textbf{Личный вклад.} Результаты работы базируются на ряде фундаментальных исследований, которые относятся
к методу погруженной границы\footnote{
    Peskin, C. S.: The immersed boundary method. Acta Numerica 11, 479–517 (2002).
}, который позволяет расчитывать движение сколь угодно тонких лепестков под давлением жидкости, а также подходам
к решению задач о течении вязкой несжимаемой неоднородной жидкости под воздействием перепада давления\footnote{
    Gummel E.E., Milosevic H., Ragulin V.V., Zakharov Y.N., Zimin A.I.: Motion of viscous inhomogeneous incompressible fluid of variable viscosity. Zbornik radova konferencije MIT 2013, Beograd, 267-274 (2014)
}. В настоящей работе автор объединил эти подходы и реализовал программный комплекс для
численного решения задачи о движении створок сердечного клапана, который использует эти методы.


%\underline{\textbf{Объем и структура работы.}} Диссертация состоит из~введения, четырех глав, заключения и~приложения. Полный объем диссертации \textbf{ХХХ}~страниц текста с~\textbf{ХХ}~рисунками и~5~таблицами. Список литературы содержит \textbf{ХХX}~наименование.

%\newpage
\subsection*{\Large Содержание работы}
Диссертация состоит из введения, трех глав, заключения и приложений. Во \textbf{введении} обосновывается актуальность выбранной темы диссертационной работы, приводится обзор научной литературы по изучаемой проблеме, излагаются цели и задачи исследования, формулируются основные положения, выносимые на защиту.

\textbf{Первая глава} посвящена описанию используемой математической модели, применяемых методов решения, разностных схем, а также используемого программного комплекса.
В \textbf{параграфе 1.1} рассматривается система дифференциальных уравнений Навье-Стокса, которая моделирует нестационарную задачу о течении вязкой неоднородной несжимаемой жидкости под воздействием перепада давления:
\begin{gather}
    \label{eq:navier_stokes:motion}
    \frac{\partial \vec{u}}{\partial t} + (\vec{u} \cdot \nabla) \vec{u} = - \frac{1}{\rho} \nabla p + \nabla \sigma + \vec{f}\\
    \label{eq:navier_stokes:continuity}
    \frac{\partial \rho}{\partial t} + \nabla \cdot (\rho \vec{u}) = 0 
\end{gather}
с начальными и краевыми условиями:
\begin{gather}
    \label{eq:navier_stokes:velocity_conditions}
    \vec{u}(\bar{x}, 0) = \vec{u}_0 \qquad \vec{u}|_{\Gamma_1, \Gamma_4} = \vec{u}_b \qquad u_{\Gamma_2, \Gamma3} = 0\\
    \label{eq:navier_stokes:pressure_conditions}
    p_{\Gamma_2} = p_{in} \qquad p_{\Gamma_3} = p_{out}
\end{gather}
где $\bar{x}=(x,y,z) \in \Omega$, $\vec{u}=(u,v,w)$ - вектор скорости, $\vec{u}_b$ - скорость, с которой двигаются стенки сосуда и створки клапана при деформации,
$\rho=\rho(\bar{x}, t)$ - плотность, $p=p(\bar{x}, t)$ - давление, $\sigma = \mu (\nabla \vec{u} + (\nabla \vec{u})^T)$ - вязкий тензор напряжений,
$\mu = \mu(\bar{x}, t)$ - вязкость жидкости, $\vec{f} = \vec{f}(\bar{x}, t)$ - вектор массовых сил, который в дальнейшем используется для определения формы сосуда и створок клапана. 

Область $\Omega$ изображена на рис. \ref{img:boundaries} и представляет собой сосуд с границей $\Gamma = \Gamma_1 \cup \Gamma_2 \cup \Gamma_3 \cup \Gamma_4$,
где $\Gamma_1$ - стенка кровеносного сосуда, $\Gamma_2$ и $\Gamma_3$ -  области втекания и вытекания, $\Gamma_4$ - створки клапана.

Одна из сложностей при численном решении подобных задач заключается в отсутствии одного компонента вектора скорости на границах $\Gamma_2$,
$\Gamma_3$. Для того, чтобы решить эту проблему, в данной работе на указанных границах используется оригинальные уравнения (\ref{eq:navier_stokes:motion}) - 
(\ref{eq:navier_stokes:pressure_conditions}), что позволяет определить недостающие компоненты.

\begin{figure}[h] 
  \center
  \includegraphics [width=10cm] {common_model.png}
  \caption{Изображение границ расчетной области} 
  \label{img:boundaries}
\end{figure}

В \textbf{параграфе 1.2} рассматриваются уравнения, используемые для моделирования изменения плотности и вязкости жидкости, а также концентрации примесей.
Плотность $\rho$ и вязкость $\mu$ определяются следующими соотношениями:
\begin{gather}
    \label{eq:viscosity}
    \mu = c (\mu_2 - \mu_1) + \mu_1\\
    \label{eq:density}
    \rho = c (\rho_2 - \rho_1) + \rho_1
\end{gather}
где $\rho_1$, $\mu_1$ - плотность и вязкость жидкости (плазмы), $\rho_2$, $\mu_2$ - плотность и вязкость примеси (форменных элементов), $c$ - концентрация примеси.
Концентрация $c=c(\bar{x}, t)$, $c \in [0, 1]$ примеси определяется как решение уравнения:
\begin{gather}
    \label{eq:convection}
    \frac{\partial c}{\partial t} + \vec{u} \cdot \nabla c = 0
\end{gather}

с начальными условиями и краевыми условиями на границе втекания:
\begin{gather}
    \label{eq:convection:conditions}
    c(\bar{x}, 0) = c_0(\bar{x}), \bar{x} \in \Omega \qquad c(\bar{x}, t)|_{\Gamma_2} = c_s(\bar{x}, t)
\end{gather}

В \textbf{параграфе 1.3} рассматривается задача моделирования движения стенок сосуда и лепестков клапана,
а также их взаимодействия с течением жидкости. Деформация $\Gamma_1 \cup \Gamma_4$ под воздействием давления жидкости
определяется силами, которые возвращают их в равновесное состояние. При этом створки клапана $\Gamma_4$ могут 
деформироваться гораздо сильнее, чем стенки сосуда $\Gamma_1$. Для описания сил, возникающих при деформации клапана,
мы используем следующую формулу:
\begin{gather}
    \label{eq:boundary_force}
    F = \frac{\partial}{\partial s} (T \tau) + \frac{\partial^2}{\partial s^2} (E \cdot I \frac{\partial^2}{\partial s^2} X)
\end{gather}
где $\bar{q}=(q, r, s) \in \Gamma_4$, $X(\bar{q})$ - функция, описывающая поверхность створок клапана в момент времени $t$,
координаты $q, r, s$ мы выбираем так, чтобы поверхность $X$ была представлена набором параметрических линий $s \rightarrow X(q^0, r^0, s)$,
$T$ -напряжение, возникающее при растяжении вдоль $s$, $\tau$ - единичный вектор, касательный к поверхности клапана, $E$ - модуль Юнга,
$I$ - момент инерции поперечного сечения. Формула (\ref{eq:boundary_force}) позволяет учитывать любые изменения формы створок клапана.  
Для вычисления сил, возникающих при изменении формы сосуда, мы используем другую формулу, которая позволяет учитывать только 
небольшие изменения формы:
\begin{gather}
    \label{eq:boundary_force_simple}
    F = k \|X - X_0\|
\end{gather}
где $\bar{q} = (q, r, s, t) \in \Gamma_1$, $X(\bar{q}, t)$, $X_0(\bar{q}, 0)$ - функции, которые описывают поверхность сосуда в момент времени $t$
и в начальный момент времени, $k$ - коэффициент жесткости.

Для того, чтобы учитывать взаимодействие стенок сосуда и створок клапана с течением жидкости, необходимо на основе силы $F$
вычислить поле внешних сил $f$ в уравнении Навье-Стокса и исходя из поля скоростей жидкости $\vec{u}(\bar{x}, t)$
определить текущую форму $X(bar{q}, t)$ сосуда и клапана. Это делается с помощью следующих уравнений:
\begin{gather}
    \label{eq:interaction:velocity}
    \frac{\partial X}{\partial t}(\bar{q}, t) = \int_{\Omega} \vec{u}(\bar{x}, t) \cdot \delta (x - X(\bar{q}, t))\; dx\; dy\; dz\\
    \label{eq:interaction:force}
    \vec{f}(\bar{x}, t) = \int_{\Gamma} \vec{F}(\bar{q}, t) \cdot \delta (x - X(\bar{q}, t))\; dq\; dr\; ds
\end{gather}
где $\bar{q} = (q, r, s) \in \Gamma$ - точка поверхности сосуда или клапана, $X = X(\bar{q}, t)$ - функция, описывающая поверхность сосуда и клапана
в момент времени $t$, $F = F(\bar{q}, t)$ - сила сопротивления деформации в данной точке,
$\vec{u}(\bar{x}, t)$ - вектор скорости течения, $\vec{f}(\bar{x}, t)$ - вектор массовых сил, $\delta$ - дельта-функция Дирака.

В \textbf{параграфе 1.4} рассматриваются методы, используемые для решения поставленной задачи, приводятся известные теоремы
о существовании и единственности решения.

Для решения поставленных краевых задач используется метод погруженной границы, который основывается на том, что при 
обтекании какого-либо тела жидкостью, она испытывает влияние силы, действующие по направлению нормали к поверхности тела, а также 
сдвиговые силы, если на границе тела поставлено условие прилипания. Поверхность тела также испытывает влияние тех же сил с 
противоположным знаком. Поэтому моделирование обтекания тела возможно с помощью формирования соответствующего поля внешних 
массовых сил.
В результате поставленная задача разбивается на несколько более мелких. Вводятся новые области $\tilde{\Omega}$, которая представляет собой
параллепипед, содержащий в себе $\Omega$, а также $\Gamma$ с лагранжевой системой координат, которая соотносится со стенками сосуда и лепестками
клапана.
Помимо этого вводятся две сетки $\tilde{\Omega_h}$ с шагами $h_x$, $h_y$, $h_z$, и $\tilde{\Gamma_h}$, с шагами
$h_q$, $h_r$, $h_s$. $\tilde{\Omega_h}$ является обычной декартовой сеткой и предназначена для расчет течения жидкости,
$\tilde{\Gamma_h}$ - для расчета деформации стенок сосуда и клапана.

В \textbf{параграфе 1.5} описывается численное решение задачи о течении вязкой несжимаемой жидкости с помощью
схемы расщепления по физическим факторам:
\begin{gather}
    \label{eq:splitting:intermediate_velocity}
    \frac{u^* - u^n}{\triangle t} = - (u^n \cdot \nabla) u^* - \frac{1}{\rho} \nabla \sigma + f^n\\
    \label{eq:splitting:poisson}
    \rho \triangle p^{n+1} - \nabla \rho \cdot p^{n+1} = \frac{\rho^2 \nabla u^*}{\triangle t}\\
    \label{eq:splitting:velocity}
    \frac{u^{n+1} - u^*}{\triangle t} = - \frac{1}{\rho} \triangle p^{n+1}
\end{gather}

Численная реализация этой схемы состоит из трех шагов. Вначале вычисляется промежуточное поле скоростей $u^{*}$,
для этого методом стабилизирующей поправки решается уравнение (\ref{eq:splitting:intermediate_velocity}).
После этого с помощью метода бисопряженных градиентов из уравнения (\ref{eq:splitting:poisson}) определяется новое поле давления и 
по формулам (\ref{eq:splitting:poisson}) расчитывается итоговое поле скоростей.

В \textbf{параграфе 1.6} рассматривается численное решение уравнения переноса примесей, а также способ расчета вязкости и плотности жидкости.
Для этого методом стабилизирующей поправки решается следующее уравнение:
\begin{gather}
    \label{eq:numerical_concentration}
    \frac{c^{n+1} - c^{n}}{\triangle t} + u^{n} \cdot \nabla c^{n+1} = 0
\end{gather}
после чего по явным формулам (\ref{eq:viscosity}), (\ref{eq:density}) вычисляются значения $\mu_{n+1}$, $\rho_{n+1}$.

В \textbf{параграфе 1.7} описывается метод численного решения задачи деформации гибких лепестков клапана,
а также взаимодействия "жидкость-клапан".
Деформация определяется по явным формулам (\ref{eq:boundary_force}),(\ref{eq:boundary_force_simple}) исходя из формы сосуда в текущий $X_{n}$ и предыдущий $X_{n-1}$ шаг по времени.
Взаимодействие "жидкость-клапан" расчитывается с помощью формул (\ref{eq:interaction:velocity}), (\ref{eq:interaction:force}) в конечно-разностном виде,
где интегралы вычислены с помощью квадратурной формулы прямоугольников:
\begin{gather}
    \label{eq:numerical_interpolation}
    U_n = \sum_{ijk}u_{ijk} \cdot D(x_{ijk} - x_n) h_{ijk}^3 \\
    \label{eq:numerical_spreading}
    f_{ijk} = \sum_n F_n \cdot D(x_{ijk} - x_n) h^2_n
\end{gather}
где $U_n$, $F_n$ - скорость движения лепестков клапана под воздействием давления жидкости и сила сопротивления деформации,
$u_n$, $f_n$ - скорость движения жидкости и массовые силы в уравнении (\ref{eq:navier_stokes:motion}),
$D$ - конечно-разностный аналог функции Дирихле, который в данной работе представлен следующим соотношением:
\begin{gather}
    \label{eq:numerical_dirichlet}
    \begin{cases}
        D(r) = \frac{1}{4h} (1 + cos(\frac{\pi r}{2h})) & |r| < 2h\\
        D(r) = 0 & |r| > 2h\\
    \end{cases}
\end{gather}


В \textbf{параграфе 1.8} дается описание программного комплекса,
предназначеного для моделирования работы створок клапана
под воздействием давления жидкости. Он является модулем для комплекса
<<XFlow>> и содержит библиотеки классов, реализующих описанный выше алгоритм,
а также компоненты пред- и постобработки. Эти компоненты предоставляют
возможность использовать для расчета CAD модели реальных объектов,
полученных путем сканирования, задавать их внутреннюю структуру, т.е.
связи между отдельными узлами, а также проводить визуализацию и анализ
полученных результатов с помощью ParaView и Matplotlib.

\textbf{Вторая глава} посвящена тестированию предложенной технологии решения задачи обтекания гибких препятствий,
помещенных в поток вязкой несжимаемой неоднородной жидкости, а также решению задачи о развитии аневризмы
на стенке кровеносного сосуда.

В \textbf{параграфе 2.1} представлено решение тестовой задачи о течении вязкой несжимаемой однородной жидкости
под воздействием перепада давления внутри канала цилиндрической формы. Приведены результаты расчетов для разных
параметров сетки, на Рис. \ref{img:velocity_profile} показано сравнение рассчитанной скорости в центре канала
с соответствующим аналитическим решением.
\begin{figure}[h] 
  \center
  \includegraphics [width=14cm] {velocity_profile.png}
  \caption{Сравнение профиля скорости, полученной в расчетах, с аналитическим решением} 
  \label{img:velocity_profile}
\end{figure}

В \textbf{параграфе 2.2} представлено решение тестовой задачи об обтекании сферы вязкой несжимаемой однородной жидкостью.

В \textbf{параграфе 2.3} описаны результаты решения задачи о моделировании возникновения аневризмы на стенках кровеносного сосуда.
Постановка задачи аналогична описанной в \textbf{главе 1}, но т.к. исследуется явление аневризмы, в сосуде отсутствует клапан
(см. Рис. \ref{img:aneurysm_boundaries}).  На стенках сосуда выделена область $\Gamma_5$, на которой жесткость задана меньше, чем для остальной части сосуда.
Это позволяет моделировать истончение стенки сосуда и возникновение аневризмы под воздействием давления жидкости.
\begin{figure}[h] 
  \center
  \includegraphics [width=14cm] {aneurysm_schema.png}
  \caption{Изображение границ расчетной области} 
  \label{img:aneurysm_boundaries}
\end{figure}

\textbf{Третья глава} посвящена решению задачи о движении лепестков искусственного сердечного клапана 
под воздействием давления вязкой несжимаемой неоднородной жидкости, a также верификации описанной модели. Приводятся
результаты расчетов работы клапана, их сравнение с экспериментальными данными, а также сравнение с результатами,
полученными в других работах.
Постановка задачи аналогична описанной в \textbf{главе 1}, где поверхность сосуда является твердой стенкой. 
Схема расположения клапана изображена на Рис. \ref{img:aorta_boundaries}

В \textbf{параграфе 3.1} представлены результаты решения задачи динамике искусственного сердечного клапана
для случая однородной жидкости (см. Рис \ref{img:valve_dynamics}).

\begin{figure}[H] 
  \center
  \includegraphics [width=8cm] {valve_1.png}

  \includegraphics [width=8cm] {valve_2.png}

  \includegraphics [width=8cm] {valve_3.png}
  \caption{Динамика деформации лепестков клапана} 
  \label{img:valve_dynamics}
\end{figure}

В \textbf{параграфе 3.2} описаны результаты решения задачи динамике искусственного сердечного клапана
для случая неоднородной жидкости (см. Рис \ref{img:valve_dynamics}).

\begin{figure}[H] 
  \center
  \includegraphics [width=8cm] {valve_in_mixture1.png}

  \includegraphics [width=8cm] {valve_in_mixture2.png}

  \includegraphics [width=8cm] {valve_in_mixture3.png}
  \caption{Динамика деформации лепестков клапана} 
  \label{img:valve_dynamics}
\end{figure}

В \textbf{параграфе 3.3} приведены результаты распределения напряжения по поверхности клапана и фиброзного кольца в процессе деформации.

В \textbf{параграфе 3.4} приводятся сравнения полученных результатов с работами других авторов\footnote{
    Griffith B.E., Immersed boundary model of aortic heart valve dynamics with physiological driving and loading conditions. // Int J Numer Meth Biomed Eng, 28:317-345, 2012 
}и экспериментальными данными.

В \textbf{заключении} кратко приводятся основные результаты работы:
\begin{enumerate}
 \item Предложенный в работе подход позволяет успешно решать трехмерные задачи о теении вязкой неоднородной несжимаемой жидкости в сосудах с деформируемыми стенками и тонкими гибкими препятствиями.
 \item Разработанный программный комплекс может применяться для моделирования широкого круга задач, связанных с работой сердечного клапана и течением крови в сосудах.
\end{enumerate}


%\newpage
\renewcommand{\refname}{\Large Публикации автора по теме диссертации}
\nocite{*}
\bibliography{biblio}
