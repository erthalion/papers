\documentclass[14pt, compress]{beamer}

% can be compiled by xelatex -shell-escape presentation.tex

\usetheme[usetitleprogressbar]{m}

\usepackage[utf8]{inputenc}
\usepackage[russian, english]{babel}
\usepackage{booktabs}
\usepackage[scale=2]{ccicons}
\usepackage{listings}
\usepackage{marvosym}
\usepackage{color}
\usepackage{xcolor}
\usepackage[document]{ragged2e}
\usepackage{tikz}
\usepackage{adjustbox}
\usepackage{fontawesome} 
\usepackage{enumitem}

\usetikzlibrary{shapes,arrows,positioning}
\graphicspath{{images/}}
\newfontfamily{\FA}{FontAwesome}

\def\twitter{{\FA \faTwitter}}
\def\github{{\FA \faGithubSign}}
\def\email{{\FA \faEnvelope}}

\tikzstyle{block} = [rectangle, draw, fill=blue!20, text width=11em, text centered, rounded corners, minimum height=2em]
\tikzstyle{line} = [draw, -latex]

\definecolor{light-gray}{gray}{0.95}

\usepgfplotslibrary{dateplot}
\renewcommand{\ttdefault}{pcr}
\newfontfamily{\ttfamily}{Fira Code}

\lstset{
  language=SQL,
  escapeinside={(*@}{@*)},
  basicstyle=\tiny\ttfamily,
  %keywordstyle=\bfseries,
  keywordstyle=\color{blue}\ttfamily,
  backgroundcolor=\color{light-gray},
  breaklines=true,
  framexleftmargin=10pt,
  framexrightmargin=10pt,
  framextopmargin=11pt,
  framexbottommargin=10pt,
}

\title{Все, что вы хотели знать о jsonb, но боялись спросить.}
\subtitle{Справочник по синтаксису.}
\date{\today}
\author{Дмитрий Долгов}
\institute{}

\begin{document}

\maketitle

\section{}

\begin{frame}[fragile]
    \frametitle{History}

    \begin{adjustbox}{max totalsize={1.0\textwidth}{.7\textheight},center}
        \begin{tikzpicture}[node distance = 2.5cm, auto]
            \node[block] (hstore){
                Hstore
            };

            \node[block, below of=hstore] (nested_hstore){
                Nested Hstore
            };

            \node[block, below of=nested_hstore] (jsonb){
                Jsonb
            };

            \path[line] (hstore) -- (nested_hstore);
            \path[line, dashed] (nested_hstore) -- (jsonb);

        \end{tikzpicture}
    \end{adjustbox}
\end{frame}
\note{
  Hstore is a key-value binary storage.
  Doesn't support tree-like nested structures,
  but there is a nested version of hstore.

  Binary JSON storage. JSONB was introduced in PostgreSQL 9.4 and
  supports fast lookups and simple expression search queries using
  Generalized Inverted Indexes (GIN). It supposed to be document-oriented
  and was designed for the schema-less data.
}

\begin{frame}[fragile]
    \frametitle{Many faces of jsonb}

  \begin{itemize}
      \item<+->
          \begin{lstlisting}[
              basicstyle=\footnotesize\ttfamily,
              showstringspaces=false,
          ]
select '"Ivan Pomidorov"'::jsonb;
         
          \end{lstlisting}

      \item<+->
          \begin{lstlisting}[
              basicstyle=\footnotesize\ttfamily,
              showstringspaces=false,
          ]
select '42'::jsonb;
         
          \end{lstlisting}

      \item<+->
          \begin{lstlisting}[
              basicstyle=\footnotesize\ttfamily,
              showstringspaces=false,
          ]
select 'true'::jsonb;
         
          \end{lstlisting}

      \item<+->
          \begin{lstlisting}[
              basicstyle=\footnotesize\ttfamily,
              showstringspaces=false,
          ]
select '[1, 2, 3]'::jsonb;
         
          \end{lstlisting}

      \item<+->
          \begin{lstlisting}[
              basicstyle=\footnotesize\ttfamily,
              showstringspaces=false,
          ]
select '["Ivan Pomidorov", 42]'::jsonb;
         
          \end{lstlisting}

      \item<+->
          \begin{lstlisting}[
              basicstyle=\footnotesize\ttfamily,
              showstringspaces=false,
          ]
select '{"warm": {"hot": "data"}}'::jsonb;
         
          \end{lstlisting}

  \end{itemize}

\end{frame}
\note{
 * When a naked
 * scalar value needs to be stored as a Jsonb value, what we actually store is
 * an array with one element, with the flags in the array's header field set
 * to JB\_FSCALAR | JB\_FARRAY.

 Note about deduplication and space trimming
}

\begin{frame}
    \frametitle{Jsonb in 9.4}
    \vspace{-30pt}
    Is jsonb fully functional in 9.4?
    \begin{figure}
        \includegraphics[width=10cm]{cat.jpg}
    \end{figure}

    \vspace{-10pt}
    Yes and no.

\end{frame}

\begin{frame}[fragile]
  \frametitle{Lack of some functionality}

  \begin{itemize}
      \item[\MVRightarrow] Get an element at arbitrary path ( \#> )
      \item[\MVRightarrow] Delete an element at arbitrary path  ( ? )
      \item[\MVRightarrow] Update an element at arbitrary path ( ? )
      \item[\MVRightarrow] Add a new element to arbitrary path ( ? )
  \end{itemize}

\end{frame}

\begin{frame}[fragile]
  \frametitle{jsonbx}

  Pgxn extension for PostgreSQL 9.4, which contains implementation of
  some missing functionality. It based on nested version of hstore
  and provided this functions for the corresponding patch for 9.5

\end{frame}

\section{Get data}

\begin{frame}[fragile]
  \frametitle{By key}

  \begin{itemize}
      \item
          \begin{lstlisting}[
              basicstyle=\footnotesize\ttfamily,
              showstringspaces=false,
          ]
select '{"key": "value"}'::jsonb -> 'key';
         
          \end{lstlisting}

      \item
          \begin{lstlisting}[
              basicstyle=\footnotesize\ttfamily,
              showstringspaces=false,
          ]
select '{"key": "value"}'::jsonb ->> 'key';
         
          \end{lstlisting}

  \end{itemize}

\end{frame}

\begin{frame}[fragile]
  \frametitle{By index}

  \begin{itemize}
      \item
          \begin{lstlisting}[
              basicstyle=\footnotesize\ttfamily,
              showstringspaces=false,
          ]
select '[(*@\alert{"WAAAGH"}@*), 111]'::jsonb -> 0;
         
          \end{lstlisting}

      \item
          \begin{lstlisting}[
              basicstyle=\footnotesize\ttfamily,
              showstringspaces=false,
          ]
select '["WAAAGH", (*@\alert{111}@*)]'::jsonb -> -1;
         
          \end{lstlisting}

  \end{itemize}

\end{frame}

\begin{frame}[fragile]
  \frametitle{By path}

  \begin{itemize}
      \item[\MVRightarrow] Point out an element inside a jsonb field
      \item[\MVRightarrow] Technically, it's just an array of items
      \item[\MVRightarrow] Each element is an index (include negative values) or a key
  \end{itemize}

\end{frame}

\begin{frame}[fragile]
  \frametitle{By path}

  \begin{itemize}
      \item
          \begin{lstlisting}[
              basicstyle=\footnotesize\ttfamily,
              showstringspaces=false,
          ]
select '{
    (*@\alert{"key"}@*): {(*@\alert{"nested\_key"}@*): "value"}
}'::jsonb #> '{key, nested_key}'
         
          \end{lstlisting}

      \item
          \begin{lstlisting}[
              basicstyle=\footnotesize\ttfamily,
              showstringspaces=false,
          ]

select '{
    (*@\alert{"half\_life"}@*): [1, 2, (*@\alert{"episode one"}@*)]
}'::jsonb #> '{half_life, -1}'
         
          \end{lstlisting}

  \end{itemize}

\end{frame}

\section{Know your data}

\begin{frame}[fragile]
  \frametitle{Contains or contained?}

  \begin{itemize}
      \item
          \begin{lstlisting}[
              basicstyle=\footnotesize\ttfamily,
              showstringspaces=false,
          ]

select
    '{"a": 1, "b": 1}'::jsonb
    @>
    '{"a": 1}'::jsonb
         
          \end{lstlisting}

      \item
          \begin{lstlisting}[
              basicstyle=\footnotesize\ttfamily,
              showstringspaces=false,
          ]

select '[1]'::jsonb <@ '[1, 2]'::jsonb
         
          \end{lstlisting}


  \end{itemize}

\end{frame}

\begin{frame}[fragile]
  \frametitle{Who's there?}
  \vspace{-20pt}

  \begin{itemize}
      \item
          \begin{lstlisting}[
              basicstyle=\footnotesize\ttfamily,
              showstringspaces=false,
          ]

select '{"key": "value"}'::jsonb ? 'key';
         
          \end{lstlisting}

      \item
          \begin{lstlisting}[
              basicstyle=\footnotesize\ttfamily,
              showstringspaces=false,
          ]

select 
    '{"key": "value"}'::jsonb
    ?|
    '{key, other_key}' -- exists any
         
          \end{lstlisting}

      \item
          \begin{lstlisting}[
              basicstyle=\footnotesize\ttfamily,
              showstringspaces=false,
          ]

select 
    '{"key": "value"}'::jsonb
    ?&
    '{key, other_key}' -- exists all
         
          \end{lstlisting}

  \end{itemize}

\end{frame}

\begin{frame}[fragile]
  \frametitle{Check type}

  \begin{itemize}
      \item
          \begin{lstlisting}[
              basicstyle=\footnotesize\ttfamily,
              showstringspaces=false,
          ]

select jsonb_typeof('1'::jsonb);
         
          \end{lstlisting}

      \item
          \begin{lstlisting}[
              basicstyle=\footnotesize\ttfamily,
              showstringspaces=false,
          ]

select jsonb_typeof('"string"'::jsonb);

          \end{lstlisting}

      \item
          \begin{lstlisting}[
              basicstyle=\footnotesize\ttfamily,
              showstringspaces=false,
          ]

select jsonb_typeof('{"key": "value"}'::jsonb);

          \end{lstlisting}

  \end{itemize}

\end{frame}

\begin{frame}[fragile]
  \frametitle{jsonb\_pretty}

  \begin{lstlisting}[]
select jsonb_pretty('{"a":"test","b":[1,2,3],"c":"test3","d":{"dd":"test4","dd2":{"ddd":"test5"}}}'::jsonb);

        jsonb_pretty        
----------------------------
 {                         +
     "a": "test",          +
     "b": [                +
         1,                +
         2,                +
         3                 +
     ],                    +
     "c": "test3",         +
     "d": {                +
         "dd": "test4",    +
         "dd2": {          +
             "ddd": "test5"+
         }                 +
     }                     +
 }
(1 row)
  \end{lstlisting}

\end{frame}

\section{Manipulation with jsonb}

\begin{frame}[fragile]
  \frametitle{jsonb\_set: replace value}

  \begin{lstlisting}[
      basicstyle=\footnotesize\ttfamily,
      showstringspaces=false,
  ]
select jsonb_set(
    '{"n":null, "a":{"b": 2}}'::jsonb,
    '{n}',
    '[1,2,3]'
);

            jsonb_set                                 
-------------------------------------
{"a": {"b": 2}, "n": [1, 2, 3]}
 (1 row)
 
  \end{lstlisting}

\end{frame}

\begin{frame}[fragile]
  \frametitle{jsonb\_set: create mode by default}

  \begin{lstlisting}[
      basicstyle=\footnotesize\ttfamily,
      showstringspaces=false,
  ]
select jsonb_set(
    '{"a":{"b": 2}}'::jsonb,
    '{c}',
    '[1,2,3]'
);

            jsonb_set                                 
-------------------------------------
{"a": {"b": 2}, (*@\alert{"c": [1, 2, 3]}@*)}
 (1 row)
 
  \end{lstlisting}

\end{frame}

\begin{frame}[fragile]
  \frametitle{jsonb\_set: turn off the create mode}

  \begin{lstlisting}[
      basicstyle=\footnotesize\ttfamily,
      showstringspaces=false,
  ]
select jsonb_set(
    '{"a":{"b": 2}}'::jsonb,
    '{c}',
    '[1,2,3]',
    (*@\alert{false}@*)
);

    jsonb_set    
-----------------
 {"a": {"b": 2}}
(1 row)
 
  \end{lstlisting}

\end{frame}

\begin{frame}[fragile]
  \frametitle{jsonb\_delete: by key}

  \begin{lstlisting}[
      basicstyle=\footnotesize\ttfamily,
      showstringspaces=false,
  ]
select '{"a":1 , "b":2, "c":3}'::jsonb - 'a';

     ?column?     
------------------
 {"b": 2, "c": 3}
(1 row)

  \end{lstlisting}
\end{frame}
\note{jsonb\_delete(jsonb, text) allow to delete only top level key from object}

\begin{frame}[fragile]
  \frametitle{jsonb\_delete: by index}

  \begin{lstlisting}[
      basicstyle=\footnotesize\ttfamily,
      showstringspaces=false,
  ]
select '["a", "b", "c"]'::jsonb - 2;

  ?column?  
------------
 ["a", "b"]
(1 row)

  \end{lstlisting}
\end{frame}

\begin{frame}[fragile]
  \frametitle{jsonb\_delete: by path}

  \begin{lstlisting}[
      basicstyle=\footnotesize\ttfamily,
      showstringspaces=false,
  ]

select 
    '{"a": {"b": [1, 2, (*@\alert{3}@*)]}}'::jsonb
    #-
    '{a, b, -1}';
       ?column?       
----------------------
 {"a": {"b": [1, 2]}}
(1 row)

  \end{lstlisting}
\end{frame}

\begin{frame}[fragile]
\frametitle{jsonb\_concat ("shallow concatenation")}

  \begin{lstlisting}[
      basicstyle=\footnotesize\ttfamily,
      showstringspaces=false,
  ]
select 
    '{"a": 1, "b": [2, 3]}'::jsonb
    ||
    '{"a": 4, "c": [5, 6]}'::jsonb

              ?column?              
------------------------------------
 {"a": 4, "b": [2, 3], "c": [5, 6]}
(1 row)

  \end{lstlisting}
\end{frame}
\note{
    This operator is questionable, and I'm not sure how it will be in 9.5,
    but I think it will be present in this form in jsonbx extension
}

\begin{frame}[fragile]
\frametitle{jsonb\_concat ("shallow concatenation")}

  \begin{lstlisting}[
      basicstyle=\footnotesize\ttfamily,
      showstringspaces=false,
  ]
select 
    '{"key": {"nested_key": "value"}}'::jsonb
    ||
    '{"key": {"another_key": "another_value"}}'::jsonb

                 ?column?                  
-------------------------------------------
 {"key": {"another_key": "another_value"}}
(1 row)

  \end{lstlisting}
\end{frame}

\begin{frame}[fragile]
\frametitle{jsonb\_concat ("shallow concatenation")}
    Array will consume an object
  \begin{lstlisting}[
      basicstyle=\footnotesize\ttfamily,
      showstringspaces=false,
  ]
select 
    '{"key": "value"}'::jsonb
    ||
    '[1, 2, 3]'::jsonb 

          ?column?           
-----------------------------
 [{"key": "value"}, 1, 2, 3]
(1 row)

  \end{lstlisting}
\end{frame}

\begin{frame}[fragile]
\frametitle{jsonb\_concat ("shallow concatenation")}
    Scalars are acting like arrays
  \begin{lstlisting}[
      basicstyle=\footnotesize\ttfamily,
      showstringspaces=false,
  ]
select '"b"'::jsonb || '"a"'::jsonb

  ?column?  
------------
 ["b", "a"]

  \end{lstlisting}
\end{frame}

\section{Plans for the future}

\begin{frame}[fragile]
  \frametitle{Array-style subscripting}

  \begin{itemize}
      \item
      \begin{lstlisting}[
          basicstyle=\footnotesize\ttfamily,
          showstringspaces=false,
      ]
update some_table
    set jsonb_data['a']['a1']['a2'] = 42;

      \end{lstlisting}

      \item
      \begin{lstlisting}[
          basicstyle=\footnotesize\ttfamily,
          showstringspaces=false,
      ]
select jsonb_data['a']['a1']['a2']
    from some_table;

      \end{lstlisting}
  \end{itemize}

\end{frame}

\begin{frame}[fragile]
  \frametitle{New functions and improvements}

  There are still missing functionality and improvements, that can be useful for JSONB.
  Some of them will be implented as parts of jsonbx extension (for 9.5), and will be proposed for 9.6
\end{frame}

\begin{frame}[fragile]
  \frametitle{jsonb\_delete}

  \begin{lstlisting}[
      basicstyle=\footnotesize\ttfamily,
      showstringspaces=false,
  ]
select jsonb_delete_jsonb(
    '{"a": 1, "b": {"c": 2}, (*@\alert{"f": [4, 5]}@*)}'::jsonb,
    '{"a": 4, (*@\alert{"f": [4, 5]}@*), "c": 2}'::jsonb
);

      jsonb_delete
----------------------------
  {"a": 1, "b": {"c": 2}}
(1 row)

  \end{lstlisting}

\end{frame}

\begin{frame}[fragile]
  \frametitle{jsonb\_intersection}

  \begin{lstlisting}[
      basicstyle=\footnotesize\ttfamily,
      showstringspaces=false,
  ]
select jsonb_intersection(
    '{(*@\alert{"a":2}@*), "d": {"f": 3}, (*@\alert{"g":[4, 5]}@*)}'::jsonb,
    '{(*@\alert{"a":2}@*), "f": 3, (*@\alert{"g":[4, 5]}@*)}'::jsonb
);

   jsonb_intersection
--------------------------
  {"a": 2, "g": [4, 5]}
(1 row)

  \end{lstlisting}

\end{frame}

\begin{frame}[fragile]
  \frametitle{jsonb\_deep\_merge}

  \begin{lstlisting}[
      basicstyle=\footnotesize\ttfamily,
      showstringspaces=false,
  ]
select jsonb_deep_merge(
    '{"a": (*@\alert{\{"b":1\}}@*)}'::jsonb,
    '{"a": (*@\alert{\{"c":2\}}@*)}'::jsonb
);

     jsonb_deep_merge            
-------------------------------
   {"a": (*@\alert{\{"b":1, "c":2\}}@*)}
(1 row)

  \end{lstlisting}

\end{frame}
\note{
    Form of this function is questionable,
    but I think it will be present in this form in jsonbx extension
}

\section{Finish}

\begin{frame}{Contacts}
    \begin{itemize}[label={}]
        \item \href{https://github.com/erthalion/jsonbx}{\github\ github.com/erthalion/jsonbx}
        \item \href{http://twitter.com/erthalion}{\twitter\ @erthalion}
        \item \email\ 9erthalion6 at gmail dot com
    \end{itemize}
\end{frame}

\plain{Questions?}

\end{document}
